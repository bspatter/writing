\section{Conclusions}
\label{sec:usbe_lung_conclusions}
This work is unique in that we propose a previously unconsidered
potential damage mechanism for \ac{DUS}-induced \ac{LH}. We
hypothesize that baroclinic torque occurs at fragile air-tissue
interfaces of the lung due to misalignment between the \ac{US}
pressure gradient and material interface density gradient, causing
stress, deformation, and ultimately rupture at the interface. This
mechanism arises as a result of nonlinear, compressible fluid
mechanics, and cannot be predicted through traditional linear
acoustics. We suggest that nonlinear effects such as baroclinic
vorticity are important to this problem because of the sharp density
discontinuities between air and tissue within the lungs. To
investigate our hypothesis we develop a numerical model of \ac{DUS}
wave-alveolus interaction and simulate the physics underlying
acoustically driven, perturbed liquid-gas interfaces.

We aim to investigate three specific questions presented in Section
\ref{sec:usbe_lung_introduction}. To address these questions, we
enumerate three conclusions of this work based on our results.  First,
acoustic waves can generate vorticity at perturbed liquid-gas
interfaces as a result of baroclinic torque. Second, this vorticity is
capable of appreciably deforming the interface in the inviscid,
inelastic case. We note that at this preliminary stage, additional
simulations, run for longer time, will be useful in solidifying this
conclusion.  Third, acoustic properties relevant to \ac{DUS} including
acoustic amplitude, wave duration, and repetition frequency, are
important to circulation deposition and subsequently any
circulated-driven deformation or hemorrhage. For the case of a simple
trapezoidal pressure wave, we demonstrate that the amount of
circulation deposited along the interface scales linearly with the
acoustic pressure amplitude. More subtly, because the interface is
deforming throughout its interaction with the wave, affecting the
alignment of pressure and density gradients, the acoustic wave
duration can play an important role in determining how much
circulation is ultimately deposited. Because the deformation can
continue long after the wave has gone, similar arguments can be made
for timing between subsequent waves. With regard to ultrasound, this
could play an important role in choosing optimal \acs{PD} and
\acs{PRF}.
%%% Local Variables:
%%% mode: latex
%%% TeX-master: "../../prelim"
%%% End:
