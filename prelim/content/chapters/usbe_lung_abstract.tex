\subsection{Abstract}
Over the past few decades, \ac{DUS} of the lung has been shown to
cause hemorrhage in a variety of mammals, though the underlying damage
mechanism is yet to be determined. While there do not appear to be
serious health risks associated with this problem under typical
clinical conditions, the use of \ac{DUS} for imaging of the lung is
increasing rapidly. It is important we understand this phenomena to
ensure that lung \ac{DUS} remains safe as new procedures and
technologies are developed. In this work we investigate the underlying
physics associated with acoustic waves and liquid-gas interfaces and
propose a previously unconsidered physical damage mechanism for
\ac{DUS}-induced \ac{LH}. Specifically we propose that misalignments
between ultrasound pressure gradients and tissue-air interface density
gradients result in the generation of baroclinic vorticity, which
could drive fragile cellular barriers around the alveoli to deform and
ultimately hemorrhage. To investigate our hypothesis, we treat the
lung as an inviscid, compressible fluid system and develop a
simplified, numerical model of the problem to simulate \ac{DUS}
pulse-alveolus interaction. We show that acoustic waves, such as
\ac{DUS} pulses, are capable of generating baroclinic vorticity at
sharp liquid-gas interfaces such as those found in the lungs, and that
this drives subsequent deformation of the interface. We perform
analysis to describe the vorticity and interface dynamics and propose
a scaling law based on dimensional analysis to predict the growth of a
purely circulation driven interface. We compare predicted results with
numerical experiments to verify that baroclinic vorticity is the
mechanism responsible for the deformation. 

Finally we suggest future work to be completed for this dissertation
in the upcoming year. We plan to increase the relevance of this work
to lung \ac{US} by using more realistic geometries and computing
theoretical stresses within the lungs. Additionally we will model the
circulation and interface deformation associated with simple expansion
and compression waves and use these models to design optimal waveforms
for minimizing interface growth after the passage of the wave.

%%% Local Variables:
%%% mode: latex
%%% TeX-master: "../../prelim"
%%% End:
