In this chapter we present work in which experimentally-measured
\ac{US} pulses are used to simulate \ac{US} contrast agent microbubble
dynamics. The pulses were previously used in experiments to determine
capillary breaching thresholds in rat kidneys \citep{Miller2008b}. We
compare the calculated bubble dynamics to the
experimentally-determined bioeffects thresholds to investigate the use
of theoretical \ac{IC} thresholds as a predictor for bioeffects. This
work was published in the Journal of the Acoustical Society of America
\citep{Patterson2012, Patterson2012a}. Here we present the abstract, key figures, and
conclusions of the published work.
%
\section{Abstract}
  In order to predict bioeffects in contrast-enhanced diagnostic and
  therapeutic ultrasound procedures, the dynamics of cavitation
  microbubbles in viscoelastic media must be determined.  For this
  theoretical study, measured 1.5-7.5 MHz pulse pressure waveforms,
  which were used in experimental determinations of capillary
  breaching thresholds for contrast-enhanced diagnostic ultrasound in
  rat kidney, were used to calculate cavitation nucleated from
  contrast agent microbubbles.  A numerical model for cavitation in
  tissue was developed based on the Keller-Miksis equation (a
  compressible extension of the Rayleigh-Plesset equation for
  spherical bubble dynamics), with a Kelvin-Voigt constitutive relation. From
  this model, the bubble dynamics corresponding to the experimentally
  obtained capillary breaching thresholds were determined. Values of
  the maximum radius and temperature corresponding to previously
  determined bioeffect thresholds were computed for a range of
  ultrasound pulses and bubble sizes for comparison to inertial
  cavitation threshold criteria.  The results were dependent on
  frequency, the gas contents, and the tissue elastic properties.  The
  bioeffects thresholds were above previously determined inertial
  cavitation thresholds, even for the tissue models, suggesting the
  possibility of a more complex dosimetry for capillary injury in
  tissue.


%%% Local Variables:
%%% mode: latex
%%% TeX-master: t
%%% End:


\section{Key figures}
\label{sec:usbe_bubble_key_figures}
%\subsection{Bubble Response}
In this section we present key figures from \cite{Patterson2012a}. The
results presented are based on simulations of microbubbles in a Voigt
viscoelastic medium as modeled by \cite{Yang2005}. Experimentally
determined input pressure waveforms and the associated bioeffects are
based on the work performed in \cite{Miller2008b}.

To illustrate typical bubble responses, Figures
\ref{fig:sample_bubble_linear} and \ref{fig:sample_bubble_nonlinear}
show sample experimental input pressure waveforms from
\citep{Miller2008b} and the calculated bubble radius histories
corresponding to each. Both simulations are for the case of the wave
impinging upon an initially $R_0=1 \mu$m radius microbubble in Voigt
viscoelastic media with shear moduli, $G=5$ kPa, $100$ kPa, and $1$
MPa as indicated. Figure \ref{fig:sample_bubble_linear} represents an
essentially linear case with \ac{US} center frequency 1.5 MHz and 0.35
MPa \ac{PRPA}, in which no bioeffects were observed. Figure
\ref{fig:sample_bubble_nonlinear} represents a highly nonlinear case
with 7.5 MHz and 6.0 MPa \ac{PRPA}. In bioeffects, in the form of
bleeding on the surface of the rat kidney, were observed
\cite{Miller2008b}.
\begin{figure}%[h!]
  \centering
  \includegraphics[width=0.66\textwidth]{./figs/bubble_figs/rt_linear}
  \caption[Bubble radius history and input-pressure for an essentially linear case]{History of the bubble radius (top) and input-pressure
    waveform (bottom) for an essentially linear case (frequency: 1.5 MHz; \ac{PRPA}: 0.35 MPa). No bioeffects are observed
    here. $R_0=1$ $\mu$m; solid: $G=5$ kPa; dashed: $G=100$ kPa; dotted: $G=1$ MPa.}
  \label{fig:sample_bubble_linear}
\end{figure}
%
\begin{figure}%[h!]
  \centering \includegraphics[width=0.66\textwidth]{./figs/bubble_figs/rt_nonlinear}
  \caption[Bubble radius history and input-pressure for a nonlinear case]{History of the bubble radius (top) and input-pressure
    waveform (bottom) for a highly nonlinear case (frequency: 
    7.5 MHz; peak negative pressure: 6.0 MPa). Bioeffects are observed
    here. $R_0=1$ $\mu$m; solid: $G=5$ kPa; dashed: $G=100$ kPa; dotted: $G=1$ MPa.}
  \label{fig:sample_bubble_nonlinear}
\end{figure}

To study the dependence of the bubble dynamics on the gas content, we
compare results obtained for bubbles containing \ac{PFP} and air.
Figure \ref{fig:gascontents} shows sample bubble radii histories for
each gas and the corresponding input pressure wave. Additionally, we
plot of the maximum temperature for \ac{PFP} (circles) and air
(squares), calculated based on isentropic relationships, for bubbles
with initial radii $R_0=0.1-2$ $\mu$m exposed to ultrasound
frequencies 1.5 - 7.5 MHz and \ac{PRPA} 0.35 - 6 MPa.
\begin{figure*}%[h!]
  \includegraphics[width=0.47\textwidth]{./figs/bubble_figs/pfpair}% }
  \includegraphics[width=0.47\textwidth]{./figs/bubble_figs/tmaxpfpair}% }
  \caption[Dependence of the bubble dynamics on the gas contents]{ Dependence of the bubble dynamics on the gas contents ($G=100$ kPa). (Left) History of the bubble radius for \ac{PFP} (solid) 
    and air (dashed). (Right) Maximum temperature for \ac{PFP} (circles) and air (squares). $R_0=0.1-2$ $\mu$m; frequency: 1.5 - 7.5 MHz. }
  \label{fig:gascontents}
\end{figure*}

To illustrate the dependence of the bubble response on the pulse
frequency Figure \ref{fig:freq} shows the maximum dimensionless and
dimensional radius for all initial bubble sizes and amplitudes vs
frequency. The square symbols denote cases in which bioeffects were
observed in the experiments, while the circular symbols represent no
bioeffects. The initial bubble sizes are not discriminated here for
simplicity. With the exception of a few outliers, a clear separation
between cases for which bioeffects did and did not occur is observed;
in other words, the bioeffect threshold has a strong dependence on the
frequency. The trend appears to be approximately linear with
frequency. Large growth may be achieved with no evident bioeffects,
especially at high frequencies. The quantity $R_{max}$ is a measure of
cavitation collapse since it is related to the available energy of the
bubble. Thus, the present results indicate that cavitation collapse is
expected to play an important role regarding bioeffects, although the
precise mechanism cannot be inferred. Again, the existing criteria for
inertial cavitation thresholds are frequency-independent and do not
correlate well with the bioeffects threshold, which clearly shows a
strong dependence on frequency.
\begin{figure*}%[h!]
  \includegraphics[width=0.47\textwidth]{./figs/bubble_figs/rstarmax_f}  
  \includegraphics[width=0.47\textwidth]{./figs/bubble_figs/rmax_f}      
  \caption[Dependence of the bubble dynamics on the ultrasound frequency]{ Dependence of the bubble dynamics on the frequency for
    $G=100$ kPa. $R_0=0.1-2$ $\mu$m; empty circles: no bioeffects; squares:
    bioeffects. (Left) Dimensionless (Left) and dimensional (Right) maximum bubble radius.}
  \label{fig:freq}
\end{figure*}

To explore the effect of the elasticity on the results and the
correlation to bioeffects, Figure \ref{fig:freq_tissue} shows the
maximum dimensionless radius for all initial bubble sizes and
amplitudes vs frequency for $G=5$ kPa and $G=1$ MPa. Although
seemingly high, the latter elasticity is chosen to match the work of
\begin{figure*}%[h!]
  \includegraphics[width=0.47\textwidth]{./figs/bubble_figs/rstarmax_f_ca=20}
  \includegraphics[width=0.47\textwidth]{./figs/bubble_figs/rstarmax_f_ca=0,1}    
  \caption[Dependence of the dimensionless maximum bubble radius on
    the ultrasound frequency]{ Dependence of the dimensionless maximum bubble radius on
    the frequency for $G=5$ kPa (Left) and $G=1$ MPa (Right). $R_0=0.1-2$ $\mu$m; empty circles: no bioeffects; squares:
    bioeffects. }
  \label{fig:freq_tissue}
\end{figure*}

% \begin{figure*}[h!]
%   \subfigure[$G=5$ kPa.]{
%     \includegraphics[width=0.47\textwidth]{./figs/bubble_figs/rstarmax_f_ca=20}
%   }
%   \subfigure[$G=1$ MPa.]{
%     \includegraphics[width=0.47\textwidth]{./figs/bubble_figs/rstarmax_f_ca=0,1}    
%   }
%    \caption{ Dependence of the dimensionless maximum bubble radius on
%      the frequency. $R_0=0.1-2$ $\mu$m; empty circles: no bioeffects; squares:
%      bioeffects.}
%   \label{fig:freq_tissue}
% \end{figure*}
\clearpage
\pagebreak
\section{Conclusions}
\label{sec:conclusions}

In the present work, a numerical model is used
to investigate experimentally observed bioeffects as a result of
contrast-enhanced ultrasound. This work is unique in its 
combination of experimental results and numerical modeling.
For the experimentally generated input
pressure waveforms, it is known which of these triggered bioeffects,
and from the numerical model we obtained calculated values for
the dimensionless maximum radius and dimensional maximum temperature for each of these cases.  By comparing the
results of this study to previously established inertial cavitation
thresholds used by \cite{Apfel1991} and \cite{Yang2005},
$T_{max}=5000$ K and $R_{max}=2$, it would appear that the inertial
cavitation threshold does not play a role in determining the bioeffects
threshold.  However, it is unlikely that the inertial cavitation
threshold is irrelevant. Instead, it is far more probable that these
thresholds are not defined appropriately for cavitation in a
viscoelastic medium, such as soft tissue. This work suggests the need for
further experimental and numerical studies of cavitation in viscoelastic media.

The present work shows a strong correlation between cavitation dynamics and bioeffects
when considering the pulse frequency.
From the plot of maximum
dimensionless radius vs. frequency, there is a clear separation
between when bioeffects do and do not occur, and based on these
results it appears that the frequency of the input pressure waveforms
is of key importance to the definition of a bioeffect threshold, and
likely the inertial cavitation threshold as well. 

The present work shows that the elasticity of tissue significantly
affects the bubble dynamics. This finding is perhaps not completely
unexpected given that bubble dynamics are known to strongly depend
on viscoelastic properties and model. The present study shows the need
for more accurate measurements of material properties and for
determining appropriate constitutive models for soft tissue,
particularly at high strain rates. Finally, although the present work
suggests that inertial cavitation collapse plays an important role with respect
to bioeffects, it does not shed light on the exact mechanism,
\emph{e.g.}, shock emission upon collapse, growth beyond a given size,
high temperatures generating free radicals, re-entrant jets in
non-spherical collapse, etc.  In future work we plan on investigating 
this injury mechanism by conducting direct simulations of
the full equations of motion for bubble dynamics in a viscoelastic medium.


%%% Local Variables:
%%% mode: latex
%%% TeX-master: "../../prelim"
%%% End:
