\section{Abstract}
  In order to predict bioeffects in contrast-enhanced diagnostic and
  therapeutic ultrasound procedures, the dynamics of cavitation
  microbubbles in viscoelastic media must be determined.  For this
  theoretical study, measured 1.5-7.5 MHz pulse pressure waveforms,
  which were used in experimental determinations of capillary
  breaching thresholds for contrast-enhanced diagnostic ultrasound in
  rat kidney, were used to calculate cavitation nucleated from
  contrast agent microbubbles.  A numerical model for cavitation in
  tissue was developed based on the Keller-Miksis equation (a
  compressible extension of the Rayleigh-Plesset equation for
  spherical bubble dynamics), with a Kelvin-Voigt constitutive relation. From
  this model, the bubble dynamics corresponding to the experimentally
  obtained capillary breaching thresholds were determined. Values of
  the maximum radius and temperature corresponding to previously
  determined bioeffect thresholds were computed for a range of
  ultrasound pulses and bubble sizes for comparison to inertial
  cavitation threshold criteria.  The results were dependent on
  frequency, the gas contents, and the tissue elastic properties.  The
  bioeffects thresholds were above previously determined inertial
  cavitation thresholds, even for the tissue models, suggesting the
  possibility of a more complex dosimetry for capillary injury in
  tissue.


%%% Local Variables:
%%% mode: latex
%%% TeX-master: t
%%% End:
