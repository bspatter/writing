\section{Analysis}%
\label{sec:analysis}%
We perform analysis to make quantifiable predictions about the
vorticity generation and interface growth. The results of these
analyses are compared with the results of our numerical experiments in
section \ref{sec:results}.

To better understand the source of circulation within our problem we
look to the vorticity generation equation for a 2D inviscid fluid
system,
\begin{align} \label{eq:vorticity_euler}
  \frac{\partial \vec{\omega}}{\partial t}+\left(\vec{u}\cdot\nabla\right)\vec{\omega} =% 
  - \vec{\omega}\left(\nabla\cdot\vec{u}\right) + \frac{\nabla\rho\times\nabla p}{\rho^2}.%
\end{align}
Each term in equation \eqref{eq:vorticity_euler} represents a
different physical mechanism by which the vorticity $\vec{\omega}$ is
changing, with the terms on the left-hand side represent changes in
the existing vorticity field and the terms on the right representing
vorticity sources and sinks. The first term on the left represents the
total change of vorticity at a location in the flow field with respect
to time. The second term on the left represents the advection of
vorticity within the field. The first term on the right describes
changes in vorticity due to compressibility in the flow. The last term
on the right is the baroclinic term which describes vorticity
generated by the misalignment of the pressure and density gradients in
the flow. We seek to understand the relative importance of these
mechanisms on the dynamics of the acoustically accelerated interface.

\subsection{Order of magnitude analysis of vorticity generation mechanisms}
To quantifiably compare the various mechanisms by which vorticity
changes within the flow, we recognize that any vorticity generated
must be a result of acoustic energy being converted to kinetic energy
within the flow. As the only mechanism for this to occur in an
inviscid fluid without pre-existing vorticity is baroclinic, we
require misaligned density and pressure gradients. Thus we choose to
perform our analysis at the water-air interface during the period
which is interacting with the incoming wave. For simplicity, we narrow
this down even further to only consider the period in which the
incoming compressive portion of the wave encounters the interface. As
this interaction occurs quickly, over an approximate time span
$\Delta t_a\approx5\lambda/c_{w}$, we assume that the interface is
static and remains undeformed from its initial state during this
interaction. We will show in section \ref{sec:results}, this is a
reasonable assumption for this period. Having now established the
point at which the analysis is to be performed we evaluate the order
of magnitude of compressible, advective, and baroclinic terms of the
vorticity generation equation \eqref{eq:vorticity_euler}. We note that
the advective term is not a true source of vorticity, but is useful in
understanding the change of vorticity at any given time and location
within the flow.

In our evaluation of the individual terms of the vorticity generation
equation \eqref{eq:vorticity_euler}, we treat gradient, curl, and
divergence terms of any arbitrary quantity $f$ such that
$\nabla f= \orderof{\left|\Delta f\right|/\Delta L}$,
$\nabla\cdot f=\orderof{\left|\Delta f\right|/\Delta L}$, and
$\nabla\times f=\orderof{\left|\Delta f\right|/\Delta L}$. Here
$\Delta f$ is a change in $f$ over a characteristic length scale
$\Delta L$. Because the only motion in the flow is generated by the
acoustic wave. Accordingly, we consider acoustic pressure, velocity,
and density perturbations such that $\Delta p=\Delta p_a$,
$\Delta \vec{u}=\Delta \vec{u}_a$, and $\Delta \rho=\Delta \rho_a$,
and use acoustic relations to relate these quantities
\citep{Anderson1990},
\begin{align}%
  \label{eq:acoustic_relations}%
  \Delta p_a=\pm\Delta u_a \rho c=c^2\Delta \rho_a%
\end{align}
Additionally, for the values presented in this section we consider our base
trapezoidal wave case where $p_a = \Delta p_a = 10$ MPa. The length
scale associated with the acoustic wave is the initial length of the
pressure rise $\Delta L_a=5\lambda$. The initial interface length
scale $\Delta L_I$, defined as the thickness of the thickness of the
mixed layer from 0.05 to 0.95 volume fraction is estimated as
$\Delta L_I \approx 0.05\lambda$. We approximate the order of theta
based on its average value along a half-wavelength of the interface
for $a_0=0.03\lambda$ such that $\overline{\abs{\theta}}\approx0.12$.

To assess the baroclinic contribution to vorticity, we write the cross
product of the density and pressure gradients as
$\abs{\nabla \rho} \abs{\nabla p} \sin{\left(\theta\right)}$. Here
$\theta$ is the angle between the acoustic pressure gradient, treated
as being in the $\plus y$-direction, and the direction of the density
gradient which we treat as the outward normal direction to the
interface. For $a_0/\lambda<<1$, we can approximate
$\sin{\left(\theta\right)}\approx\theta$ at the interface. The density
gradient due to the water-air interface is far greater than that due
to the acoustic wave. As such we use the change in density across the
interface $\Delta \rho_I$ and associated length scale $\Delta L_I$ to
write the density gradient. The pressure change is a result of the
acoustic wave, and as such we use the acoustic pressure change
$\Delta p_a$ and associated length scale $\Delta L_a$ to express the
pressure gradient. And thus we write the order of magnitude of the
baroclinic vorticity generation term at the interface,
\begin{align}
  \label{eq:baroclinic_vorticity}%
  \norm{\frac{\nabla\rho\times\nabla p}{\rho^2}} = \orderof{\frac{\abs{\Delta \rho_I}}{\abs{\Delta L_I}}\frac{\abs{\Delta p_a}}{\abs{\Delta L_a}}\frac{1}{\abs{\rho}^2}\abs{\theta}}.%
\end{align}

In the evaluation of the compressible and advective terms we consider
two possible evaluations of the vorticity $\omega$ as either the curl
of the acoustic velocity field $\vec{\omega}=\nabla\times\vec{u}$ or
the integral of the baroclinic vorticity generation term from
\eqref{eq:baroclinic_vorticity}, treated as constant, over the
characteristic time of the pressure rise
$\Delta t_a\approx\Delta L_a/c_w$. Evaluating the vorticity using both
of these expressions and the previously presented values reveals that
the baroclinic expression of the vorticity is dominant for
the regimes of interest to this study and is thus what we will focus
on in our further analysis. Hence we write an expression for the
approximate order of magnitude of the compressible and advective
contribution to vorticity as
\begin{align}
  \label{eq:compressible_advective_vorticity}%
\norm{-\vec{\omega}\left(\nabla\cdot\vec{u}\right)}\sim \norm{\left(\vec{u}\cdot\nabla\right)\vec{\omega}} = %
\orderof{\frac{\abs{\Delta u_a}}{\abs{\Delta L_a}} \frac{\abs{\Delta \rho_I}}{\abs{\Delta L_I}}\frac{\abs{\Delta p_a}}{\abs{\Delta L_a}}\frac{1}{\abs{\rho}^2}\abs{\theta}\frac{\abs{c}}{\abs{\Delta L_a}}}.%
\end{align}

Now, to compare the relative importance of the baroclinic and
compressible (or advective) contributions to vorticity we will look at
the ratio of the two vorticity generation approximations. We divide
equation \eqref{eq:baroclinic_vorticity} by equation
\eqref{eq:compressible_advective_vorticity} use
\eqref{eq:acoustic_relations} to express acoustic quantities in terms
of the acoustic perturbation quantities $\Delta \rho_a,\,\Delta u_a$ and simplify,
\begin{align} \label{eq:vorticity_comparison}
\frac{\norm{\frac{\nabla\rho\times\nabla p}{\rho^2}}}{\norm{-\vec{\omega}\left(\nabla\cdot\vec{u}\right)}} = \frac{c}{\abs{\Delta u_a}} = \frac{\rho}{\abs{\Delta \rho_a}}%
\end{align}
Evaluating this expression we expect that the relative contribution of
baroclinic to compressible/advective vorticity generation is
approximately of order $\orderof{10^2}$ at
$t=\Delta t_a=1.05$. Hence we expect that the total circulation within
the left and right halves of domain to be nearly entirely
baroclinically generated, and thus we will neglect vorticity generated
through other mechanisms as we continue with our analysis.

\subsection{Comparison of vorticity generation in air and water}
Having established that the dominant source of vorticity is
baroclinicity we now aim to determine where this vorticity will
generated within the mixed region of the interface. Specifically, we
aim to compare the order of baroclinic vorticity generation from
equation \eqref{eq:baroclinic_vorticity} in pure water and air. As
this can already be evaluated in water from what we have provided up
to this point, we will focus on evaluation of the order of baroclinic
vorticity generation in air, from equation
\eqref{eq:baroclinic_vorticity}. Throughout the analysis we will
denote the properties of the incoming wave and water with a subscript
$-$, and the transmitted wave and air with a subscript $+$. For water,
we will use the values for
$\Delta \rho_I, \Delta L_I, \Delta \rho_a, \Delta L_a$ and $\theta$
defined in the previous section based on our initial condition. Our
treatment of the density gradient at the interface will remain
unchanged for evaluation in air such that
$\Delta \rho_I^-=\Delta \rho_I^+$ and $\Delta L_I^-=\Delta L_I^+$.

To evaluate the remaining terms in air we will borrow techniques from
linear acoustics. To find the pressure change in the transmitted wave
$\Delta p_a^+$, we recognize that $a_0/\lambda<<1$ and treat the
incoming wave as a plane wave impinging normally on a flat material
interface such that $\Delta p_a^+=\bs{T} \Delta p_a^-$, where $\bs{T}$
is the acoustic transmission coefficient,
$\bs{T}=2\rho^+ c^+/\left(\rho^+ c^+ + \rho^- c^- \right)$
\citep{Kinsler1982}. For our water-air interface
$\bs{T}\approx4.97\times10^{-4}$. Because of the strong impedance
mismatch between fluids, the acoustic wave is almost entirely
reflected, decreasing the pressure gradient in the air. Because of the
drop in sound speed across the interface, the transmitted wave is
compressed into a smaller physical area (i.e., the wavelength
decreases) relative to the incoming wave, such that
$\Delta L_a^+=\Delta L_a^- (c^+/c^-)$. This effect increases the
pressure gradient in the air. To evaluate $\theta^+$, we utilize
Snell's law which states that
$c^-\sin{\theta^-}=c^+\sin{\theta^+}$. Because $a_0/\lambda<<1$ it is
also true that $\theta^-<<1$, thus we use the small angle
approximation of $\sin$ to find that
$\theta^+\approx\theta^-(c^+/c^-)$. We note that this decreases the
misalignment between the pressure and density gradients in air, and
quantitatively approximately cancels the increase in pressure gradient
due to the decrease in wavelength of the transmitted wave.

To determine where the vorticity will be generated at the interface,
we consider equation \eqref{eq:baroclinic_vorticity} in air and water
and write the ratio to find
\begin{align}%
\label{eq:baroclinic_air_water}%
%\left(\frac{\partial\omega}{\partial t}\right)_{\substack{\text{baroclinic}\\\text{air}}} / \left(\frac{\partial\omega}{\partial t}\right)_{\substack{\text{baroclinic}\\\text{water}}}%
\frac{\norm{\frac{\nabla\rho\times\nabla p}{\rho^2}}_{air\quad}}{\norm{\frac{\nabla\rho\times\nabla p}{\rho^2}}_{water}}
=&\orderof{\frac{\left[\frac{\abs{\Delta \rho_I^+}}{\abs{\Delta L_I^+}}\frac{\abs{\Delta p_a^+}}{\abs{\Delta L_a^+}}\frac{1}{\abs{\rho^+}^2}\abs{\theta^+}\right]}
{\left[\frac{\abs{\Delta \rho_I^+}}{\abs{\Delta L_I^+}}\frac{\left(\abs{\Delta p_a^+}/\abs{\bs{T}}\right)}{\abs{\Delta L_a^+}\left(\abs{c^+}/\abs{c^-}\right)}\frac{1}{\abs{\rho^-}^2}\left(\abs{c^+}/\abs{c^-}\right)\abs{\theta^+}\right]}},\nonumber\\%
=&\orderof{\abs{\bs{T}}\left(\frac{\abs{\rho^-}}{\abs{\rho^+}}\right)^2}.%
\end{align}
For our water-air interface, we evaluate equation
\eqref{eq:baroclinic_air_water} to find that the ratio of baroclinic
vorticity generation in air to that in water would be of order
$\orderof{10^2}$. While this result considers vorticity generation in
pure air and water, as opposed to the mixed fluid region relevant to
this work, it provides a useful upper bound on the change we expect in
the vorticity across the interface. Additionally, this result suggests
that for the mixed water-air region, where the strongest density
gradient exists, vorticity generation is likely to occur in areas with
a lower volume fraction of water.

\subsection{Considerations of circulation}
In order to verify our analyses numerically we will consider not the
vorticity generation, but rather the circulation as a function of
time. As circulation is a global quantity of vorticity over a region,
it is more practical to compare to our numerical experiments. The
expressions previously obtained for estimates of vorticity generation
can be integrated in space to obtain integral expressions for
circulation generation. As the expressions derived were approximate
and spatially independent, we expect that the approximate vorticity
relationships found in this section will remain relevant in
considerations of the circulation.  For instance, based on the results
of equation \eqref{eq:vorticity_comparison} we expect the baroclinic
circulation generation to be $\orderof{10^2}$ larger than the
compressible and advective terms toward the end of interaction between
the interface and the acoustic compression.

Finally, as we expect the interface growth to be purely circulation driven
long after all waves have left the domain, we perform dimensional
analysis to find a scaling law for the corresponding interface
perturbation amplitude $a(t)$ as a function of circulation and time,
\begin{align} \label{eq:intf_circ_scaling}
  a(t) \sim \sqrt{\Gamma t}.
\end{align}
This proposed scaling law will be compared to the late time dynamics
of the interface, after the acoustic wave has left the domain in
Section \ref{subsubsec:amplitude_dependence}.


% We integrate equation
% \eqref{eq:vorticity_euler} over the half-domain, $A_R$, to get

% \begin{align} \label{eq:circulation_generation}
%   \left(\frac{\partial \Gamma}{\partial t}\right)_{total} =
%   \left(\frac{\partial \Gamma}{\partial t}\right)_{compressible} + \left(\frac{\partial \Gamma}{\partial t}\right)_{baroclinic} - \left(\frac{\partial \Gamma}{\partial t}\right)_{advective},
% \end{align}

% Each term will be analyzed separately to determine the individual
% physical contributions to circulation. Here

% \addtocounter{equation}{-1}
% \begin{subequations}\label{eq:circulation_generation_components}
%   \begin{align}
%     &\left(\frac{\partial \Gamma}{\partial t}\right)_{compressible} &=& -\int_{A_R} \vec{\omega}\left(\nabla\cdot\vec{u}\right) \, dA_R,&\\
%     &\left(\frac{\partial \Gamma}{\partial t}\right)_{baroclinic} &=& +\int_{A_R} \frac{\nabla\rho\times\nabla p}{\rho^2} dA_R,&\\
%     &\left(\frac{\partial \Gamma}{\partial t}\right)_{advective} &=& +\int_{A_R} \left(\vec{u}\cdot\nabla\right)\vec{\omega} \, dA_R.&
%   \end{align}
% \end{subequations}
%

% as we expect the interface growth to be purely circulation
% driven long after all waves have left the domain, we perform
% dimensional analysis to find a scaling law for the corresponding
% interface perturbation amplitude $a(t)$ as a function of circulation
% and time,

% \begin{align} \label{eq:intf_circ_scaling}
%   a(t) \sim \sqrt{\Gamma t}.
% \end{align}

% This proposed scaling law will be compared to the late time dynamics
% of the interface, after the acoustic wave has left the domain in
% Section \ref{subsubsec:amplitude_dependence}.




% \subsection{old}
% above in our results, we
% will look for two things. First, as the above analysis suggests
% circulation generated during the compression wave-interface
% interaction is predominantly baroclinically generated. Because our
% acoustic pressure is linearly-increasing we predict that circulation
% deposited during this will also increase linearly with maximum
% acoustic pressure $p_a$, i.e.,
% \begin{align} \label{eq:linear_circulation}
%   \Gamma \sim \norm{\frac{\nabla \rho\times\nabla p}{\rho^2}} \sim p_a.
% \end{align}

% Second, to numerically verify our predictions for the types of
% vorticity generated in a visualizable way, we integrate the vorticity
% generation equation \eqref{eq:vorticity_euler} over the
% half-domain. 
% %
% \begin{align} \label{eq:circulation_generation}
%   \left(\frac{\partial \Gamma}{\partial t}\right)_{total} = \left(\frac{\partial \Gamma}{\partial t}\right)_{compressible} + \left(\frac{\partial \Gamma}{\partial t}\right)_{baroclinic} - \left(\frac{\partial \Gamma}{\partial t}\right)_{advective},
% \end{align}
% %
% Each term will be analyzed separately to determine the individual
% physical contributions to circulation at any point time. Here
% %
% \addtocounter{equation}{-1}
% \begin{subequations}\label{eq:circulation_generation_components}%
%   \begin{align}% 
%     &\left(\frac{\partial \Gamma}{\partial t}\right)_{compressible} &=& -\int_{A_R} \vec{\omega}\left(\nabla\cdot\vec{u}\right) \, dA_R,&\\
%     &\left(\frac{\partial \Gamma}{\partial t}\right)_{baroclinic} &=& +\int_{A_R} \frac{\nabla\rho\times\nabla p}{\rho^2}, dA_R,&\\
%     &\left(\frac{\partial \Gamma}{\partial t}\right)_{advective} &=& +\int_{A_R} \left(\vec{u}\cdot\nabla\right)\vec{\omega} \, dA_R,&
%   \end{align}
% \end{subequations}
% %

% Finally, as we expect the interface growth to be purely circulation
% driven long after all waves have left the domain, we perform
% dimensional analysis to find a scaling law for the corresponding
% interface perturbation amplitude $a(t)$ as a function of circulation
% and time,
% %
% \begin{align} \label{eq:intf_circ_scaling}%
%   a(t) \sim \sqrt{\Gamma t}.
% \end{align}
% %
% This proposed scaling law will be compared to the late time dynamics
% of the interface, after the acoustic wave has left the domain in
% Section \ref{subsubsec:amplitude_dependence}.

%%% Local Variables:
%%% mode: latex
%%% TeX-master: "../main"
%%% End:



