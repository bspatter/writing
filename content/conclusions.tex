\section{Summary and Conclusions}
\label{sec:conclusions}
This paper demonstrates that acoustic waves may trigger significant
deformation of perturbed liquid-gas interfaces over long periods of
time. The driving mechanism behind this deformation is baroclinic
vorticity, which occurs as a result of misalignment between the
pressure gradient of the acoustic wave and density gradient of the
perturbed interface. To demonstrate this we simulate trapezoidal
acoustic waves of MPa-order amplitude impinging from water onto
air.

The work presented here supports the following three conclusions: (1)
Acoustically-generated baroclinic vorticity is capable of
significantly deforming perturbed liquid-gas interfaces. We observed
that much of the vorticity generated by the acoustic wave at the
interface remains with the interface as it evolves and deforms even
long after the passage of all acoustic waves. Part of this is
attributed to a lack of physical mechanism for dissipating vorticity
in the inviscid case considered. From dimensional analysis we find
scaling law \eqref{eq:intf_circ_scaling}, suggesting that the
interface perturbation amplitude will grow as $t^{0.5}$ for purely
circulation driven growth. In our computed results we find the actual
perturbation amplitude grows as $t^{0.6}$. This slight discrepancy
could be a result of the inability of a global quantity $\Gamma$ to
completely describe $a(t)$ which is governed by local fluid mechanics.
%
(2) Baroclinic vorticity is predominantly deposited in the gaseous
fluid. We perform analysis to predict that on either side of an
infinitely sharp water-air interface, the vorticity generation rate
would be approximately two orders of magnitude greater on the air side
of the interface than in the water. This is qualitatively supported by
our computational results which find that near the end of the initial
compression wave-interface interaction nearly all of the circulation
exists in fluid dominated by air. For the $10$ MPa wave, for instance,
97\% of the circulation is found in fluid with volume fraction of
water $\alpha<0.5$ at $t=1$, after 91\% of the compression has passed.
%
(3) Changes in the acoustic waveform that have little effect on the
interface dynamics during their interaction can substantially effect
the interface over longer periods of time, via vorticity. By comparing
the effects of $10$ MPa trapezoidal waves with varying static pressure
durations between compression and expansion, we observe that the
evolution of the interface between these two wave components
drastically effects the ultimate growth rate of the interface. The
phase and amplitude of the interface perturbation at the time it
encounters the expansion wave determine the direction and magnitude
respectively of the vorticity deposited. Consequently, the amount of
vorticity remaining at the interface and in the surrounding fluid
after the passage of the wave changes greatly based on the
time-dependent features of the wave.

This work is a step toward understanding the effects of acoustically
generated vorticity on gas-liquid interfaces, but there are many
questions left to be answered. Physical effects not considered here
such as viscosity, which is critical to understanding the dissipation
of viscosity, may be highly important in certain
regimes. Additionally, to fully understand the application of these
findings to the motivating problem of diagnostic induced lung
hemorrhage, it will be important to consider waveforms and geometries
more relevant to the problem.


% \subsection{old}

% As such, we expect the late-time deformation to depend on the
% circulation in a predictable way. From dimensional analysis, we expect
% that the perturbation amplitude of a purely circulation driven
% interface will grow as $(\Gamma t)^{0.5}$ after all waves have
% passed. This is in contrast to our experimental results which find the
% amplitude to grow as $(\Gamma t)^{0.6}$. This discrepancy may be a
% result of the limitations of attempting to use circulation, a global
% quantity, to describe the perturbation amplitude, which depends on the
% local dynamics of the peaks and troughs of the interface.

% The actual deposition of circulation we expect to depend on the
% interface and acoustic waveform. We demonstrate analytically that the
% dominant form of vorticity is baroclinic, which is supported by the
% directly proportional increase in circulation deposited and acoustic
% pressure gradient for a linear increase in pressure impinging upon the
% interface. 




% Deformation of the interface that occurs during the interaction with
% the wave plays an important role in the total circulation deposited
% and hence in the deformation.

% For a linearly
% increasing pressure occurring over sufficiently short time that the
% interface does not deform appreciably, the amount of vorticity
% deposited scales linearly with the amplitude of pressure gradient.



% This work is unique in that we propose a previously unconsidered
% potential damage mechanism for \ac{DUS}-induced \ac{LH}. We
% hypothesize that baroclinic torque occurs at fragile air-tissue
% interfaces of the lung due to misalignment between the \ac{US}
% pressure gradient and material interface density gradient, causing
% stress, deformation, and ultimately rupture at the interface. This
% mechanism arises as a result of nonlinear, compressible fluid
% mechanics, and cannot be predicted through traditional linear
% acoustics. We suggest that nonlinear effects such as baroclinic
% vorticity are important to this problem because of the sharp density
% discontinuities between air and tissue within the lungs. To
% investigate our hypothesis we develop a numerical model of \ac{DUS}
% wave-alveolus interaction and simulate the physics underlying
% acoustically driven, perturbed liquid-gas interfaces.

% We aim to investigate three specific questions presented in Section
% \ref{sec:introduction}. To address these questions, we
% enumerate three conclusions of this work based on our results.  First,
% acoustic waves can generate vorticity at perturbed liquid-gas
% interfaces as a result of baroclinic torque. Second, this vorticity is
% capable of appreciably deforming the interface in the inviscid,
% inelastic case. We note that at this preliminary stage, additional
% simulations, run for longer time, will be useful in solidifying this
% conclusion.  Third, acoustic properties relevant to \ac{DUS} including
% acoustic amplitude, wave duration, and repetition frequency, are
% important to circulation deposition and subsequently any
% circulated-driven deformation or hemorrhage. For the case of a simple
% trapezoidal pressure wave, we demonstrate that the amount of
% circulation deposited along the interface scales linearly with the
% acoustic pressure amplitude. More subtly, because the interface is
% deforming throughout its interaction with the wave, affecting the
% alignment of pressure and density gradients, the acoustic wave
% duration can play an important role in determining how much
% circulation is ultimately deposited. Because the deformation can
% continue long after the wave has gone, similar arguments can be made
% for timing between subsequent waves. With regard to ultrasound, this
% could play an important role in choosing optimal \acs{PD} and
% \acs{PRF}.

%%% Local Variables:
%%% mode: latex
%%% TeX-master: "../main"
%%% End:









