\section{Summary and Conclusions}
\label{sec:conclusions}
This paper demonstrates that acoustic waves may trigger significant
deformation of perturbed liquid-gas interfaces over long periods of
time. The driving mechanism behind this deformation is baroclinic
vorticity, which occurs as a result of misalignment between the
pressure gradient of the acoustic wave and density gradient of the
perturbed interface. To demonstrate this we simulate trapezoidal
acoustic waves of MPa-order amplitude impinging from water onto a
perturbed air interface air.

The work presented here supports the following three conclusions: (1)
Acoustically-generated baroclinic vorticity is capable of
significantly deforming perturbed liquid-gas interfaces. We observed
that much of the vorticity generated by the acoustic wave at the
interface remains with the interface as it evolves and deforms even
long after the passage of all acoustic waves. Part of this is
attributed to a lack of physical mechanism for dissipating vorticity
in the inviscid case considered. From dimensional analysis we find
scaling law \eqref{eq:intf_circ_scaling}, suggesting that the
interface perturbation amplitude will grow as $t^{0.5}$ for purely
circulation driven growth. In our computed results we find the actual
perturbation amplitude grows as $t^{0.6}$. This discrepancy could be a
result of the inability of a global quantity $\Gamma$ to completely
describe $a(t)$ which is governed by local fluid mechanics.
%
(2) Baroclinic vorticity is predominantly deposited in the gaseous
fluid. We perform analysis to predict that on either side of an
infinitely sharp water-air interface, the vorticity generation rate
would be approximately two orders of magnitude greater on the air side
of the interface than in the water. This is qualitatively supported by
our computational results which find that near the end of the initial
compression wave-interface interaction nearly all of the circulation
exists in fluid dominated by air. For the $10$ MPa wave, for instance,
97\% of the circulation is found in fluid with volume fraction of
water $\alpha<0.5$ at $t=1$, after 91\% of the compression has passed.
%
(3) Changes in the acoustic waveform that have little effect on the
interface dynamics during their interaction can substantially effect
the interface over longer periods of time, via vorticity. By comparing
the effects of $10$ MPa trapezoidal waves with varying static pressure
durations between compression and expansion, we observe that the
evolution of the interface between these two wave components
drastically effects the ultimate growth rate of the interface. The
phase and amplitude of the interface perturbation at the time it
encounters the expansion wave determine the direction and magnitude
respectively of the vorticity deposited. Consequently, the amount of
vorticity remaining at the interface and in the surrounding fluid
after the passage of the wave changes greatly based on the
time-dependent features of the wave. A consequence of this is that it
should be possible to design waves that minimize the growth of the
interface perturbation.

This work is a step toward understanding the effects of acoustically
generated vorticity on gas-liquid interfaces, but there are many
questions left to be answered. Physical effects not considered here
such as viscosity, which is critical to understanding the dissipation
of viscosity, may be important in certain
regimes. Additionally, to fully understand the application of these
findings to the motivating problem of diagnostic induced lung
hemorrhage, it will be pertinent to consider waveforms and geometries
more relevant to the problem.

%%% Local Variables:
%%% mode: latex
%%% TeX-master: "../main"
%%% End:









