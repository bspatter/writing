\begin{center}
  \begin{minipage}{0.8\textwidth}
    \subsection*{Abstract}
    Diagnostic ultrasound has been shown to cause lung hemorrhage in a
    variety of mammals, though the underlying damage mechanisms are still
    unclear. Motivated by this problem, we use numerical simulations to
    investigate the interaction of an ultrasound wave with the tissue-air
    interface. A half positive cycle represented by a trapezoidal waveform
    in tissue (modelled as water) impinges upon the lung (modelled as
    air); to represent the rough surface of the lung, the interface
    includes a small single-mode perturbation. Our simulations show that
    the perturbation amplitude grows to sizes many times larger than the
    original value, \emph{well after the wave has passed}. We demonstrate
    that conventional (linear) acoustics cannot account for such
    deformations. Instead, the perturbation growth is driven by nonlinear
    effects, i.e., the baroclinic vorticity deposited along the interface,
    due to the misalignment of the pressure gradient (acoustic wave) and
    the density gradient (perturbed gas-liquid interface). Despite its low
    amplitude, the acoustic wave deposits sufficient vorticity because of
    the sharp density gradient. Scaling analysis predicts a square-root
    dependence of the perturbation amplitude on time; \hl{our simulations
    achieve a $t^{0.6}$ behavior}. Since the configuration is
    heavy-to-light, the perturbation undergoes phase inversion. If the
    interval between the pressure increase and decrease is sufficient,
    both deposit vorticity of the same sign, which enhances the
    perturbation growth. Conversely, if the interval is too short, the
    vorticity deposited by the pressure increase is canceled by the
    decrease. A further consequence is that one may be able to control the
    growth of such perturbed interfaces by modulated the ultrasound wave.
  \end{minipage}
\end{center}


\begin{comment}
      \ac{DUS} has been shown to cause \ac{LH} in a variety of mammals,
    though the underlying damage mechanisms are still
    unclear. Motivated by this problem aim to investigate the physics
    underlying interactions between acoustic waves and liquid-gas
    interfaces such as those in the lung. We model an alveolar
    tissue-air interface as a perturbed water-air interface and
    simulate its interaction with trapezoidal acoustic waves to
    investigate the underlying physics. Though order of magnitude
    analysis and simulation, we show that baroclinic vorticity is
    generated within gas-dominated fluid at the interface, as a result
    of misalignment between acoustic pressure gradients and the
    interface density gradients. This vorticity deforms the interface
    long after all of the acoustic waves have passed. This nonlinear
    effect is has potential importance at liquid-gas interfaces
    because of the sharp density discontinuity. Dimensional analysis
    is performed to show that the amplitude of a purely
    circulation-driven interface will grow as $t^{0.5}$. This result
    is compared to computational results obtained for the acoustically
    driven interface perturbation which appears to grow as
    $t^{0.6}$ at late time. \par
    
    \hspace*{4ex} Additionally we subject water-air interfaces to
    waves of similar size and amplitude, but varying duration to
    demonstrate that interface deformation that occurs during the
    wave-interface interaction appreciably changes the long-term
    dynamics of the interface by altering the vorticity
    deposition. This effect is relevant to acoustic waves in
    particular, which travel occupy a finite amount of space, and
    hence interact with the interface over variable, finite amounts of
    time. This is in contrast to the well studied shock-interface
    interactions, which occur nearly instantaneously.
\end{comment}

%%% Local Variables:
%%% mode: latex
%%% TeX-master: "../main"
%%% End:

