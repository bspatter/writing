\begin{center}
  \begin{minipage}{0.8\textwidth}
    \subsection*{Abstract}
    \ac{DUS} has been shown to cause \ac{LH} in a variety of mammals,
    though the underlying damage mechanisms are still
    unclear. Motivated by this problem aim to investigate the physics
    underlying interactions between acoustic waves and liquid-gas
    interfaces such as those in the lung. We model an alveolar
    tissue-air interface as a perturbed water-air interface and
    simulate its interaction with trapezoidal acoustic waves to
    investigate the underlying physics. Though order of magnitude
    analysis and simulation, we show that baroclinic vorticity is
    generated within gas-dominated fluid at the interface, as a result
    of misalignment between acoustic pressure gradients and the
    interface density gradients. This vorticity deforms the interface
    long after all of the acoustic waves have passed. This nonlinear
    effect is has potential importance at liquid-gas interfaces
    because of the sharp density discontinuity. Dimensional analysis
    is performed to show that the amplitude of a purely
    circulation-driven interface will grow as $t^{0.5}$. This result
    is compared to computational results obtained for the acoustically
    driven interface perturbation which appears to grow as
    $t^{0.6}$ at late time. \par
    
    \hspace*{4ex} Additionally we subject water-air interfaces to
    waves of similar size and amplitude, but varying duration to
    demonstrate that interface deformation that occurs during the
    wave-interface interaction appreciably changes the long-term
    dynamics of the interface by altering the vorticity
    deposition. This effect is relevant to acoustic waves in
    particular, which travel occupy a finite amount of space, and
    hence interact with the interface over variable, finite amounts of
    time. This is in contrast to the well studied shock-interface
    interactions, which occur nearly instantaneously.
  \end{minipage}
\end{center}
%%% Local Variables:
%%% mode: latex
%%% TeX-master: "../main"
%%% End:
