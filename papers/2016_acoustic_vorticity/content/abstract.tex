\begin{center}
  \begin{minipage}{0.8\textwidth}
    \subsection*{Abstract}
    \ac{DUS} of the lung has been shown to cause hemorrhage in a
    variety of mammals, though the underlying damage mechanism is yet
    to be determined. Motivated by this problem we model an alveolar
    tissue-air interface as a perturbed water-air interface and
    simulate its interaction with trapezoidal acoustic waves to
    investigate the underlying physics. We find that baroclinic
    vorticity is generated along the interface as a result of
    misalignment between acoustic pressure gradients and the density
    gradients across the interface. This vorticity remains continues
    to deform the interface long after all of the acoustic waves have
    passed. We postulate that this nonlinear effect is important
    because of the sharp density gradient that occurs at the water-air
    interface.

    Unlike shocks, whose interactions with fluid-fluid interfaces is
    well studied and nearly instantaneous, the acoustic waves
    considered here interact with the interface over variable, finite
    amounts of time. The effect of this interaction time is shown to
    have a significant impact on the growth of the interface
    perturbation for acoustic waves of similar amplitude and shape. We
    show that this is a result of changes in vorticity deposition that
    occur due to interface deformation that occurs during the
    interaction with the
    wave.

    Finally, We additionally perform analysis to predict the location
    of vorticity generation and growth of the interface, which we
    compare to the observed computational results.
  \end{minipage}
\end{center}
%%% Local Variables:
%%% mode: latex
%%% TeX-master: "../main"
%%% End:
