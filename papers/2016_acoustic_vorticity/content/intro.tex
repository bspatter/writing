\section{Introduction}%
\label{sec:introduction}%
%
\ac{DUS} is one of the safest forms of medical imaging and has become
ubiquitous in clinical practice. However it has been demonstrated to
cause \ac{LH} in a variety of mammals. While the problem does not
appear to be one of medical concern under typical clinical conditions,
the underlying physical mechanism is not well understood and more
information is needed so that future developments in ultrasound remain
safe. The purpose of this work is to investigate the underlying
physics of acoustically-driven liquid-gas interfaces such as those in
the lungs.

Over the last few decades \ac{DUS}-induced \ac{LH} has been studied
extensively and is the only known bioeffect of non-contrast, pulsed
\ac{US} known to occur in mammals under clinically relevant
conditions. While it has been not been directly demonstrated in
humans, it has been shown to occur in a variety of mammals including
mice, rats, rabbits, pigs, and monkeys
\citep{Child1990,OBrien2006a,Tarantal1994a}. Presently the physical
damage mechanisms underlying \ac{DUS}-induced \ac{LH} are not well
understood.

\hl{
\ac{DUS}-induced \ac{LH} does not appear to be caused by traditionally
expected \ac{US} bioeffects mechanisms. Typically, \ac{US} bioeffects
mechanisms are classified as thermal or non-thermal with the bulk of
non-thermal bioeffects being a result of acoustic
\ac{IC}. \cite{Zachary2006} finds that \ac{DUS}-induced lung lesions
do not appear similar to those induced by heat and concludes that
thermal mechanisms are not likely to be the cause. \cite{OBrien2000}
observes that the severity of \ac{DUS}-induced \ac{LH} in mice
increases under raised hydrostatic pressure, and \cite{Raeman1996}
finds that the \ac{LH} is unaffected by the introduction of \ac{US}
contrast agents. Both of these findings are inconsistent with what
would be expected of \ac{IC}-induced hemorrhage. One study reports
detecting cavitation during \ac{DUS}-lung interaction in rats
\cite{Holland1996}. \cite{Tjan2007} considers another potential damage
mechanism, that focused \ac{US} may lead to the ejection of droplets
capable of puncturing the air-filled sacs within the lung. To
investigate this, they perform numerical simulations of as an
inviscid, free surface subjected to a Gaussian velocity potential and
show show that the proposed droplet ejection may occur under certain
circumstances. Similarly, \cite{Simon2012} observed \ac{HIFU} induced
atomization of tissue at air interfaces. Despite these efforts, the
precise damage mechanism underlying \ac{DUS}-induced \ac{LH} is still
unknown.
}

Within the fluids community, there has been extensive research into
the fundamental physics describing interactions between mechanical
waves and fluid-fluid interfaces. Much of this research is motivated
by applications in fusion energy and astrophysics and accordingly has
sought to investigate regimes outside of those of acoustic
interests. One topic of considerable past study is the \ac{RMI}, in
which a perturbed fluid-fluid interface is instantaneously accelerated
by a shock, causing the interface perturbation to grow
\citep{Brouillette2002,Drake2006}. This growth is driven by a sheet of
baroclinic vorticity deposited along the interface as a result of
misalignment between the pressure gradient across the shock and the
density gradient across the perturbed interface. This physical
mechanism by which these misaligned gradients create a torque on fluid
particles and generate vorticity can be thought of in terms of a
hydrostatic balance upon a particle. Pressure gradients result in
acceleration of the flow that is inversely proportional to
density. When these two gradients are misaligned, the result is a
shearing effect on the fluid and vorticity is generated. A graphical
explanations of baroclinic vorticity generation and the resulting
interface deformation can be found in
\citep{Heifetz2015}. Analytically, baroclinic vorticity generation can
be shown by taking the curl of the conservation of momentum equation
for a compressible fluid. However we note that it is a nonlinear
effect and cannot be explained by traditional linear acoustics.

The physics of the \ac{RMI} are fairly well understood. For the
classical \ac{RMI} setup a planar shock impinges normally upon the
peaks and troughs of a sinusoidal interface. As the degree of
misalignment varies along the interface, the interface is accelerated
non-uniformly. The direction of the vorticity changes where the slope
of the interface changes. This counter rotation on either side of
interface peaks and troughs entrains nearby fluid causing interface
peaks to accelerate in one direction and troughs to accelerate in the
opposite direction. This results in a ``bubble'' of light fluid
penetrating the heavy fluid, and a ``spike'' of heavy fluid
penetrating the light fluid. How exactly this occurs varies slightly
depending on the relative densities of the two fluids. For the case of
a wave moving from a light fluid into a heavy one, the peaks and
troughs of the interface are initially accelerated to move away from
one another, and the interface perturbation amplitude undergoes growth
exclusively. For the case of a wave moving from a heavy fluid to a
lighter fluid, the peaks and troughs of the interface are accelerated
such that they initially move closer to one another decreasing the
perturbation amplitude. They then pass one another, inverting the
phase of the interface perturbation, and then continue moving in
opposite directions, growing the perturbation amplitude. This process
is illustrated in \citep{Brouillette2002}.
\begin{comment}
  \begin{figure}
    \centering \def\svgwidth{0.9\textwidth}
    \import{./figs/lung_figs/}{brouillette_fig3_mod.pdf_tex} \hfill%
    \caption[A schematic view of the \ac{RMI} instability for a
    heavy-light interface]{Adapted from \cite{Brouillette2002}. The
      \ac{RMI} for a heavy-light interface is illustrated. The initial
      condition (left), circulation post wave-interface interaction
      (center), and perturbation growth (right) are shown.}
    \label{fig:rmi_schematic}
  \end{figure}
\end{comment}
Previous studies of the \ac{RMI} have utilized theory, computation,
and experiments to describe the behavior of the interface after the
wave has passed. \cite{Richtmyer1960} performed the linear stability
perturbation analysis developed by \cite{Taylor1950} for the case of
an impulsive acceleration to create a model for the initial growth of
the interface perturbation. \cite{Meshkov1969} experimentally
confirmed Richtmyer's qualitative predictions, hence the name of the
instability. \cite{Meyer1972} performed numerical simulations of the
\ac{RMI} and found good agreement with Richtmyer for the case of a
shock impinging upon a light-heavy interface. \cite{Fraley1986} used
Laplace transforms in order to find the first analytical solution for
the asymptotic growth rate for a shocked interface between perfect
gases. To describe the late time, nonlinear growth of the
perturbation, \cite{Zhang1997} used single mode perturbation, keeping
many high order terms, to describe the velocity of the bubble and
spike regions of the fluid. \cite{Sadot1998} combined the linear,
impulsive solution with potential flow models of the asymptotic
behavior of the bubble and spike to develop a model for the
perturbation growth that is in good agreement with shock tube
experiments for shocks with Mach numbers Ma=1.3, 3.5. Vortex theory
has also been used to describe the behavior of the
interface. \cite{Jacobs1996} horizontally oscillated a container with
two vertically stratified liquids to obtain standing waves and then
bounced the container off of a coil spring to study the incompressible
\ac{RMI}. The late time evolution of is interface is modeled using a
row of line vorticies to obtain qualitatively similar results to those
experimentally observed, however the late-time growth rate is
underestimated. \cite{Samtaney1994} used shock polar analysis to find
the circulation deposited by a shock on planar and non-planar
interfaces. Their results are validated using and Euler code and found
to be within 10\% of the computed value for $1.0\,<\,$Ma$\,\leq\,1.32$
for all $\rho_2/\rho_1\,>\,1$, and
$5.8\,\leq\,\rho_2/\rho_1\,\leq\,32.6$ for all Ma.

Beyond the \ac{RMI}, computational solutions of the nonlinear
equations of fluid motion have been used to study wave-interface
problems such as shock-bubble and shock-droplet
interactions. \cite{Ball2000} solved the Euler equations to simulate
interaction between a shock-wave and a cylindrical cavity and found
that baroclinicity effects the dynamics of the cavity.
%% ADD ANOTHER SHOCK-DROP AND A SHOCK-DROPLET

This work is separate from previous research into the \ac{RMI} as a
result of the acoustic regime considered here. Unlike shock waves,
which occur over a few molecular mean free paths and interact nearly
instantaneously, acoustic waves have a finite spacial wavelength and
can occupy a much larger portion of space. Consequently, their
interaction with interfaces occurs over a longer period of time
depending on the waveform and the geometric and material properties of
the media. This duration can also be thought of in terms of the
relative sizes of the physical features of the interface and the
wavelengths of each feature of the acoustic wave of interest. Simple
\ac{RMI} analysis assumes an impulsive acceleration, and does not
apply to this work because the interface deforms throughout its
interaction with the wave. Transient interactions between deforming
interfaces and shocks are known to be capable of affecting the long
term growth of the interface. \cite{HenrydeFrahan2015b} demonstrated
that interface growth could be controlled through transient
wave-interface interactions by simulating shock passage through
layered media, resulting in subsequent interactions between the
interfaces and reflected and transmitted shocks and rarefactions.
This work aims to add to the current body of work on this topic by
investigating the importance of transient acoustic wave behavior on
interface growth.

Additionally, we argue that the basic problem setup of the \ac{RMI}, a
mechanical wave impinging upon a material interface, has many
physical similarities to the motivating problem of \ac{DUS} of the
lungs. While the pressure gradient due to the \ac{US} pulse is not as
sharp as it would be across the shock, there is a strong density
discontinuity across the tissue-air interfaces of the
lungs. Accordingly, we study problem geometries and acoustic waves
within the regimes of \ac{DUS} of the lung.

\begin{figure}[h]
  \centering
  \def\svgwidth{0.48\textwidth}
  \import{./figs/lung_figs/}{usbe_lung_schematic2.pdf_tex} \hfill%
  \caption[A schematic view of the model problem]{A schematic of an
    ultrasound wave impinging from tissue onto an air-filled alveolus.}
  \label{fig:problem_schematic}
\end{figure}
In the remainder of this work we will first present a model problem
and a set of numerical experiments designed to investigate the
fundamental physics underlying interactions between acoustic waves and
perturbed interfaces between fluids. We will present results that
specifically explore the dependence of the interface dynamics on
acoustically relevant properties including the amplitude and
wavelength of the acoustic wave. We pay particular attention to the
finite duration of the wave-interface interaction, and the interface
deformation that occurs during this period. We perform analysis to
explain the vorticity deposition and late time behavior of the
interface after all acoustic waves have passed. The results will be
discussed to address the following questions:
\begin{enumerate} \label{itm:usbe_lung_questions}
\item Are acoustic waves capable of generating sufficient baroclinic
  vorticity at perturbed liquid-gas interfaces for substantial
  deformations?
\item What is the impact of the acoustic wave properties, such as
  amplitude and wave duration, on the vorticity and interface
  dynamics?
\end{enumerate}
We will then discuss on the significance of these results as
they regard to the motivating problem of \ac{DUS}-induced lung
hemorrhage. We will finally end by summarizing the main conclusions
drawn from this work and suggest the next steps to be taken.



% \hl{ (MOVE ALL THIS TO LATER)
% We consider a compressible multi-fluid system and
% solve the Euler equations of inviscid fluid motion to study the
% dynamics of fluid-fluid interfaces exposed to acoustic waves relevant
% to \ac{DUS}. We observe that acoustically-generated vorticity at
% perturbed water-air interfaces drives the interface to deform. We
% hypothesize that a similar mechanism may be responsible for deforming
% and ultimately rupturing the fragile tissue barriers around alveoli,
% tiny air sacs within the lungs, leading to \ac{DUS} induced \ac{LH}.}




%%% Local Variables:
%%% mode: latex
%%% TeX-master: "../main"
%%% End:
