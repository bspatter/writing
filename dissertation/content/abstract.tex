\acresetall%
\setlength\parindent{20pt}%
Numerical simulations are useful for approaching a wide range of
problems within the field of acoustics. From the propagation of sound
across vast, dynamic oceans to the ultrasonically driven collapse of
microbubbles in the blood stream, there are a variety of contemporary
acoustic problems with challenges that make them difficult or
infeasible to investigate experimentally. Computation is useful in
overcoming some of the difficulties of experimentation to gain
insights that might otherwise remain elusive. In this dissertation, we
present novel work in two very different and distinct areas of
acoustics: the modeling and study of diagnostic ultrasound bioeffects
and the estimation of acoustic transmission loss uncertainty in
uncertain ocean environments. While the problems we approach in each
of these areas are similar in that the are concerned with nonlinear
acoustics in complex, uncertain media, they are otherwise quite
separate. As such, this document is split into two separate, but
unequal parts and the summary of relevant background information, work
performed, and conclusions in each area will be presented separately.

The first part of this dissertation, which represents the bulk of the
work herein, investigates the physics associated with two biological
effects of medical \ac{US}: hemorrhage resulting from \ac{CEUS} and
lung hemorrhage resulting non-contrast \ac{DUS}. First, we consider
\ac{CEUS}-induced hemorrhage in mammals, which is well documented and
thought to be a result of inertial cavitation, though the specific
physical damage mechanisms and thresholds are not well understood. To
advance our knowledge in this area we aimed to use computational
modeling and simulation, combined with experimental results to relate
standard cavitation metrics to known bioeffects thresholds. In this
work we developed a computational model of spherical bubble collapse
in a compressible, viscoelastic soft tissue. In a novel contribution
to this area we drove the bubbles using experimentally measured
ultrasound waveforms ($1.5$ - $7.5$ MHz), with known peak rarefaction
pressure amplitude thresholds for glomerular capillary hemorrhage in
rats. Using our computational model we studied the dependence of
common cavitation metrics on physical parameters including viscosity,
elasticity, equilibrium bubble radius, and gas contents. Cavitation
metrics such as the maximum bubble radius relative to equilibrium were
compared with known bioeffects thresholds and accepted thresholds for
the onset of inertial cavitation in water (e.g., $R_{max}/R_0=2$,
$T_{max}=5000$ K). We find that the experimentally determined
bioeffects thresholds were greater than the previously determined
inertial cavitation thresholds and when applied to our results were
poor predictors of biological effects. Additionally, we found that the
frequency dependence of the calculated dimensionless maximum bubble
radius correlated strongly with that of the bioeffects thresholds.

For the second problem we aimed to use full numerical simulations to
understand some of the detailed physics associated with interactions
between alveolar tissue-air interfaces and ultrasound waves. To
accomplish this we modeled the problem as an acoustic wave in water
impinging upon a sinusoidally-perturbed water-air interface. We
performed a detailed study of the dynamics of the interface driven by
a trapezoidal acoustic wave with variable amplitude and wavelength,
determined from common \ac{DUS} parameters (Peak amplitudes from $5$
v-- $12.5$ MPa). We observed that the interface perturbation grew
vboth during its interaction with the wave and continued to grow long
after the wave had passed. We showed that this late time growth is a
result of baroclinic vorticity generated along the interface by
misalignment between the ultrasound pressure gradient and the sharp
interface density gradient. This is a second order effect that cannot
be explained by linear acoustics. Using dimensional analysis, we show
that the interface length and perturbation amplitude grow according to
power laws in time and scale with the circulation
density. Furthermore, because the vorticity generation depends
strongly on the dynamic interface morphology during its interaction
with the wave, we concluded that the long term growth depends heavily
on the duration and form of the acoustic wave. This may have important
implications for ultrasound-induced alveolar wall strain.

Having developed an understanding of the physics underlying the
acoustically driven liquid-gas interface, we then aimed to use our
model to estimate alveolar stresses and strains resulting from
interactions with a single ultrasound pulse. We expanded our study to
consider liquid-gas interfaces driven by $1.5$ MHz ultrasound pulses
with peak amplitudes $1$, $2.5$, and $5$ MPa. For these cases we
calculated passive viscous stresses and linear strains for the
interface and compare with previously observed alveolar failure
criteria from the literature. We found that the estimated viscous
stress maximums are on the order of tens of Pascals and are much
smaller than those expected to cause alveolar failure. However, at the
end of simulation, $t=288~\mu$s, interfacial strains of up to 38\% were
observed for the $5$ MPa wave. As this work considers only a single pulse,
and the vorticity mechanisms driving the deformation are expected to accumulate
from pulse to pulse, we conclude that these strains are worthy of
further investigation.

In the second part of this dissertation we develop \textit{area
  statistics}, a computationally efficient method for estimating the
\ac{PDF} of acoustic \ac{TL} in uncertain ocean
environments. Knowledge of transmission loss uncertainty is useful in
a variety of practical naval applications, but standard methods for
determining \ac{TL} uncertainty are computationally expensive and
poorly suited for real-time applications. This work describes how
\ac{TL} statistics in a range-depth area surrounding a point of
interest can be used to obtain an approximate \ac{PDF} of \ac{TL}. The
technique is based on the idea that \ac{TL} variations at a location
of interest that occur because of random environmental fluctuations
are represented by spatial variations in \ac{TL} in single baseline
\ac{TL} field calculation. To perform this technique we used a single
\ac{TL}-field calculation based on the most probable values for each
of the uncertain environmental parameters, which describe the sound
speed profile, bathymetry, and geoacoustic bottom layer sound speed,
depth, and attenuation. The $L_1$ error norm is used to compare
\textit{area statistics}-generated \acp{PDF} to those generated from
2000-sample Monte Carlo calculations. The technique is tested for
source frequencies of 100, 200, and 300 Hz and source depths of 91,
137, 183m (300, 450, 600 ft), in ten uncertain ocean environments with
a wide range of geometric and acoustic parameters and uncertainties. Over 11,000 test
locations were considered at depths from 20 m to 4.5 km and
ranges from less than 1 km to greater than 70 km. The area
statistics-generated \acp{PDF} had an $L_1$ error $<$ 0.5 in 91\% of
tested locations and were thus considered engineering level
accurate. On average the \ac{PDF}s were generated in milliseconds of
real time on a typical desktop computer and required less than one
millionth the computational effort required to generate the Monte
Carlo \ac{PDF}s.  \setlength\parindent{0pt}%


% % The use of \ac{US} for diagnostic and therapeutic purposes has grown
% % very quickly over the last few decades, as technological advancements
% % have allowed us to use \ac{US} for everything from drug delivery and
% % destruction of unwanted tissue to imaging of the internals of the
% % human body. However, certain \ac{US} procedures have been shown to
% % cause unwanted biological effects that are still poorly understood
% % because they occur on such small length and time scales that they
% % cannot be directly observed as they occur in the body. We develop
% % numerical models to investigate experimentally observed hemorrhage
% % associated with \ac{CEUS} and \ac{DUS} of the lung. In doing so, we
% % compare calculated cavitation bubble dynamics pertinent to \ac{CEUS}
% % to previously obtained experimental results to show that accepted
% % thresholds for \ac{IC} are unlikely to be useful for predicting
% % \ac{CEUS} bioeffects. For the case of \ac{DUS} of the lung, we propose
% % a new physical damage mechanism based on ultrasonically-induced
% % baroclinic torque at tissue air interfaces within the lungs. We
% % perform analysis and simulations to demonstrate that, given certain
% % simplifying assumptions, predictable deformation of the interface
% % occurs via the proposed mechanism. We conclude that it is a feasible
% % injury mechanism, which will be investigated in detail over the
% % remainder of this thesis.

%\hl{FINISH ME}
% In the second part of this work we develop area statistics, a
% computationally efficient method for estimating the \ac{PDF} of
% acoustic \ac{TL} in uncertain ocean environments. This is useful in a
% variety of practical naval applications. The area statistics method is
% tested in four different ocean environments of varying geometry and
% acoustic properties with uncertain bathymetry, ocean floor properties,
% and sound speed profiles. We compare \ac{TL} \acp{PDF} obtained with
% area statistics to those obtained via accepted, traditional \ac{MC}
% methods and find that the area statistics method is engineering-level
% accurate in 93\% of tests in ocean environments with consistent bottom
% reflection, and can be produced with $\orderof{10^{−6}}$ the computational
% effort required for the Monte-Carlo calculations.

% And many
% systems of modern acoustic interest such as the ocean and human body
% are immensely complex, rapidly changing, and poorly understood such
% that at any given time our best knowledge of the system is still
% fraught with uncertainty.


%%% Local Variables:
%%% mode: latex
%%% TeX-master: "../main"
%%% End:
