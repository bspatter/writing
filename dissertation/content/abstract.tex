\acresetall
Numerical simulations are highly useful for approaching a wide range
of problems within the field of acoustics. From submarine and whale
sounds propagating across vast oceans to ultrasound striking
microscopic bubbles in the blood stream behind layers of optically
opaque tissue, there are a variety of contemporary acoustic problems
with challenges that make them incredibly difficult or infeasible to
investigate experimentally. Furthermore, many systems of acoustic
interest such as the ocean and human body are immensely complex,
rapidly changing, and poorly understood such that at any given time
our best knowledge of the system is still fraught with
uncertainty. Computation is used to overcome the difficulties of
experimentation, explore a range of possibilities in uncertain
systems, and to obtain detailed information about physical quantities
that are not readily measurable. In this work, we present advancements
in two very different areas of acoustics, made possible through the
use of computation.

In this work, we investigate two problems related to
biological effects of medical \ac{US}. The use of \ac{US} for
diagnostic and therapeutic purposes has grown very quickly over the
last few decades, as technological advancements have allowed us to use
\ac{US} for everything from drug delivery and destruction of unwanted
tissue to imaging of the internals of the human body. However, certain
\ac{US} procedures have been shown to cause unwanted biological
effects that are still poorly understood because they occur on such
small length and time scales that they cannot be directly observed as
they occur in the body. We develop numerical models to investigate
experimentally observed hemorrhage associated with \ac{CEUS} and
\ac{DUS} of the lung. In doing so, we compare calculated cavitation
bubble dynamics pertinent to \ac{CEUS} to previously obtained
experimental results to show that accepted thresholds for \ac{IC} are
unlikely to be useful for predicting \ac{CEUS} bioeffects. For the
case of \ac{DUS} of the lung, we propose a new physical damage
mechanism based on ultrasonically-induced baroclinic torque at tissue
air interfaces within the lungs. We perform analysis and simulations
to demonstrate that, given certain simplifying assumptions, predictable
deformation of the interface occurs via the proposed mechanism. We
conclude that it is a feasible injury mechanism, which will be
investigated in detail over the remainder of this thesis.

\hl{FINISH ME}

% In the second part of this work we develop area statistics, a
% computationally efficient method for estimating the \ac{PDF} of
% acoustic \ac{TL} in uncertain ocean environments. This is useful in a
% variety of practical naval applications. The area statistics method is
% tested in four different ocean environments of varying geometry and
% acoustic properties with uncertain bathymetry, ocean floor properties,
% and sound speed profiles. We compare \ac{TL} \acp{PDF} obtained with
% area statistics to those obtained via accepted, traditional \ac{MC}
% methods and find that the area statistics method is engineering-level
% accurate in 93\% of tests in ocean environments with consistent bottom
% reflection, and can be produced with $\orderof{10^{−6}}$ the computational
% effort required for the Monte-Carlo calculations.

%%% Local Variables:
%%% mode: latex
%%% TeX-master: "../prelim"
%%% End:
