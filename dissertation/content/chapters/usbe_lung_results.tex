\section{Preliminary results and discussion}%
\label{sec:usbe_lung_results}%
In this section we present the results of the numerical experiments
and compare them to our analysis. We focus specifically on the
vorticity/circulation and interface dynamics. We first investigate the
response of the the trapezoidal wave case in detail and then
qualitatively compare this to results for the \ac{US} pulse case.
%
%
\subsection{Interface response to the $p_a=10$ MPa trapezoidal wave}
\label{subsec:Interface response to }
\subsubsection{Qualitative behavior of the interface and vorticity}
\label{subsubsec:Qualitative}
To provide a qualitative understanding of the underlying physics, we
consider our reference case in which a $p_a=10$ MPa trapezoidal wave
(See Figure \ref{fig:p0}) impinges on the water-air interface. Nearly
all of the acoustic energy is reflected back into the
water as a tension wave due lower acoustic impedance of the second
fluid. The transmitted compression wave is weakly focused due to the sound
speed mismatch across the curved interface perturbation. These
reflected and transmitted waves dissipate at the inflow and outflow
boundaries.

To illustrate the evolution of the interface and vorticity fields,
Figure \ref{fig:interface_snapshots} contains color plots of the
density (Top) and vorticity (Bottom) fields at different instances in
the flow's evolution. Areas of high density (i.e., water) are dark
blue and areas of low density (i.e., air) are light-blue. On the
vorticity contours, counterclockwise (positive) vorticity is red, and
clockwise (negative) vorticity is blue. The purpose of the vorticity
plots is only to show the location and direction of vorticity at each
time. For sake of visualization, the range of the vorticity color
scale changes at each time slice because the vorticity spreads over
time. Hence the vorticity magnitudes are not shown here. Contours of
$y_0=0.5$ volume fraction are indicated in black on both plots. 

The initially smooth interface perturbation grows from a smooth
sinusoid to a sharp spike at late time. At $t=1$, the
compression-interface interaction has nearly completed and the
vorticity is heavily concentrated in the air such that 97\% of the
total circulation in the left or right half domain exists in fluid
with volume fraction of water $y_0<0.5$. This is qualitatively
consistent with our analysis . As time progresses, it can be seen that
the vorticity disperses throughout the domain, but remains
concentrated around the interface and the vertical center of the
domain.

To more closely exam the interface and circulation dynamics associated
with the compression wave-interface interaction, Figure
\ref{fig:trapz10_circ_interface} shows the early-time histories of the
interface amplitude $a(t)$ and half-domain circulation
$\Gamma$. $t_{1-4}$ are the times at which the interface first
encounters each features of the incoming wave: 1-pressure rise,
2-static elevated pressure, 3-pressure fall, and 4-return to ambient
pressure. These points are denoted with black $\bs{\times}$s along the
curves in these figures and those hereafter. From $t_1=0^+$ to $t_2$
the compression wave encounters the interface. During this interaction
the perturbation amplitude decreases, and the right half-domain
circulation $\Gamma$ rises sharply. At $t_2\approx1.1$, the pressure
reaches its maximum amplitude, $p_a=10$ MPa, and remains constant
until $t_3$. We note that at $\overline{a(t_{1-2})}/a_0\approx0.96$,
suggesting that the static interface assumption made in our vorticity
generation order of magnitude analysis was reasonable. The interface
amplitude continues to decrease and the half-domain circulation
$\Gamma$ stops its rapid growth and changes little during this static
elevated pressure period, until the expansion wave hits at $t_3$. At
$t\approx 5.0$, the perturbation undergoes a phase inversion and
begins to grow, as is observed for the heavy-light interface
Richtmyer-Meshkov problem. At $t_3\approx8.5$ the expansion wave first
hits the interface. The perturbation amplitude continues to grow, and
$\Gamma$ increases sharply again. At $t_4\approx9.7$ the acoustic wave
has finished traversing the interface, and atmospheric pressure is
resumed. The perturbation amplitude $a_0$ continues to grow long after
the wave-interface interaction has finished.
%
\begin{figure}[h] 
  \centering
\includegraphics[width=0.9\textwidth]{./figs/lung_figs/snapshots_t1}
\caption[The evolution of the acoustically perturbed interface and vorticity field]{Surface plots of density (Top) and vorticity (Bottom)
  throughout the evolution of the interface for the $10$ MPa
  trapezoidal wave case. Areas of high density (i.e., water) are
indicated in dark blue. Areas of low density (i.e., air) are indicated
in white.  Positive (counterclockwise) vorticity is indicated in red,
and negative (clockwise) vorticity can be seen in blue.}
  \label{fig:interface_snapshots}
\end{figure}
%
\begin{figure}[h] 
  \centering
  \includegraphics[width=0.48\textwidth]{./figs/lung_figs/trapz10_intf_schematic}
  \includegraphics[width=0.48\textwidth]{./figs/lung_figs/trapz10_circ_schematic}
  \caption[The interface amplitude and circulation histories for the $10$ MPa trapezoidal wave]{The interface amplitude (left) and circulation (right)
    histories corresponding to the $10$ MPa trapezoidal waves are
    shown for $t\leq25$. Indicated times, $t_{1-4}$, are the times at
    which different stages of the incoming trapezoidal pressure wave
    shown in Figure \ref{fig:p0} first encounter the interface.}
  \label{fig:trapz10_circ_interface}
\end{figure}

\subsubsection{Dependence on acoustic wave amplitude}%
\label{subsubsec:usbe_lung_amplitude_dependence}%
To investigate the dependence of the dynamics on the trapezoidal wave
amplitude, we compare results for $p_a=1$, $5$, and $10$ MPa while
keeping the initial lengths of the wave $L$ and the rise and fall
$\Delta L_a$ constant such that $p_a$ scales linearly with the
acoustic pressure gradient. Figure
\ref{fig:trapz_circ_interface_early}, illustrates the interface
amplitude and $p_a$-normalized circulation histories for $t\leq25$,
during and shortly after the wave-interface interaction. Black
$\bs{\times}$s along the curves indicate $t_{1-4}$, described
previously in Subsection \label{subsec:Qualitative}. During the
interaction between the interface and the compression wave, the rate
at which the perturbation amplitude decreases is greater for higher
amplitude waves. The circulation deposited during this period scales
linearly with $p_a$ as is consistent with baroclinically-generated
circulation based on our analysis. For the $10$ MPa wave, the phase of
the interface inverts at, before the expansion hits, causing
circulation deposited by the expansion to have the same sign as that
deposited by the compression. For the $1$ and $5$ MPa waves interface
phase inversion occurs after the expansion and consequently deposits
circulation opposite that of the compression wave.

Figure \ref{fig:trapz_circ_interface_loglog} shows the interface
amplitude and circulation histories for $5$ and $10$ MPa trapezoidal
wave cases for $0 \leq t\leq 1000$. The perturbation amplitude history
is plotted on logarithmically-scaled axes. For both waves, the slope
of the perturbation amplitude is approximately $0.60$ long after the
waves have left the interface. This is slightly higher than the 0.5
slope predicted by scaling law \eqref{eq:intf_circ_scaling}. The
results for the $1$ MPa trapezoidal wave were not included because
interface evolved too slowly to obtain useful data given the
computational resources available.
%
\begin{figure}[h] 
  \centering
  \includegraphics[width=0.48\textwidth]{./figs/lung_figs/interface_multi-amp_norm}
  \includegraphics[width=0.48\textwidth]{./figs/lung_figs/circulation_multi-amp_norm}
  \caption[The interface and circulation dependence on wave amplitude at early time]{The interface amplitude (left) and circulation (right)
    histories corresponding to the $1$(yellow), $5$(orange), and
    $10$(blue) MPa trapezoidal waves are shown for $t\leq 25$. The
    circulation history is normalized by the acoustic amplitude of the
    incoming wave to illustrate that circulation deposition by the
    compression wave scales linearly with $p_a$.}
  \label{fig:trapz_circ_interface_early}
\end{figure}
%
\begin{figure}[h] 
  \centering
  \includegraphics[width=0.48\textwidth]{./figs/lung_figs/interface_multi-amp_loglog_roe_extra}
  \includegraphics[width=0.48\textwidth]{./figs/lung_figs/circulation_multi-amp2_roe_fixed}
  \caption[The interface and circulation dependence on wave amplitude
  at long time]{The interface amplitude (left) and circulation (right)
    histories corresponding to the $5$(orange) and $10$(blue) MPa
    trapezoidal waves are shown for $t\leq 500$. To appropriately
    compare late time dynamics, time has been offset in the interface
    amplitude history such that the phase reversal appears to occur
    simultaneously in both simulations. Dashed lines of the same color
    are used to demonstrate the expected slope of pure circulation
    driven interface growth, based on Equation
    \eqref{eq:intf_circ_scaling}. The red dashed line shows the slope we
    appear to be approaching for the $10$ MPa wave case for the end time.}
  \label{fig:trapz_circ_interface_loglog}
\end{figure}
%
\subsubsection{Circulation and vorticity dynamics}
We observe that the wave deposits a sheet of vorticity along the
interface that moves with the interface in time. Figure
\ref{fig:interface_snapshots} shows a surface plot of vorticity in the
region of the domain around the interface for the $10$ Mpa trapezoidal
wave case, at $t=1.0$, during the middle of the interface-compression
wave interaction (Left). Not shown is the rest of the domain, where
vorticity was relatively insignificant. The vorticity is antisymmetric
across the $x=0.5$ center line. To analyze the physical mechanisms
generating the vorticity, we plot each term of the circulation
generation equation \eqref{eq:circulation_generation} during the
period around the compression wave-interface interaction. Near the end
of the interaction at $t=1.0$,
$\left(\partial \Gamma/\partial t\right)_{advective} =
-5.3\,\text{e}{-5}$; %
$\left(\partial \Gamma/\partial t\right)_{compressible} =
2.7\,\text{e}{-5}$; %
$\left(\partial \Gamma/\partial t\right)_{baroclinic} =
7.7\,\text{e}{-3}$; %
$\left(\partial \Gamma/\partial t\right)_{total} =
7.7\,\text{e}{-3}$. %
This result is quantitatively consistent with expected vorticity generation
based on our analysis \eqref{eq:vorticity_comparison}. Furthermore, it
supports our hypothesis that vorticity is primarily baroclinically
generated. 
%
\begin{figure}[h] 
  \centering
%  \includegraphics[width=0.35\textwidth]{./figs/lung_figs/vorticity2}
  \includegraphics[width=0.48\textwidth]{./figs/lung_figs/ddtcirc_fixed}
  \caption[The individual contributions to circulation generation by physical mechanism]{Each term of the
    circulation generation equation \eqref{eq:circulation_generation} is plotted as a function of time:
    $\left(d\Gamma/dt\right)_{advective}$ (blue),
    $\left(d\Gamma/dt\right)_{compressible}$ (orange),
    $\left(d\Gamma/dt\right)_{baroclinic}$ (yellow),
    $\left(d\Gamma/dt\right)_{total}$ (purple, dashed).}
  \label{fig:trapz_ddt_circ}
\end{figure}
%
\subsubsection{Dependence on the length of the wave}%
To investigate the dependence of the dynamics on the length of the
trapezoidal wave $L$, and comparably the wave-interface interaction
time, we compare results for $p_a=10$ MPa waves of constant rise and
fall length $\Delta L_a$. This effectively changes the time the
interface has to evolve while experiencing the constant elevated
pressure portion of the wave between the compression and expansion.
Figure \ref{fig:trapz_circ_interface_multi-lag} shows the interface
amplitude and circulation histories corresponding to waves with
$L=45\lambda, 35\lambda ,30\lambda ,25\lambda ,15\lambda ,10\lambda$
for $0 \leq t\leq 25$.  For the three longest waves, $L \geq 30\lambda$,
the expansion encounters the interface after the perturbation reverses
phase. In these cases, the expansion deposits additional positive
circulation along the right half of the interface. For the shorter
waves, $L \leq 25\lambda$, the expansion encounters the interface before
the perturbation reverses phase and the net half-domain circulation is
decreased. Comparing cases in which the interface inverts phase before
the expansion occurs the larger $a(t)$ is at the time, the more
circulation is generated. The same is true when comparing cases in
which the phase inversion occurs after the interface inverts phase.
%
\begin{figure}[h] 
  \centering
  \includegraphics[width=0.48\textwidth]{./figs/lung_figs/interface_multi-lag}
  \includegraphics[width=0.48\textwidth]{./figs/lung_figs/circulation_multi-lag_fixed}
  \caption[The interface and circulation dependence on wave
  duration]{The interface amplitude (left) and circulation (right)
    histories for waves of varying total length $L$ and elevated
    static pressure duration between the expansion and compression
    . Here we show results for $L=45\lambda$ (blue), $L=35\lambda$
    (orange), $L=30\lambda$ (yellow), $L=20\lambda$ (purple),
    $L=15\lambda$ (green), $L=10\lambda$ (light blue)}
  \label{fig:trapz_circ_interface_multi-lag}
\end{figure}

\subsection{Interface response to \acf{DUS} waves}%
\label{subsec:usbe_lung_trapezoidal_results}%
To evaluate the relevance of our trapezoidal wave experiments we simulate
a $p_a=1, 5$ and $10$ MPa \ac{DUS} pulse waves (See Figure
\ref{fig:p0}) impinging onto the water air interface. In figure
\ref{fig:us_circ_interface} we illustrate the circulation and
interface amplitude histories for the $p_a=10$ MPa \ac{DUS} like-pulse
case. The post-wave interface dynamics are similar to those observed
for trapezoidal wave cases. During the wave-interface interaction, the
interface amplitude is compressed overall, but oscillations are
observed in correspondence with the acoustic pulse oscillations. After
the wave has left the interface, the perturbation amplitude continues
to decrease until the interface undergoes a phase inversion, after
which the perturbation amplitude grows for the remainder of the
simulation. half-domain circulation oscillates during wave-interface
interaction before settling to a nearly constant non-zero value after
the wave has passed. We note that the total circulation deposited is
of the same order of magnitude as that generated by the trapezoidal
wave of the same amplitude and duration. Qualitatively similar results
were observed for the $5$ MPa case. For the one $1$ MPa case, the
evolution of the system was slow such that running the simulation long
enough to obtain useful results was computationally prohibitive.

\begin{figure}
  \centering
  \includegraphics[width=0.48\textwidth]{./figs/lung_figs/us_intf_schematic} \hfill
  \includegraphics[width=0.48\textwidth]{./figs/lung_figs/us_circ_schematic}
  \caption[The interface amplitude and circulation histories for the \ac{DUS} pulse]{The interface amplitude (left) and circulation (right)
    histories corresponding to the a water-air interface disturbed by
    the US-like pulse shown in Figure \ref{fig:p0}.}
  \label{fig:us_circ_interface}
\end{figure}

\subsection{Further discussion of the results}%
\label{subsec:usbe_lung_further_discussion}%
For both the trapezoidal and \ac{DUS} pulse acoustic waves, the
pressure, velocity, and density return to initial, ambient conditions
after the passing of the wave. As these waveforms are continuous, this
implies that the integral of the pressure gradient $\nabla p$ at each
point along the interface, over all time must be zero. Hence we
surmise that if the interface remains unchanged during the interaction
with the wave, as it would for a wave moving with infinite velocity,
$\nabla \rho$ remains constant and the net baroclinic circulation
deposited must be zero. Thus for any finite duration acoustic wave
such as ours to deposit net baroclinic circulation upon an interface,
the interface itself must deform during interaction with the
wave. This deformation alters the misalignment of the pressure and
density gradients at the interface causing positive and negative
circulation deposited to not cancel out entirely. Note that this is
unique to waves that begin and end at the same pressure. This is not
the case for the traditional \ac{RMI} problem, for which conditions do
not return to their original state after the passage of the shock.

For the cases varying the length of the static elevated pressure in
the trapezoidal wave we previously noted that whether the expansion
increased or decreased the total half-domain circulation depended on
whether it encountered the interface before or after the phase
change. If indeed circulation is driving the deformation of the
interface, then changes in the waveform that appear to have 
little effect on the interface dynamics during the wave-interface
interaction period, may have far more significant impacts on the long
term dynamics of the interface. To put this in the context of
\ac{DUS}, which uses repeated pulses, if ultrasonically-deposited
circulation is causing deformation within the lungs, longer \acp{PD}
may allow for greater deformation and increased circulation deposition
as a result of any individual pulse. If the system acts as we have
modeled it, the \ac{PRF} would determine the degree of interface
deformation experienced by pulses subsequent to the first and may
influence deformation and hemorrhage. Finally, in recognition of the
limitations of this study, we note that the true physical nature of
lung tissue is viscoelastic \citep{Bayliss1939}, and neither viscosity
nor elasticity is included in our model problems. While preliminary
results with a Navier-Stokes code showed similar early time results,
we expect that viscosity would dissipate circulation over a long
enough period of time. Furthermore, elasticity may provide a mechanism
by which the alveolar walls could resist deformation or retard to
their original shape between pressure perturbations.

In the context of \ac{DUS}, which uses repeated pulses, if
ultrasonically-deposited circulation is causing deformation within the
lungs, longer \acp{PD} may allow for greater deformation and increased
circulation deposition as a result of any individual pulse. If the
system acts as we have modeled it, the \ac{PRF} would determine the
degree of interface deformation experienced by pulses subsequent to
the first and may influence deformation and hemorrhage. Finally, in
recognition of the limitations of this study, we note that the true
physical nature of lung tissue is viscoelastic \citep{Bayliss1939},
and neither viscosity nor elasticity is included in our model
problems. While preliminary results with a Navier-Stokes code showed
similar early time results, we expect that viscosity would dissipate
circulation over a long enough period of time. Furthermore, elasticity
may provide a mechanism by which the alveolar walls could resist
deformation or retard to their original shape between pressure
perturbations.



%-----------------------------------------------------------------------


% %
% In this section we present the results of the numerical experiments
% and compare them to our analysis. We briefly touch on the general
% behavior of the acoustic waves, during the experiments, then go on to
% discuss the interface dynamics associated with the trapezoidal
% acoustic waves. We present results to illustrate the behavior of the
% interface during and after interactions with the acoustic waves. We
% compare the late time interface growth to the scaling law we obtained
% for purely circulation-driven interface growth based on dimensional
% analysis (Relationship \eqref{eq:intf_circ_scaling}). We additionally
% provide plots of the half-domain circulation as a function of time and
% contours of vorticity to show that the compression and expansion waves
% deposit vorticity at the interface. We further plot the individual
% advective, compressible, and baroclinic contributions
% \eqref{eq:circulation_generation_components} to the circulation
% generation equation \eqref{eq:circulation_generation} as functions of
% time to demonstrate the specific physical mechanisms responsible for
% generating circulation at teach stage of the interaction. We next
% investigate the dependence of the interface and circulation dynamics
% on the time dependent features of the wave by varying the lag time
% between the compression and expansion portions of the trapezoidal
% wave. Then we present circulation and interface results for the
% \ac{US} pulse waveform case for comparison to the trapezoidal
% wave cases. Lastly, we discuss broadly some of the implications of the
% results as a whole.

% \subsection{Acoustic wave behavior}

% Trapezoidal and \ac{US} pulse waves (see Figure \ref{fig:p0})
% propagate from water toward the perturbed water-air interface. Nearly
% all ($>99.99\%$) of the acoustic energy is reflected back into the
% water. The sign of the reflected wave is opposite that of the incoming
% wave due to the movement of the incoming wave from media of higher to
% lower acoustic impedance, or simply, compression waves reflect
% expansion waves and vice versa. Due to the strong impedance mismatch,
% a very much weakened acoustic wave, with shape similar to the initial
% acoustic wave condition, is transmitted into the air. The curvature of
% the interface combined with the sound speed change across the
% interface causes slight redirection of the transmitted wave in
% accordance with Snell's law. Reflected and transmitted waves dissipate
% at the inflow and outflow boundaries.

% \subsection{Interface response to trapezoidal acoustic waves} \label{subsec:usbe_lung_trapz_results}
% \subsubsection{Qualitative observations for the $p_a=10$ MPa trapezoidal wave case}
% We first show typical interface amplitude and circulation histories to
% provide a qualitative understanding of the physics. For each
% trapezoidal wave case we observe that the interface begins to compress
% (i.e., the interface amplitude $a(t)$ decreases) when contacted by the
% wave and continues to deform throughout and after the interface-wave
% interaction period. At some point during this process the perturbation
% undergoes a phase change and the begins to grow in amplitude. This is
% consistent with the previously discussed \ac{RMI} for the case of a
% shock moving from a heavy fluid to a light fluid.

% To illustrate the evolution of the interface Figure
% \ref{fig:interface_snapshots} shows snapshots of the density (Top) and
% vorticity (Bottom) fields at different points in the flow's evolution
% for the case of a $10$ MPa trapezoidal wave impinging on the water-air
% interface. In the density plots, areas of high density (i.e., water)
% are dark blue and areas of low density (i.e., air) are white. From
% these contours we see that the initially smooth interface perturbation
% grows from a smooth sinusoid to a sharp spike at late time. On the
% vorticity contours, positive (counterclockwise) vorticity is red, and
% negative (clockwise) vorticity is blue.  When we compare the density
% and vorticity contours, we see that the vorticity is heavily
% concentrated in the air and remains so throughout the flow.

% Figure \ref{fig:trapz10_circ_interface} shows the
% early-time interface amplitude and half-domain circulation histories
% for the same case. At $t_1=0^+$ the compression portion of the wave
% first hits the interface. The interface begins to compress and the
% perturbation amplitude decreases. From $t_1$ to $t_2$ the half-domain
% circulation $\Gamma$ rises sharply. At $t_2\approx1.1$, the
% compression portion of the wave has passed, the interface amplitude
% continues to decrease. The half-domain circulation $\Gamma$ stops its
% rapid growth and changes little during this static elevated pressure
% period, until the expansion wave hits at $t_3$. At $t\approx 5.0$, the
% perturbation undergoes a phase inversion and begins to grow. At
% $t_3\approx8.5$ the expansion wave first hits the interface. The
% perturbation amplitude continues to grow, and $\Gamma$ increases
% sharply again. At $t_4\approx9.7$ the acoustic wave has finished
% traversing the interface, and atmospheric pressure is resumed.
% %
% \begin{figure}[h] 
%   \centering
% \includegraphics[width=0.9\textwidth]{./figs/lung_figs/snapshots_t1}
% \caption[The evolution of the acoustically perturbed interface and vorticity field]{Surface plots of density (Top) and vorticity (Bottom)
%   throughout the evolution of the interface for the $10$ MPa
%   trapezoidal wave case. Areas of high density (i.e., water) are
% indicated in dark blue. Areas of low density (i.e., air) are indicated
% in white.  Positive (counterclockwise) vorticity is indicated in red,
% and negative (clockwise) vorticity can be seen in blue.}
%   \label{fig:interface_snapshots}
% \end{figure}
% %
% \begin{figure}[h] 
%   \centering
%   \includegraphics[width=0.48\textwidth]{./figs/lung_figs/trapz10_intf_schematic}
%   \includegraphics[width=0.48\textwidth]{./figs/lung_figs/trapz10_circ_schematic}
%   \caption[The interface amplitude and circulation histories for the $10$ MPa trapezoidal wave]{The interface amplitude (left) and circulation (right)
%     histories corresponding to the $10$ MPa trapezoidal waves are
%     shown for $t\leq25$. Indicated times, $t_{1-4}$, are the times at
%     which different stages of the incoming trapezoidal pressure wave
%     shown in Figure \ref{fig:p0} first encounter the interface.}
%   \label{fig:trapz10_circ_interface}
% \end{figure}
% %
% \subsubsection{Dependence on wave amplitude}%
% \label{subsubsec:usbe_lung_amplitude_dependence}%
% To illustrate the effects of varying the trapezoidal wave amplitude,
% while keeping the duration of each wave feature constant, we show
% interface amplitudes and half-domain circulation histories for
% $p_a=1$, $5$, and $10$ MPa trapezoidal waves. In Figure
% \ref{fig:trapz_circ_interface_early}, we look closely at the period
% around the wave interaction for $0 \leq t\leq 25$. We note that the
% for the $p_a10$ MPa case, the phase reversal of the interface happened
% around $t=5.0$, which is about haft the time it took for this to occur
% for the for the $p_a=5$ MPa case. The $p_a1$ MPa case, the evolution
% of the interface is sufficiently slow as to not phase invert during
% the period shown. For each wave amplitude, the circulation is
% normalized by the $p_a$ to show that the circulation generated by the
% interface-compression wave interaction, $0^+<t<scales<0.12$, increases
% linearly with $p_a$. To show the longer term effects of varying
% amplitude and show the late time behavior of the interface, defined as
% the behavior significantly after the acoustic waves have left the
% domain, Figure \ref{fig:trapz_circ_interface_loglog} shows the
% interface amplitude and half-domain circulation histories for
% $0 \leq t\leq 500$ as functions of time for $5$ and $10$ MPa
% trapezoidal waves impinging on the interface. Here, the interface
% amplitude is plotted on logarithmically-scaled axes. From scaling law
% \eqref{eq:intf_circ_scaling} we expect that for purely circulation
% driven interface growth, $a(t)$ will grow with $\sqrt{t}$, which is
% shown by dashed lines of corresponding colors. For the $10$ MPa wave
% case the slope of the observed growth appears to grow as $t^{0.6}$.
% Longer time simulations will be used to see if this settles to the
% expected $\sqrt{t}$ growth. Note that results for the $1$ MPa
% trapezoidal wave were not included because the slow evolution of the
% interface made the computation prohibitively expensive.
% %
% \begin{figure}[h] 
%   \centering
%   \includegraphics[width=0.48\textwidth]{./figs/lung_figs/interface_multi-amp_norm}
%   \includegraphics[width=0.48\textwidth]{./figs/lung_figs/circulation_multi-amp_norm}
%   \caption[The interface and circulation dependence on wave amplitude at early time]{The interface amplitude (left) and circulation (right)
%     histories corresponding to the $1$(yellow), $5$(orange), and
%     $10$(blue) MPa trapezoidal waves are shown for $t\leq 25$. The
%     circulation history is normalized by the acoustic amplitude of the
%     incoming wave to illustrate that circulation deposition by the
%     compression wave scales linearly with $p_a$ }
%   \label{fig:trapz_circ_interface_early}
% \end{figure}
% %
% \begin{figure}[h] 
%   \centering
%   \includegraphics[width=0.48\textwidth]{./figs/lung_figs/interface_multi-amp_loglog_roe_extra}
%   \includegraphics[width=0.48\textwidth]{./figs/lung_figs/circulation_multi-amp2_roe}
%   \caption[The interface and circulation dependence on wave amplitude
%   at long time]{The interface amplitude (left) and circulation (right)
%     histories corresponding to the $5$(orange) and $10$(blue) MPa
%     trapezoidal waves are shown for $t\leq 500$. To appropriately
%     compare late time dynamics, time has been offset in the interface
%     amplitude history such that the phase reversal appears to occur
%     simultaneously in both simulations. Dashed lines of the same color
%     are used to demonstrate the expected slope of pure circulation
%     driven interface growth, based on Equation
%     \eqref{eq:intf_circ_scaling}. The red dashed line shows the slope we
%     appear to be approaching for the $10$ MPa wave case for the end time.}
%   \label{fig:trapz_circ_interface_loglog}
% \end{figure}
% %
% \subsubsection{Circulation and vorticity dynamics}
% We observe that the wave deposits a sheet of vorticity along the
% interface that moves with the interface in time. Figure
% \ref{fig:trapz_ddt_circ} shows a surface plot of vorticity in the
% region of the domain around the interface for the $10$ Mpa trapezoidal
% wave case, at $t=0.6$, during the middle of the interface-compression
% wave interaction (Left). Not shown is the rest of the domain, where
% vorticity was relatively insignificant. The vorticity is antisymmetric
% across the $x=0.5$ center line. To analyze the physical mechanisms
% generating the vorticity, we plot each term of the circulation
% generation equation \eqref{eq:circulation_generation} during the
% period around the compression wave-interface interaction. Near the end
% of the interaction at $t=1.0$,
% $\left(\partial \Gamma/\partial t\right)_{advective} =
% 5.3\,\text{e}{-5}$; %
% $\left(\partial \Gamma/\partial t\right)_{compressible} =
% -2.7\,\text{e}{-5}$; %
% $\left(\partial \Gamma/\partial t\right)_{baroclinic} =
% -7.7\,\text{e}{-3}$; %
% $\left(\partial \Gamma/\partial t\right)_{total} =
% -7.7\,\text{e}{-3}$. %
% This result is quantitatively consistent with expected vorticity generation
% based on our analysis \eqref{eq:vorticity_comparison}. Furthermore, it
% supports our hypothesis that vorticity is primarily baroclinically
% generated. 
% %
% \begin{figure}[h] 
%   \centering
%   \includegraphics[width=0.35\textwidth]{./figs/lung_figs/vorticity2}
%   \includegraphics[width=0.48\textwidth]{./figs/lung_figs/ddtcirc}
%   \caption[The vorticity field and individual contributions to circulation by physical mechanism]{A surface plot of vorticity for the $10$ MPa trapezoidal
%     wave case, at time $t=0.6$, during the middle of the
%     interface-compression wave interaction (Left). Each term of the
%     circulation generation equation \eqref{eq:circulation_generation} is plotted as a function of time:
%     $\left(d\Gamma/dt\right)_{advective}$ (blue),
%     $\left(d\Gamma/dt\right)_{compressible}$ (orange),
%     $\left(d\Gamma/dt\right)_{baroclinic}$ (yellow),
%     $\left(d\Gamma/dt\right)_{total}$ (purple, dotted) is plotted as a
%     function of time (Right).}
%   \label{fig:trapz_ddt_circ}
% \end{figure}
% %
% \subsubsection{Dependence on time-dependent wave features: time lag between compression and expansion waves}
% To demonstrate the importance of time-dependent wave features, we
% simulate $p_a=10$ MPa trapezoidal waves of varying duration impinging
% onto the water-air interface. The compression and expansion portions
% of the waveform are exactly the same as is in the other trapezoidal
% wave cases, with pressure rising and falling over an initial distance
% of $5\lambda$. We vary the duration of interaction between interface
% and the elevated static pressure portion of the wave, we will consider
% in terms of the static portion of the wave's initial length, defined
% as $\Delta x_{lag}$. We decrease this duration from the typical
% $\Delta x_{lag}=35\lambda$ to
% $\Delta x_{lag}=25\lambda, 20\lambda, 15\lambda, 5\lambda,$ and
% $0\lambda$. For each of these cases the system dynamics are virtually
% identical to the original case until the expansion encounters the
% interface. Figure \ref{fig:trapz_circ_interface_multi-lag} shows the
% interface amplitude and circulation histories for each case. For the
% three longest duration trapezoidal waves, with static elevated
% pressure durations of $\Delta x_{lag}=35\lambda, 25\lambda$ and
% $20\lambda$, we note that the expansion encounters the interface after
% the phase reversal has already occurred. In these cases, the expansion
% deposits additional circulation at the interface. For the shorter
% duration waves, with static elevated pressure durations of
% $\Delta x_{lag}=10\lambda, 5\lambda$ and $0\lambda$, the expansion
% encounters the interface before the phase inversion and the net
% half-domain circulation is decreased. We note that before or after the
% phase change of the interface, the larger $a(t)$ is at the time the
% expansion encounters the interface, the more circulation is generated
% by the wave, though this does not necessarily hold true across the
% phase inversion.
% %
% \begin{figure}[h] 
%   \centering
%   \includegraphics[width=0.48\textwidth]{./figs/lung_figs/interface_multi-lag}
%   \includegraphics[width=0.48\textwidth]{./figs/lung_figs/circulation_multi-lag}
%   \caption[The interface and circulation dependence on wave duration]{The interface amplitude (left) and circulation (right)
%     histories for varying elevated static pressure durations or lag
%     time $\Delta x_{lag}$ between the expansion and compression
%     waves. Here we show results for $\Delta x_{lag}=35\lambda$ (blue),
%     $\Delta x_{lag}=25\lambda$ (orange), $\Delta x_{lag}=20\lambda$
%     (yellow), $\Delta x_{lag}=10\lambda$ (purple),
%     $\Delta x_{lag}=5\lambda$ (green), $\Delta x_{lag}=0\lambda$
%     (light blue)}
%   \label{fig:trapz_circ_interface_multi-lag}
% \end{figure}
% %
% \subsection{Interface response to \acf{DUS} waves}%
% \label{subsec:usbe_lung_trapezoidal_results}%
% To evaluate the relevance of our trapezoidal wave experiments we simulate
% a $p_a=1, 5$ and $10$ MPa \ac{DUS} pulse waves (See Figure
% \ref{fig:p0}) impinging onto the water air interface. In figure
% \ref{fig:us_circ_interface} we illustrate the circulation and
% interface amplitude histories for the $p_a=10$ MPa \ac{DUS} like-pulse
% case. The post-wave interface dynamics are similar to those observed
% for trapezoidal wave cases. During the wave-interface interaction, the
% interface amplitude is compressed overall, but oscillations are
% observed in correspondence with the acoustic pulse oscillations. After
% the wave has left the interface, the perturbation amplitude continues
% to decrease until the interface undergoes a phase inversion, after
% which the perturbation amplitude grows for the remainder of the
% simulation. half-domain circulation oscillates during wave-interface
% interaction before settling to a nearly constant non-zero value after
% the wave has passed. We note that the total circulation deposited is
% of the same order of magnitude as that generated by the trapezoidal
% wave of the same amplitude and duration. Qualitatively similar results
% were observed for the $5$ MPa case. For the one $1$ MPa case, the
% evolution of the system was slow such that running the simulation long
% enough to obtain useful results was computationally prohibitive.

% \begin{figure}
%   \centering
%   \includegraphics[width=0.48\textwidth]{./figs/lung_figs/us_intf_schematic} \hfill
%   \includegraphics[width=0.48\textwidth]{./figs/lung_figs/us_circ_schematic}
%   \caption[The interface amplitude and circulation histories for the \ac{DUS} pulse]{The interface amplitude (left) and circulation (right)
%     histories corresponding to the a water-air interface disturbed by
%     the US-like pulse shown in Figure \ref{fig:p0}.}
%   \label{fig:us_circ_interface}
% \end{figure}

% \subsection{Further discussion of the results}%
% \label{subsec:usbe_lung_further_discussion}%
% For both the trapezoidal and \ac{DUS} pulse acoustic waves, the
% pressure, velocity, and density return to initial, ambient conditions
% after the passing of the wave. As these waveforms are continuous, this
% implies that the integral of the pressure gradient $\nabla p$ at each
% point along the interface, over all time must be zero. Hence we
% surmise that if the interface remains unchanged during the interaction
% with the wave, as it would for a wave moving with infinite velocity,
% $\nabla \rho$ remains constant and the net baroclinic circulation
% deposited must be zero. Thus for any finite duration acoustic wave
% such as ours to deposit net baroclinic circulation upon an interface,
% the interface itself must deform during interaction with the
% wave. This deformation alters the misalignment of the pressure and
% density gradients at the interface causing positive and negative
% circulation deposited to not cancel out entirely. Note that this is
% unique to waves that begin and end at the same pressure. This is not
% the case for the traditional \ac{RMI} problem, for which conditions do
% not return to their original state after the passage of the shock.

% For the cases varying the length of the static elevated pressure in
% the trapezoidal wave we previously noted that whether the expansion
% increased or decreased the total half-domain circulation depended on
% whether it encountered the interface before or after the phase
% change. If indeed circulation is driving the deformation of the
% interface, then changes in the waveform that appear to have very
% little effect on the interface dynamics during the wave-interface
% interaction period, may have far more significant impacts on the long
% term dynamics of the interface. To put this in the context of
% \ac{DUS}, which uses repeated pulses, if ultrasonically-deposited
% circulation is causing deformation within the lungs, longer \acp{PD}
% may allow for greater deformation and increased circulation deposition
% as a result of any individual pulse. If the system acts as we have
% modeled it, the \ac{PRF} would determine the degree of interface
% deformation experienced by pulses subsequent to the first and may
% influence deformation and hemorrhage. Finally, in recognition of the
% limitations of this study, we note that the true physical nature of
% lung tissue is viscoelastic \citep{Bayliss1939}, and neither viscosity
% nor elasticity is included in our model problems. While preliminary
% results with a Navier-Stokes code showed similar early time results,
% we expect that viscosity would dissipate circulation over a long
% enough period of time. Furthermore, elasticity may provide a mechanism
% by which the alveolar walls could resist deformation or retard to
% their original shape between pressure perturbations.




%%% Local Variables:
%%% mode: latex
%%% TeX-master: "../../prelim"
%%% End:
