\section{Conclusions} \label{section:asuq_astats_conclusions}
This paper describes the area statistics technique for efficiently
estimating transmission loss (TL) uncertainty in underwater
acoustics. The technique is based on the idea that the TL variation
found near the point interest in real space is similar to that found
at the location of interest when environmental parameters are
varied. The technique is simple and can be used to produce approximate
PDFs of TL in uncertain ocean sound channels from a single (baseline)
TL field calculation completed using the most probable value for each
uncertain parameter. To implement the technique, TL values near a
location of interest in the baseline TL field are collected and sorted
into a histogram that is normalized to obtain an approximate PDF of TL
at the location of interest. To determine the technique's accuracy,
PDFs of TL created using area statistics were compared to PDFs
generated using 1000-sample Monte Carlo calculations in four different
ocean environments at three acoustic frequencies (100, 200, and 300
Hz) for three different source depths (91m, 137m, 183m). The
area-statistics PDFs of TL achieved engineering-level accuracy ($\leq$0.5)
in 93\% of test cases in the three shallower environments with
consistent bottom reflection. In the environments where refraction was
more important, area statistics was less successful; engineering level
accuracy was only achieved in 56\% of test cases, initially.  However,
this success percentage was improved to 65\% by gently modifying the
area statistics algorithm, and this modification did not affect the
results in the shallower ocean environments.

The effort reported here supports the following four conclusions. (1)
The area statistics technique is a viable alternative, or worthy
complement, to Monte-Carlo calculations or other more computationally
intensive techniques for estimating the uncertainty of TL field
calculations in uncertain ocean environments with consistent
downward-refraction and bottom reflection. In each of the three
environments of this investigation meeting this bottom reflection
criterion, the technique produced engineering-level accuracy at 85\%
or more of the test locations. (2) The area statistics algorithm is
simple enough that it can be modified to improve the technique's
overall performance. One simple algorithm adjustment improved the
engineering accuracy success rate of area statistics in the deepest
environment considered in this study by approximately 10\% at all
three frequencies. (3) The area statistics technique is so inexpensive
computationally that it should be implemented even when a more
reliable but more computationally demanding approach is the primary
means for TL uncertainty estimation. As part of this investigation,
the area statistics approach was found to be millions of times faster
than Monte-Carlo calculations. Thus, the computational penalty for
implementing both, if the latter is preferred, is vanishingly
small. Moreover, the technique is computationally inexpensive enough
for use in real time applications. (4) The sample rectangle size, TL
sample weighting, and other implementation details of the area
statistics algorithm described here are likely to need adjustment if
the ocean sound channel uncertainties of interest differ from those
given in Table 1. The area statistics technique is ad-hoc and the
implementation parameters in its current formulation were tuned to
achieve a high percentage of engineering-accurate predictions for
ocean sound channels with the uncertainties specified in Table
1. However, the uncertainties specified in Table 1 are generic and may
serve as a useful starting point for many uncertain ocean sound
channels. Thus, the area statistics formulation provided here may be
broadly applicable.

%%% Local Variables:
%%% mode: latex
%%% TeX-master: "../../prelim"
%%% End:
