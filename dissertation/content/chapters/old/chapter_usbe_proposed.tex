\begin{comment}  Move this stuff to introduction
There has been a significant amount of prior research into ultrasound-induced pulmonary hemorrhage, but in spite of this there are still many questions left unanswered.  Presently the underlying damage mechanism causing the hemorrhage is not currently known.  Research has shown that the underlying mechanism of US-induced pulmonary hemorrhage is non-thermal. \citet{Zachary2006a}, for instance, compared lesions generated with DUS to those generated via laser and found multiple differences in the injured tissue. Other studies have shown that inertial cavitation, the intially suspected mechanism, is also unlikely to be responsible for the US-induced PH. \citet{Raeman1997} injected the UCA Albunex, which is expected to nucleate cavitation when exposed to US, into mice before performing pulmonarry US and showed that the hemorrhage was similar to that observed in control mice injected with saline.  In a later study \citet{Obrien2000} placed mice under increased hydrostatic pressure, to suppress the occurence of inertial cavitation, before exposure to pulmonarry US, and found that hemorrohage was enhanced by the increased pressure. While it is widely suspected that the cause of the hemorrhage is mechanical, the precise underlying mechanism by which acoustic energy is tranduced into mechanical stress and strain in the capillary has remained elusive.  

We hypothesize that sharp pressure gradients in the US wave interact with the strong density gradients at the blood-air barriers in the lungs, in turn generating baroclinic vorticity at the interface between the alveoli and the adjacent capillary sheets.  Furthermore, we propose that this vorticity drives the growth of this interface, causing strain, stress, and ultimately failure.
\end{comment}


The future work investigating ultrasound-induced pulmonarry hemorrhage will be divided into two tasks.  First, further numerical simulations of liquid-gas interfaces with pressure waves will be performed in order to investigate US-generated baroclinic vorticity in the lung as a possible mechanism for hemorrhage. Second a simplified model of ultrasound propagation into the lung will be created.

\section{Numerical investigation of ultrasound-lung interaction}
To investigate the behavior of tissue-air interfaces with DUS waves, the lung will be modeled as a compressible fluid system.  As in the prior work, the alveoli will be modeled as air. The surrounding tissue will be treated as a Newtonian fluid with the density and viscosity of blood and all other relevant properties will be set to those of water. The liquid-gas interface will be modeled as a sinusoid of wavelength $\lambda=200\mu$m, as is consistent with typical alvelolar diameter (ADD CITATION). A series of simulations will be performed in which this liquid-gas interface will be subjected to a pressure waveforms of increasing complexity.

\section{Baroclinic vorticity generation at the alveolar-capillary sheet interface blood-air barrier}
\subsection{Estimations of vorticity generation}
\subsection{Expected interface growth from vorticity at the interface }

\section{Simulations of acoustic wave interactions with a sinusoidal blood-air interface}
\begin{itemize}
\item Proposed experiment 1: Interaction of a sinusoidal blood-air interface with a planar acoustic wave with linearly-increasing amplitude
\item Proposed experiment 2: Interaction of a sinusoidal blood-air interface with a sinudoidal pressure pulse
\item Proposed experiment 3: Interaction of a sinusoidal blood-air interface with a single DUS pulse
\item Proposed experiment 4: Interaction of a sinusoidal blood-air interface with multiple DUS pulses
Space pulses based on accepted experimental procedure and see how the secondary pulses effect the interaface growth.
\end{itemize}

\section{Modeling US propagation into the lung}
I aim to model the propagation of ultrasound waves into the lungs.  Much of the lung is composed of tiny air-filled sacs called alveoli. Due to the large impedance mismatch between these alveoli and their surrounding tissue, the lung is highy acoustically reflective.

\subsection{Propogation of acoustic energy into the lung}
Use linear acoustics to predict the transmitted and reflected acoustic pressure amplitudes for air pockets separated by thin water (blood) membranes.  (1a) Model adjacent alveoli as first as normal planar slabs of air with thickness equal to alveolar diameter ($200\mu$m), seperated by slabs of water (blood) of alveolar membrane diameter ($1\mu$m).  (1b) Then model packed regular triangles, (1c) squares (faces oriented at π/4 relative to incoming plane wave.