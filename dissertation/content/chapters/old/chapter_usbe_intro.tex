Ultrasound (US) is a powerful tool for diagnostic imaging and noninvasive therapy.
And its use in medicine has become ubiquitous over the last several decades.
The reasons for the widespread use of US are many fold.
Ultrasound technology is considerably more portable and cheaper than other forms of diagnostic imaging.
It is also relatively simple to opperate, so it requires less traning then other forms of imaging, such as MRI (FIND SOME CITATION).
In addition to these benefits, Diagnostic UltraSound (DUS) is also considered to be the safest imaging modality for patients.
However, that is not to say that safety is not a concern.
Ultrasound tissue interactions do occur.
And a greater understanding of these interactions and their associated bioeffects is needed for both determing appropriate safety regulations and for developing new US techniques.  
While DUS typically seeks to minimize bioeffects, therapeutic ultrasound (TUS) explicitely seeks to use US to trigger specific bioeffects toward a medically beneficial end.
The work presented here seeks to use computational modeling to investigate two specific problems concerning potential mechanical bioeffects observed during DUS experiments.
Specifically, the first problem is motivated by hemorrage and cell death observed in a variety of scenarios when injected contrast agent microbubbles are exposed to DUS pulses.
To investigate this problem, a model for the contrast agent bubble dynamics is developed, and experimentally measured DUS pulses are used to perturb the bubble.
Second, US-lung interaction is modeled to investigate potential mechanisms for pulmonary hemorrage observed during non-contrast diagnostic imaging.
The overall goal of this work is to provide insight into the physics behind these specific US bioeffects that may be useful in future considerations of US in both research and medicine (REWRITE THIS CRAP).

The study of US bioeffects is a growing area of research that has progressed significantly over the last several decades.
However, the physical nature of US-tissue interactions can be complex and their still much we don't know about the specific mechanisms underlying many of the observed bioeffects.
In order to distintuish between the different types of physical mechanisms associated with observed bioeffects, we can classify these mechanisms into two categories: 1) Thermal and 2) Mechanical.
The first category includes a variety of bioeffects that are thought to result from ultrasound induced heating.  This heating can be either a hinderance or a benefit, depending on what the desired effect is.
One example of thermal bioeffects is tissue ablation brought on by the use of High Intensity Focused Ultrasound (HIFU).
In HIFU, ultrasound waves are focused to a small area, concentrating their energy, which is then absorbed by the tissue as heat, locally inreasing the temperature, in this case to the point of destruction.
The second category, mechanically-induced bioeffects, pertains to any changes in the tissue that occur as a consequence of the stresses and strains induced by the solid, fluid, and viscoelastic dynamics associate with the US-tissue interaction.
It is noted that the bioeffects problems of concern in the present work fit into this mechanical category.
Mechanical bioeffects result from a variety of physical phenomenon, including cavitation and acoustic radiation force and torque.
Presently, the bulk of research on mechanical bioeffect mechanisms has focused on acoustic cavitation, which is thought to be responsible for the majority of observed bioeffects.
Cavitation is the phenomenon in which tensile stresses in a liquid or viscoelastic media trigger the formation, growth, and collapse of a gas/vapor bubble from a prexisting nucleus.
Cavitation nucleii and subsequent bubbles act as a focus points for acoustic energy.
By driving bubble growth and collapse, acoustic energy concentrates and is converted into heat and kinetic energy in the fluid.
The regime in which the physics governing the bubble wall dynamics are dominated by the inertia of the surrounding fluid is referred to as inertial cavitation (IC).
During IC violent collapses with extremely high temperatures and pressures as well as shock-waves and re-entrant jets can occur, and have the potential to cause bioeffects.
Inertial cavitation is well recognized as a destructive force, and is known to be responsible for ... !!FINISH THIS THOUGHT!!



Ultrasound Contrast Agents (UCAs) are gas filled microbubbles that act as scattering surfaces for ultrasound waves.
UCAs were created to be injected into the body to provide greater contrast during US sonography, but since have been developed for including drug delivery, !!ADD UCA USES HERE!!.
UCA microbubbles are typically $1 - 10 \, \mu$m in diameter are wrapped in a lipid or protein coating to stabalize the bubble.  
It has been shown that when perturbed by a DUS pulse, UCA microbubbles can act as cavitation nucleii !!CITATION NEEDED!!.
Contrast-Enhance Ultrasound (CEUS) has also been shown to cause bioeffects, including a variety of bioeffects in various scenarios.  
!!LIST CEUS BIOEFFECTS CITATIONS HERE!!


Pulmonarry hemorrohage (PH) is the only known bioeffect of non-contrast enhanced DUS in mammals.
And while the damage mechanism underlying DUS-induce PH remains unknown, it has been observed for MI as low as !!FILL THIS IN!!. 
This is far below the legal limit for DUS, MI=1.9.
Presently, there are no known instances of negative side effects from DUS imaging of the lungs in humans in a clinical setting.
And for obvious ethical reasons, DUS-induced PH has never been experimenally observed in humans either.
As recently as !!FILL THIS IN!!, DUS imaging of the lungs was impossible due to technological limitations and the acoustically reflective nature of gas-filled lung tissue.
But with with the advancement of US technology our ability to use DUS to image the lung has improved, and the use of US to image the lung has also increased.
The use of DUS for lung imaging has become routine in certain critical care scenarios !!CITE!!.
In order for accurate safety standards to be set, we need a better physical understanding of DUS-induced PH.
!!CITE MILLER AND ANY OTHER STUDIES THAT HAVE SHOWN THIS PHENOMENON!!

Up to this point, there has been a significant amount of work researching the potential mechanisms underlying DUS-induced PH.
!!List animal studies here!!

The present work has sought to understand the interaction between gas-liquid interfaces in the lungs when perturbed with DUS waves.
!!FIX THIS!!We hypothesize that DUS-perturbed interfaces experience stresses and strains capable of causing hemorroage in fragile pulmonary capillaries!!






\begin{comment}
\subsection{Cavitation Intro}
Cavitation is phenomenon by which vapor cavities are generated in a fluid as a result of reduced pressure.
Once created, these cavitation bubbles have a tendency to cyclically collapse and expand.
These days, cavitation and bubble collapse phenomenon are well documented and have been a topic of scientific study for nearly a century \citep{Rayleigh1917}
The bulk of cavitation research has been motivated in one way or another by its ability to cause destruction.
When the inertia of the liquid surrounding a newly formed cavity is the dominant physical mechanism governing dynamics of the bubble wall, the bubble has a tendency to collapse and grow violently.
This particular regime of cavitation behavior is commonly referred to as inertial cavitation.
And through a variety of physical mechanisms related this violent growth and collapse, inertial cavitation has been known to cause significant damage and erosion tosurfaces and structures near collapsing bubbles.
Propellars and pumps are common applications in which cavitation can be a problem, due to the high-velocity, low-pressure flows associated with them.
Because of this potential for damage, engineers have more often than not, sought to avoid cavitation.
n
Up until the 1980s, concerns about the destructive behavior of cavitation were primarily limited to industrial and naval applications in water.
But in 1982, work by \cite{Flynn1982, Apfel1982}, predicted that cavitation may also be an issue of concern in the realm of biomedical ultrasound, due to the pressure fields generated by the ultrasound waves.  
Since then, the role of cavitiation in biomedical ultrasound and its potential for causing unwanted damage and bioeffects have been topics of interest in both study and regulation.

-Apfel and Holland (1991) - calculated the response of free air bubbles subject to 





introduction of cavitation into ultrasound and bioacoustics
work that showed US as potential source of cav bioeffects
background on UCAs
work done on cavitation in tissues

Over the last decades, ultrasound has become an incredibly powerful biomedical tool with a large variety of applicaitons.  In addition to being the most widely used diagnostic imaging tool, ultrasound has become a powerful therapeutic tool as well.  And, while diagnostic ultrasound is generally considered to be completely safe, there are still a few instances in which negative bioeffects have been observed in laboratory settings.  Two specific areas of concern to this work are contrast enhanced ultrasound (CEUS)-induced hemorrage and cell death, and non-contrast diagnostic ultrasound induced pulmonary hemmorhage.  Numerical simulations were used to investigate the physical nature of these bioeffects.  In this work, we will first look at theoretical bubble dynamics at capillary breaching thresholds, and then move on to simulations of a model lung-tissue interface evolving when subjected to a variety of clinically relevant pressure waves.

Ultrasound contrast agents (UCAs) are small gaseous bubbles that act as additional scattering surfaces for ultrasound waves.  They are typically used to image areas of the body where no material interface is present, such as in blood flow.  While UCAs typically start wrapped in a protein or lipid shell, this shell breaks down after a finite amount of time in the body, releasing a free bubble.

\end{comment}