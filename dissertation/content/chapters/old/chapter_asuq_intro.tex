Knowledge of transmission loss (TL) in the ocean is useful in a variety of naval applications.  TL is used for threat assessment, tactical decision aids, and sonar predition.  However, our knowledge of real ocean environments is uncertain.  As a result of this TL calculations made based on our uncertain knowledge of the ocean are also uncertain.  Therefore quantifying TL uncertainty can be useful for making decisions based on calculated TL estimates.

Currently, there are a variety of accepted methods that can be used for quantifying transmission loss uncertainty, however each of them has certain drawbacks.  Given probability distributions for the uncertain ocean parameters, Monte-Carlo (MC) TL field calculations can be performed, randomly sampling uncertain ocean parameters based on their distribtions.  However, obtaining a probability density function (PDF) for TL at a point of interest may require many MC field calculations.  As a result, the computational expense of MC methods is often prohibitive, especially in the case of real-time application.  Polynomial chaos (PC) techniques have also been proposed for propagating input parameter uncertainty through the calculation field calculation in order to calculate TL uncertainty.  These calculations can also be computationally expensive in many instances, and can be extremely difficult, particularly for hyperbolic equations such as the wave equation.  Other less expensive methods such as field shifting and U-band do not bare the computational cost associated with MC and PC based methods, but have limitations that may restrict their utility when used in practical applications.  U-Band is based on mode counting.  SOMETHING ABOUT FIELD SHIFTING HERE.

In our work area statisitics (AS) is developed to estimate TL PDFs based on a single TL field calculation.  In this work we will explain the procedure behind the use of AS for estimating TL uncertainty.  AS is then tested in a variety of ocean environments.  TL PDFs obtained are compared with those obtained from 1000-sample MC calculations.  Four main conclusions are drawn about the use of AS for estimating TL PDFs: (1)  AS is simple for practical use; (2) TL PDFs obtained using AS are in good agreement with those obtained from MC for ocean environments with consistent bottom bounce. (3) (4) AS is fast enough for practical use.  
