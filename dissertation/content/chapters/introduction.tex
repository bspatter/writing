In this chapter we establish the merit and relevance of the presented
work. The problems approached here apply to a variety of active areas
of study and modern applications within the fields of acoustics and
fluid mechanics, though the primary focus and motivation of this work
is to better understand the underlying physics of specific biological
effects of \ac{DUS}. An understanding of the physical mechanisms
underlying \ac{DUS} bioeffects is necessary for evidence-based
regulation. Accordingly, we describe the driving physical mechanisms
of interest in these problems. We also discuss the specific problems
we will be approaching and the framework we use to approach
them. Finally, an overview of the goals and contributions of this part
of the thesis are presented.

\section{A physical description of sound as it relates to this work}
Sounds are vibrations traveling through a medium. Parcels of material
perturbed or displaced collide with their neighbors, which
collide with their neighbors and so on. In this way, mechanical energy
propagates as a wave at a finite speed, away from the initial
perturbation location, through any gas, liquid, or solid medium. This
is the basic mechanism by which sound moves through all matter whether
it be the tissues in the human body, the water in the oceans, or the
plasma in the stars.

The scientific study of sound, in all its many forms, is what we refer
to as \emph{acoustics}. Through years of study and experimentation,
mankind has gained a deep understanding of the physical behavior of
sound and has learned to harness it as a tool, leading to high-impact
advancements throughout \ac{STEM} in areas ranging from climate change
(by monitoring the ocean's acoustic properties) to structural health
monitoring and diagnostic and therapeutic medicine. Much of our basic
understanding of sound has come from the theoretical study of sound
passively propagating through a constant, infinite, homogeneous
medium. However, in reality no such medium exists and sound is not
always passive. Indeed many of the interesting physical questions and
real-world applications of interest to modern acousticians are
concerned with the scenarios in which sound is traveling through a
complicated medium which it sometimes physically alters. In this
thesis, we hope to advance the study of acoustics by studying a few of
these scenarios.

The focus of this part of the thesis is on problems in which sound
travels between multiple media and causes a physical change in the
system as it travels. Typically, when sound traveling in one medium
encounters another medium, a portion of the acoustic energy is
transmitted into the new medium, while the remainder is reflected and
scattered back into the medium from which the sound originated. In
most cases, this results in little change in the media themselves,
however, in some instances, acoustic energy can be converted into
other forms of energy such as kinetic or thermal, resulting in bulk
motion or heating of the media. Conversion of energy such as this is
often consequence of nonlinearity in the system, which can arise from
physical properties of the system such as a liquid-gas interface, or
from sufficiently strong acoustic waves
($Acoustic\, pressure / [density \cdot sound\, speed^2] \nll 1$). An
example of acoustic energy becoming kinetic as a result of a
nonlinearity in the physical system is a gas-vapor bubble within water
or tissue driven by an acoustic wave. As a result of rising and
falling acoustic pressure, the bubble may oscillate or collapse,
changing the temperature and pressure, and driving the motion within
the bubble and the surrounding medium. The absorption of acoustic
energy into the medium as heat resulting in a temperature rise (due to
localized compression) in a viscous medium is an example of an effect
that can be particularly important for strong, nonlinear acoustic
waves. Additionally, localized compression/rarefaction due to strong
nonlinear acoustic waves can result in an increase in localized sound
speed, causing waves to sharpen into shocks, introducing further
nonlinearity. In any case, the resulting thermal or physical stresses
associated with the heating or movement of the media may result in
physical (e.g., phase transition) or chemical (e.g., protein
denaturation) changes. The ability of acoustic waves to physically
alter a medium is of particular interest and relevance to the field of
medical ultrasound, in which such alteration is relevant to both
safety concerns in the context of diagnostic sonography and
engineering concerns in the context of therapeutic \ac{US}.

\acresetall
\section{Ultrasound in medicine and biological effects} % 
\ac{US} refers to sounds or vibrations with frequencies beyond that
audible to the human ear, typically $>20$ kHz. The use of \ac{US} in
medicine dates back to the 1940s when Austrian neurologist Dr. Karl
Theodore Dussik attempted to use transmission ultrasound to outline
the ventricles of the brain \citep{Dussik1942,Singh2007}. Since then
the abilities and use of \ac{US} have expanded greatly and the
technology has proven to be a powerful tool for noninvasive therapies
and safe, real-time diagnostic imaging
\citep{Dalecki2004}. Consequently, the use of \ac{US} has become
ubiquitous throughout modern medicine. The bulk of the present work
will focus on \ac{DUS}, which is routinely used for noninvasive
imaging of a range of soft tissues including muscles, tendons, organs,
glands and neonatal fetuses.

For context, we explain the basic physical processes that occur during
\ac{US} procedures. In practice, high-frequency, typically MHz-range,
acoustic waves and pulses are created at the surface of the body using
a piezoelectric \ac{US} transducer. These acoustic waves, propagate
via an impedance matching, acoustic coupling medium from the
transducer into the tissue. Once in the tissue, a portion of the sound
scatters at material interfaces within the body, or more simply, some
of the sound echoes whenever it moves from one tissue to another or
encounters a cavity in the body. More precisely put, a portion of the
sound is reflected when it encounters a change in the acoustic
impedance, defined as the product of the density and sound speed of
the medium. This scattering of sound is the basic physical principle
that makes ultrasound for diagnostic imaging possible. In \ac{DUS},
some of the reflected sounds encounter a receiver. This receiver is
typically also a piezoelectric transducer, and much like the process
of generating the wave, but in reverse, the receiver is vibrated and
converts the acoustic signal to a series of electrical pulses. The
signal is amplified, recorded, transmitted to an ultrasound scanner or
another computer where it is processed. The strength of the received
signal is indicative of the impedance mismatch at the reflective
surface and is indicated as brightness in an ultrasound image. The
timing of the receipt of the reflected signal across the receiver is
determined by the shape of the reflecting surface. By processing this
information, a real-time image of the reflecting surface is
generated. An example ultrasound image of a fetus is shown in Figure
\ref{fig:fetus_example}.

\begin{figure}
  \centering
  \includegraphics[width=0.65\textwidth]{./figs/intro_figs/ultrasound_example}
  \caption[An example diagnostic ultrasound image of a featus at 12
  weeks in a sagittal scan.]{An example diagnostic ultrasound image of
    a featus at 12 weeks in a sagittal scan. Author Wolgang Moroder
    [CC BY-SA 3.0 (http://creativecommons.org/licenses/by-sa/3.0),
    via Wikimedia Commons]}
  \label{fig:fetus_example}
\end{figure}

The passage of acoustic waves of diagnostically relevant amplitudes
through tissue does not typically permanently alter or affect the
tissues structures or processes and the use of ultrasound for imaging
is typically considered safe and noninvasive. However, the passage of
acoustic waves through a medium is not entirely passive under all
circumstances \citep{Nyborg2001}. When energy from ultrasound is
converted to kinetic or thermal energy, within a tissue, it can
physically alter or damage that tissue through a variety of mechanisms
\citep{OBrien2007}. These effects are referred to as \ac{US}
bioeffects and can be beneficial or detrimental depending on their
exact nature. In therapeutic applications, \ac{US} is used to
deliberately cause desirable bioeffects that are beneficial to the
patient. \ac{DUS} is typically designed to minimize interaction
between the acoustic field and tissue \citep{Dalecki2004} (as per
United States Food and Drug Administration Regulation), and bioeffects
are generally undesirable side effects that are avoided if
possible. Ultrasound bioeffects have motivated extensive research into
the development of effective guidelines and regulations for safe
\ac{US} technologies and procedures.

A large portion of past research into ultrasound bioeffects has
focused on determining what types of \ac{US} bioeffects exist, and
under what circumstances they occur. This work has shown that
bioeffects may take on a variety of different forms, depending on the
\ac{US} parameters and type of tissue exposed
\citep{NCRP2002}. Various kinds of hemorrhage and cell death are among
the most common forms of \ac{US} bioeffects. In tissues containing
gases such as the lung and intestines, ultrasonically induced
hemorrhage has been observed. \cite{Lehmann1953} and \cite{Miller1994}
observed abdominal petechial hemorrhage as a result of unfocused
ultrasound in mice. And \cite{Child1990} found hemorrhage in mouse
lungs after the animal was exposed to lithotripter pulses. Numerous
other studies have been performed on the topic of US-induced lung
hemorrhage and a much deeper review is given in chapters
\ref{ch:usbe_lung} and \ref{ch:usbe_lung_bio}. Pulsed ultrasound of
the heart has been shown to be capable of inducing cardiac
contractions in frogs and mice \citep{Dalecki1993,MacRobbie1997}. Cell
death has been observed in liver, kidney, and heart tissue as a result
of \ac{CEUS}, which uses injections of microbubbles as
additional scatterers for image contrast\cite{Skyba1998, Miller2008a}.

\subsection{Physical mechanisms of ultrasound bioeffects and problem
  description} \label{sec:bioeffects_mechanisms}%
Depending on the type of physical damage mechanism responsible,
\ac{US} bioeffects are classified into two groups, thermal and
non-thermal \citep{Dalecki2004}. While both thermal and non-thermal
bioeffects may occur simultaneously, one or the other is often
dominant. The first group, thermal bioeffects, are characterized by
deposition of acoustic energy into tissue as heat and are often a
result of therapeutic, rather than diagnostic, ultrasound. This
heating can lead to a variety of deleterious effects including the
release of highly reactive free radicals, protein denaturation at the
molecular level, and death at the cellular level, ultimately causing
tissue damage or death. As an example, one class of therapeutic
\ac{US}, known as \ac{HIFU} uses strong, concentrated acoustic waves
to intentionally convert acoustic energy to heat through viscous
dissipation and thermal diffusion. \ac{HIFU} is typically at MHz order
frequencies and intensities up to 10,000 W/cm$^2$. \ac{HIFU} is used
to raise the temperature of unwanted tissues such as fat or cancer to
the point of destruction via thermal necrosis
\citep{Escoffre2016}. Little else will be said about thermal
bioeffects, as the bioeffects of interest here fall into the
non-thermal category. Non-thermal bioeffects are attributed to a
variety of physical phenomena including acoustic radiation force,
radiation torque, and acoustic streaming, though the bulk of
non-thermal bioeffects are commonly associated with acoustic
cavitation, which is the most widely studied non-thermal mechanism
\citep{Dalecki2004}. For certain bioeffects, such as \ac{DUS}-induced
lung hemorrhage, the underlying physical mechanisms are largely
unknown.

The work of this thesis is primarily motivated by \ac{DUS} bioeffects.
\ac{DUS} bioeffects tend to be a result of mechanical processes and
typically take the form of hemorrhage, tissue damage or cell
death. Unlike some bioeffects that occur as a result of therapeutic
\ac{US}, \ac{DUS} bioeffects are unintentional and a represent a
potential safety concern. Hence in this thesis we seek to develop a
better understanding of two particular \ac{DUS} bioeffects problems,
\begin{enumerate}
\item The first problem is motivated by \ac{CEUS} and the associated
  cavitation-induced biological effects, which include hemorrhage and
  cell death in a variety of forms (e.g., cell death and/or
  hemorrhage in the heart, kidney, muscle, etc...)
  \citep{Miller2008a}. We pursue of a better understanding of the
  relationship between \ac{US} thresholds associated with bioeffects
  and the physical dynamics of the system. To this end we take a novel
  approach, combining experimentally measured ultrasound and
  bioeffects thresholds with modeling and simulation. We model the
  problem of an ultrasound contrast agent microbubble subjected to an
  experimentally measured ultrasound pulse and simulate relevant
  bubble dynamics. Calculated cavitation dynamics are related to known
  bioeffects and thresholds, associated with properties of the driving
  waveforms.
\item The second problem we consider is that of \ac{DUS}-induced lung
  hemorrhage. As the underlying physical mechanism that drives the
  hemorrhage is not clearly understood \citep{OBrien2007}, we aim to
  gain a better conceptual understanding of the physics at play. To
  accomplish this, we develop a unique model of the interaction
  between an alveolus and an acoustic wave, traveling in soft
  tissue. The alveolar wall is modeled as an air-water interface,
  driven by trapezoidal and ultrasound pulse-like acoustic waves. We
  analyze the dynamics of the system to describe the interface
  evolution mathematically. Where possible we compare calculated
  stress and strain estimates of the interface with alveolar failure
  criteria for disruption of endothelial and epithelial tissues.
\end{enumerate}

Given these bioeffects problems, we will now provide a brief overview
of what is known about the physical mechanisms of \ac{CEUS} cavitation
bioeffects and \ac{DUS}-induced lung hemorrhage. We will also include
a description of a proposed mechanism for \ac{DUS}-induced alveolar
hemorrhage.

\subsubsection{Cavitation of ultrasound contrast agent
  microbubbles} \label{subsec:bioeffects_mechanisms_ceus}%
Acoustic cavitation is the phenomenon by which gas nano and
microbubbles, called cavitation nuclei, are cyclically grown by low
pressures and collapsed by high pressures of the field. When the
bubble dynamics during collapse are dominated by the inertia of the
surrounding fluid, the process is called \ac{IC}. \ac{IC} is typically violent,
with the bubble rapidly collapsing to a fraction of its original size
resulting in calculated internal pressures ranging from $100$ to
$7000$ MPa and temperatures from $1000$ to $20 000$ K
\citep{Flynn1982}. There are several possible damage mechanisms
associated with \ac{IC} that may be responsible for observed \ac{US}
bioeffects. Due to the pressure difference between the vapor/gas
mixture within the bubble at collapse and the surrounding medium, the
collapsed bubble can emit a powerful shock wave that can be damaging
to the bubble's surroundings. When cavitation is triggered near a
rigid surface, the bubble can collapse in a radially asymmetric
fashion causing a high-speed ``re-entrant'' jet of liquid to impinge
upon the surface, effectively striking the surface with a liquid
hammer. If cavitation occurs at an appropriate distance from a
non-rigid surface, such as soft tissue boundaries and blood vessel
walls, the jet can impinge away from the surface, potentially
invaginating the surface \citep{Brujan2011}. The resulting stresses
and strains can result in structural damage. Figure
\ref{fig:intro_cavitation_schematic} schematically illustrates
potential cavitation damage mechanisms within a blood vessel.
\begin{figure}
  \centering
  \def\svgwidth{0.9\textwidth}
  \import{./figs/intro_figs/}{Cavitation_schematic.pdf_tex} \hfill%
  \caption[Schematic of possible ultrasound bioeffects
  mechanisms]{Schematic of possible cavitation-induced ultrasound
    bioeffects mechanisms. (Left) A microbubble within a blood vessel
    interacts with an ultrasound pulse. (Middle) Subsequently, the
    bubble undergoes cavitation. (Right) A variety of possible
    cavitation bubble dynamics scenarios are potential bioeffects
    damage mechanisms (from top to bottom): a). Bubble expansion
    beyond the radius of a surrounding blood vessel. b.) A cavitation
    jet away from the wall of a surrounding blood vessel or tissue
    surface causes the surface to invaginate. c.) A cavitation jet of
    high speed liquid strikes a vessel or tissue wall. d.) A shock
    wave created by the bubble collapse encounters nearby tissue.}
  \label{fig:intro_cavitation_schematic}
\end{figure}

While \ac{IC} does not typically occur during non-contrast \ac{DUS},
it is of concern during \ac{CEUS}, which uses contrast-agent
microbubbles injected into patients' bloodstreams to act as additional,
strong scatterers. The high acoustical impedance mismatch between the
gas microbubbles and the surrounding soft tissues allows for high
contrast imaging and can be used to ultrasonically image blood flow,
which is useful for diagnosing heart valve problems, liver lesions,
and more \citep{Claudon2013,Rognin2008}. However, the use of contrast
agent microbubbles can also have potential deleterious side
effects. These microbubbles can act as cavitation nuclei and the
resulting cavitation has been associated with a variety of different
forms of cellular death and damage. The precise ultrasonic thresholds
for which cavitation and bioeffects occur have been a topic of intense
study and are not completely physically described. Furthermore, the
exact physical mechanisms through which cavitation causes bioeffects
are also not clearly understood \citep{Barnett1994}.

As a result of the potential for cavitation-related \ac{US}
bioeffects, the United States Food and Drug Administration called for
a metric to quantify cavitation dosage and predict likely cavitation
damage from ultrasound. As bioeffects are typically attributed to
\ac{IC}, efforts to predict cavitation damage considered the
likelihood of \ac{IC} based on theoretical calculations of free gas
bubbles in water. In the case of acoustic cavitation, the likelihood
of damage depends on the duration of peak negative pressure
experienced by a cavitation nucleus, with longer interactions
depositing more energy into the nucleus, and thus having a greater
likelihood of inducing \ac{IC} and bioeffects. The duration of the
\ac{PRPA} is inversely related to the \ac{US}
frequency. \cite{Holland1989} demonstrated that the threshold
\ac{PRPA} needed to trigger \ac{IC}, defined based on a maximum bubble
temperature $\geq5000$K, depended on the size of the cavitation
nucleus. Smaller cavitation nucleii, must overcome greater surface
tension effects in order to cavitate, with the Laplace pressure
scaling inversely with the radius of the nucleus. Furthermore, as the
initial radius of a nucleus increases, the inertia of the surrounding
fluid that it must be overcome also increases \citep{AiumS72000}. Thus
\cite{Holland1989} illustrated that for a given frequency there is an
optimal nucleus size for triggering \ac{IC}. Based on these
calculations and corrections for heat dissipation in tissue the
\ac{MI} was created as a measure of ultrasound-induced cavitation
related bioeffects and defined as
\begin{align}
  \text{MI} = \frac{P_{r.3}}{\sqrt{f_c}},
\end{align}
where $P_{r.3}$ is the \ac{PRPA} derated by $0.3$ dB/MHz-cm (a soft
tissue attenuation coefficient) and $f_c$ is the center frequency
\cite{Apfel1991}. As the \ac{MI} was originally created based on
theoretical thresholds for inertial cavitation in water, the derated
\ac{PRPA} was used to account for \emph{in vivo} attenuation, however
the effects of tissue's elastic properties are not accounted for by
this metric, and because the cavitation dynamics are expected to
change from tissue-to-tissue a more robust evidence-based metric would
be useful. The United States \ac{FDA} mandates that MI$\leq1.9$ for
diagnostic ultrasound, though \ac{US} bioeffects have been observed at
\ac{MI} below this in the case of \ac{DUS} of mammalian lungs
\citep{OBrien2007,FDA1997}.

\subsubsection{Ultrasound-induced lung hemorrhage}
The second \ac{US} bioeffects topic of interest in this thesis is
\ac{DUS}-induced \ac{LH}. In the relevant literature, this is also
sometimes referred to more specifically as \ac{PCH}. This phenomenon
was first discovered in mice over twenty years ago by \cite{Child1990}
and has since been shown to occur in a variety of other mammals
including rats, pigs, rabbits, and monkeys \citep{OBrien1997a,
  Miller2012, Tarantal1994a}. Research into this phenomenon has
focused on three main areas: (1) Determining the physical mechanism of
the hemorrhage; (2) Understanding how the occurrence and severity of
the hemorrhage depend on the ultrasonic properties (frequency,
amplitude, waveform, etc...); and (3) Understanding how the occurrence
and severity of the hemorrhage depend on the characteristics of
ultrasound subject (species, age, anesthesia, etc...). The work in
this thesis pertains primarily to the first of these three areas.

Despite extensive previous research into \ac{DUS}-induced \ac{LH}, the
underlying physical mechanisms are still not well
understood. Furthermore, past work has shown that common \ac{US}
bioeffects mechanisms do not explain the observed injuries. Thermal
damage mechanisms appear unlikely to be the primary source of damage
as \ac{DUS}-induced lung lesions do not appear similar to those
induced by heat \citep{Zachary2006}. Furthermore, cavitation
mechanisms do not appear to be responsible, as the severity of
\ac{DUS}-induced \ac{LH} in mice increased under raised hydrostatic
pressure \citep{OBrien2000} and was unaffected by the introduction of
\ac{US} contrast agents into both rats and mice subject to lung \ac{US}
\citep{Raeman1997,OBrien2004}. Both of these results are inconsistent
with what is expected of \ac{IC}-induced bioeffects. More recent work
by \cite{Miller2016a,Miller2016} investigating acoustical radiation surface
pressure as a potential damage mechanism found that the pressures
expected in pulsed ultrasound were likely too low to completely
explain the observed hemorrhage on their own. \cite{Simon2012} found
that atomization and fountaining occurred at tissue-air interfaces
subjected to \ac{HIFU} and suggested that this could potentially
happen at diagnostic levels as well. Similarly, works by
\cite{Tjan2007,Tjan2008} model the evolution of an inviscid, free
surface subjected to a Gaussian velocity potential and find that this
can lead to the ejection of liquid droplets. They go on to say that
\ac{DUS} of the lung may similarly lead to the ejected of droplets
capable of puncturing the air-filled sacs within the lung. The problem
of \ac{US}-lung interaction is the central motivation of chapters
\ref{ch:usbe_lung} and \ref{ch:usbe_lung_bio}. As such, a far more
in-depth literature review is provided in these chapters.

%\subsubsection{Acoustically driven fluid-fluid interfaces}
\paragraph{Proposed mechanism for \ac{DUS}-induced lung hemorrhage: vorticity-driven strain of the alveolar walls}
In this thesis we propose yet another potential physical mechanism for
causing \ac{DUS} induced lung hemorrhage, and as such we now provide
the relevant background. The physical problem underlying interactions
between ultrasound waves and the various tissue and fluid layers of
the body is that of a mechanical wave traveling in one medium
encountering a second medium of differing physical properties. As was
previously explained, this can result in acoustic energy being
converted into motion or heat. In the case of the bubble, the relevant
manifestation of this was cavitation. Another manifestation of this is
the growth of perturbations at fluid-fluid interfaces as a result of
non-uniform velocity gradients that occur at the driven
interface. Another way of thinking of this is in terms of baroclinic
vorticity, or localized fluid rotation, generated by the misalignment
of interface density gradients and mechanical wave pressure
gradients. In this dissertation we propose that (1) ultrasound-induced
baroclinic vorticity may drive the growth of perturbations at liquid-gas
interfaces such as those of the alveoli and (2) this perturbation
growth leads to alveolar strain with possible hemorrhage. In the
remaining portion of the section, we discuss in greater detail the
underlying physics at play here and some of the past work that has
been done to understand it.

There has been extensive past research into the physics that underlies
mechanical waves interacting with and accelerating fluid-fluid
interfaces. Much of this work has investigated regimes outside those
of acoustic interests, including applications in fusion energy and
astrophysics. \cite{Taylor1950} predicted that, for an interface
between two fluids of different densities, if the fluid was
accelerated normal to the interface in the heavy-to-light direction,
perturbations at the interface would grow. That is to say that a
``bubble'' of light fluid penetrates the heavy fluid, and a ``spike''
of heavy fluid penetrates the light fluid. This is known as the
\ac{RTI}. A similar topic of past study is the \ac{RMI}, which occurs
when a perturbed fluid-fluid interface is instantaneously accelerated
by a shock, causing the interface perturbation to grow
\citep{Brouillette2002,Drake2006}. This growth is driven by a sheet of
baroclinic vorticity deposited along the interface as a result of
misalignment between the pressure gradient across the shock and the
density gradient across the perturbed interface. This physical
mechanism, by which misaligned gradients create a torque on fluid
particles generating vorticity, can be thought of in terms of a
hydrostatic balance upon a particle. Pressure gradients result in
acceleration of the flow that is inversely proportional to
density. When these two gradients are misaligned, the result is a
shearing effect or velocity differential on the fluid and vorticity is
generated \citep{Heifetz2015}. A graphical explanations of baroclinic
vorticity generation, adapted from \citep{Heifetz2015} is shown in
Figure \ref{fig:usbe_lung_baroclinic_schematic}.
\begin{figure}
  \centering
  \includegraphics[width=0.9\textwidth]{./figs/intro_figs/baroclinic_schematic} \hfill
  \caption[Schematic of baroclinic torque]{Schematic of baroclinic
    torque. From \cite{Heifetz2015}. A force balance upon
    a particle subject to perpendicular pressure and density gradients
    illustrates baroclinic torque on a fluid particle.}
  \label{fig:usbe_lung_baroclinic_schematic}
\end{figure}
Analytically, baroclinic vorticity generation can be shown by taking
the curl of the conservation of momentum equation for a fluid with
variable density. It is worth noting that it is a second order,
nonlinear effect and cannot be explained by traditional linear
acoustics in a uniform medium.

The physics of the classical \ac{RMI} are fairly well understood. The
classical \ac{RMI} setup consists of a planar shock impinging normally
upon the peaks and troughs of a sinusoidal interface. The interface is
accelerated non-uniformly such that counter-rotating vortices are
generated across the interface. This drives peaks and troughs of the
interface to accelerate in opposing directions, normal to the
peak/trough surface. Much like in the case of the \ac{RTI}, this too
results in light fluid penetrating the heavy fluid and vice versa. For
the case of a wave moving from a light fluid into a heavy one, the
peaks and troughs of the interface accelerate away from one another,
growing the interface perturbation. For the case of a wave moving from
a heavy fluid into a lighter fluid, the peaks and troughs interface
initially accelerate toward one another. They then pass each other,
inverting the phase of the interface perturbation, and continue moving
in opposite directions, growing the perturbation amplitude. This
process is illustrated in Figure \ref{fig:rmi_schematic}, which has
been adapted from \cite{Brouillette2002}. This work proposes that
similar physics occur at ultrasonically driven air-tissue interfaces
within the lungs. Much greater detail regarding the proposed mechanism
is provided in Chapters \ref{ch:usbe_lung} and \ref{ch:usbe_lung_bio}.
\begin{figure}
  \centering
  \def\svgwidth{0.9\textwidth}
  \import{./figs/intro_figs/}{brouillette_fig3_mod.pdf_tex} \hfill%
  \caption[Schematic of the \acl{RMI} for a heavy-light
  interface]{Schematic of the \acl{RMI} for a heavy-light
    interface. Adapted from \cite{Brouillette2002}. The initial
    condition (left), circulation post wave-interface interaction
    (center), and perturbation growth (right) are shown.}
  \label{fig:rmi_schematic}
\end{figure}


\section{Tissue as a compressible fluid system}
To investigate \ac{CEUS} and \ac{DUS}-induced lung hemorrhage,
throughout this dissertation we model the relevant physical
problems of ultrasound in human tissue as compressible, multiphase
fluid systems. In this section we attempt to justify this general
approach and explain some of the applicable assumptions and
implications.

The underlying governing equations upon which our models are based are
the general conservation equations for mass, momentum, and energy for
a fluid,
\begin{subequations} \label{eq:intro_conservation}             
  \begin{align}
    \frac{\partial \rho}{\partial t} + \nabla\cdot\left(\rho\bs{u}\right) =& 0,\\%
    \rho\frac{D \bs{u}}{D t} =& \nabla\cdot\bs{\tau}+\rho\bs{g},\\%
    \frac{\partial E}{\partial t} + \nabla\cdot\left(E \bs{u}\right) =& \rho\left(\bs{g}\cdot\bs{u}\right) + \nabla\cdot\left(\bs{u \cdot \tau}\right) + \nabla\cdot\bs{q},%
  \end{align}
\end{subequations}
where $\rho$ is density, $\bs{u}$ is the flow velocity vector, $t$ is
time, $\bs{\tau}$ is the stress tensor, $\bs{g}$ is the body force vector,
$E = \rho \left(e + \frac{1}{2}\left[\bs{u}\cdot\bs{u}\right]\right)$
is the total energy defined as the sum of the kinetic energy per unit
mass $\frac{1}{2}\left(\bs{u}\cdot\bs{u}\right)$ and the internal
energy per unit mass $e$, and lastly $\bs{q}$ is the heat flux
vector. To model ultrasound-tissue interactions, the general
conservation equations \eqref{eq:intro_conservation} are simplified
and manipulated based on the physics appropriate to the specific
problem at hand. The closure of these equations is also treated
differently depending on the particular problem and model. Details on
the appropriate equations of state used to relate pressure and energy,
constitutive equations used to relate stress and strain, and boundary
conditions are described in greater detail in sections
\ref{subsec:usbe_bubble_model} and \ref{sec:methods}.

To consider what physical effects are at play during diagnostic
ultrasound, both contrast-enhanced and of the lung, we consider the
basic physical scenario of each of these problems: an acoustic wave
traveling through a multiphase medium consisting of soft tissue and
gas. Soft tissues are viscoelastic materials, i.e., they exhibit solid
and fluid like behaviors simultaneously under different types of
forcing, i.e, viscosity, elasticity, and relaxation may all be
simultaneously at play. These tissues include blood as well as lung,
liver, and kidney tissue, which are relevant to the motivations of
this thesis. The multiphase nature of these problems suggests that
gas-liquid/gas-viscoelastic interface phenomena such as surface
tension may also be of some relevance. As fluid motion is expected,
inertial effects are likely to be of importance. Additionally, as
ultrasonic heating is a known source of biological effects, we
consider heat transfer and thermal mechanisms as well. And for
completeness, since the vast majority of ultrasound procedures do not
occur on the International Space Station, we consider the effects of
gravity too. In the following, we introduce dimensional analysis to
assess the relative importance of each of these physical phenomena for
the problems we approach in this part of the thesis.

\subsection{Dimensional analysis and assumptions for Contrast-Enhanced Ultrasound}
\ac{CEUS}-related bioeffects are generally attributed to a process
called \ac{IC} in which a bubble or void within a fluid collapses
rapidly. This can result in high temperatures, pressures, stresses,
strains, and strain rates within the surrounding fluid. More details
about this process and its relationship to \ac{US} bioeffects are
provided in Section \ref{subsec:bioeffects_mechanisms_ceus}. In this
work, we consider the problem of a single \ac{US} pulse interacting
with a contrast agent microbubble, initially at rest within a
viscoelastic soft tissue. For the sake of justification we consider a
typical case here. In Chapter \ref{ch:usbe_bubble} a more in-depth
analysis, specific to the work presented here, is performed. Consider an
ultrasound pulse of clinically relevant frequency $f = 3$ MHz and
\ac{PRPA}$=p_a = 1$ MPa. The soft tissue is treated as a Voigt type
viscoelastic material, as in \cite{Yang2005}, and has a nominal
density $\rho = 1000$ kg/m$^3$, an elastic modulus from $G = 5$ kPa to
$1$ MPa, and a dynamic viscosity $\mu = 0.015$ Pa$\,$s, and
corresponding kinematic viscosity $\nu=\mu/\rho=1.5\times 10^{-5}$
m$^2$/s. Surface tension is based on that of blood, such that
$S = 0.056$ N/m \citep{Apfel1991}. Note that the physical properties
of soft tissue vary widely and are poorly characterized, particularly
at the strain rates associated with cavitation. Based on the work of
\cite{Patterson2012} we define a characteristic velocity of
$u = \sqrt{p_a/\rho} = 31.6$ m/s, which corresponds to a change in
radius over a period of free oscillation. As a characteristic length
scale, we use a typical bubble size such that equilibrium radius is
$R_0 = 1\mu$m.

Based on this setup we perform dimensional analysis to assess the
relative importance of each of the potentially relevant physical
mechanisms to the problem of acoustically-driven cavitation in soft tissue:\\

\noindent\textit{Viscosity:} To assess the relevance of viscosity we
consider the Reynolds Number, which is the ratio of inertial to
viscous forces within a flow and is defined as
$Re = \rho u R_0/\mu=2.1$. A Reynolds number of order unity, suggests
that inertia is on the order
of viscous forces, and hence cannot be neglected.\\

\noindent\textit{Heat transfer and thermal effects:} The Prandtl
number is the ratio of momentum diffusivity to thermal diffusivity and
is defined as $Pr = \nu/\alpha = 105.6$. The calculated $Pr$ is large
relative to $Re$ and suggests that the effects of thermal diffusivity
are dominated by momentum diffusivity. We infer that that minimal heat
transfer occurs over the course of the collapse and as such it is neglected in our model.\\
% In consideration of the role of heat transfer, we calculate the
% relevant number, which characteristic length scale of thermal
% boundary layer growth over the course of a bubble collapse and
% compare it with the characteristic length scale of collapse. The
% length scale of the collapse is simply the equilibrium radius
% $R_0$. The characteristic length scale of the thermal boundary layer
% is, $l_{thermal}=\sqrt{\alpha_w t_{collapse}} = 0.13 \mu$m, where
% $\alpha_w$ is the thermal diffusivity of water $1.43\times10^{-8}$
% m$^2$/s, and $t_{collapse} = 0.915\sqrt{\rho R_0^2 / p_{atm}} = 91$
% ns \citep{Brennen2003} for a spherical bubble, neglecting surface
% tension. This also assumes that the vapor pressure in the bubble
% $p_v$ is much smaller than the driving pressure, which is true for
% ultrasonically driven cavitation. As $l_{thermal}\ll R_0$, we infer
% that that minimal heat transfer occurs over the course of the
% collapse and as such it is neglected in our model.\\

\noindent\textit{Surface Tension:} The Weber number is the ratio of inertial to
surface tension forces in the flow and is defined as
$We = \rho u^2 R_0/S = 17.9$. The calculated $We$ is large relative to
$Re$, but not so much as to suggest that the effects of surface
tension at the bubble wall are negligible, even when the bubble is at its
equilibrium radius. Additionally, we note that the effects of surface
tension may have an even greater effect during collapse when the
bubble radius may decrease by an order of magnitude
or more. Hence surface tension is not neglected.\\

\noindent\textit{Elasticity:} The Cauchy number is a measure of the
ratio of elastic to inertial forces and is defined as
$Ca = \rho u^2 / G = 1 - 200$ for the range of elastic moduli
considered (i.e., $5$ kPa - $1$ MPa). Based on this the effects of
elasticity are not expected to be particularly important to the bubble
dynamics for the tissues of kPa order elasticity, though this is
expected to change for stiffer tissues. Additionally, we note that our
work considers a Voigt type viscoelastic model, which is important as
the microsecond timescales associated with cavitation can effect the
relative importance of viscous and elastic forces. Accordingly, elasticity is included in the cavitation bubble model.\\

\noindent\textit{Gravity:} The Froude number is the ratio
of inertial to gravitational forces, or more generally, any applicable
body forces within a flow, and is defined as
$Fr = u/\sqrt{g R_0}=10^4$, where is the gravitational constant
$g=9.81$ m/s$^2$. The calculated Froude number is much larger than
$Re$ and suggests that gravitational and buoyancy effects are minimal
relative to inertia and is neglected for the sake of this
analysis. This is of particular importance because, for a homogeneous
medium, it allows us to consider the case of a spherically symmetric collapse, which greatly simplifies the problem.\\

In summary, based on the dimensional analysis performed, we consider
spherically symmetric bubble dynamics in a Voigt-Viscoelastic medium
with surface tension. The effects of gravity and heat transfer are
neglected.

\subsection{Dimensional analysis and assumptions for an acoustically
  driven alveolus}%
\label{sec:lung_assumptions}%
In this section, we focus on the problem of an ultrasound pulse or
physically similar acoustic wave impinging upon an alveolus within an
adult human lung. To assess the relevant physical mechanisms here in
order to layout the logic for our assumptions and approach, we present
a general case relevant to the motivating problem of lung
ultrasound. A more comprehensive justification and analysis, specific
to the work presented can be found in chapters \ref{ch:usbe_lung} and
\ref{ch:usbe_lung_bio}. Consider an ultrasound pulse with central
frequency $f = 3$ MHz, and amplitude $p_a = 1$ MPa, which are within
the expected parameter range based on past research
\citep{Miller2015a}. We use the mean diameter of a typical adult human
alveolus as a characteristic length scale length scale
$\ell_A = 200 \mu$m \citep{Ochs2004}. The alveolus is treated as air
at 37$^\circ$ such that the sound speed is $c_A = 353$ m/s, the
density is $\rho_A = 1.14$ kg/m$^3$, the kinematic viscosity is
$\nu_A = 16.6\times 10^{-6}$ m$^2$/s, and no elasticity is present in
the alveolar interior. The surrounding soft-tissue is treated as
water-like, but with elasticity such that the sound speed is
$c_T=1500$ m/s, the density is $\rho_T=1000$ kg/m$^3$, the kinematic
viscosity is $\nu_T = 0.7 \times 10^{-6}$ m$^2$/s and the elastic
modulus of the alveolar wall is $G = 5$ kPa \citep{Cavalcante2005}. We
use a characteristic velocity $u_T = \sqrt{p_a/\rho_T}=31.6$ m/s, a
convenient scale based on the wave and tissue properties. Based on the
physical problem described here we use dimensional analysis to access
the relative importance of potentially relevant physical mechanisms:\\

\noindent\textit{Viscosity:} In consideration of effects of viscosity, we calculate the Reynold's Number $Re = u_T \ell / \nu_T = 9035$, which suggests that the inertial forces dominate the viscous forces. Additionally, in consideration of late time effects, we calculate
the approximate viscous boundary layer thickness at $t_{final}$, the
time at the end of the simulated period such that
$l_{viscous}=\sqrt{\nu_{water} t_{final}} \approx
20~\mu$m. Furthermore, typical acoustic viscous boundary layer
thicknesses for MHz frequency ultrasound are $\lesssim1~\mu$m. As
$l_{viscous}/\ell\ll1$, the viscous boundary layer is expected to
remain far less than either a typical alveolar diameter or the length
scales associated with relevant geometrical structures of the
perturbed interface at the end of our baseline simulations (described
in Chapters \ref{ch:usbe_lung} and \ref{ch:usbe_lung_bio}). Hence we
exclude the effects of viscosity in our calculations.\\

\noindent\textit{Heat transfer and thermal effects:} We use similar arguments to
those used for viscous effects in consideration of thermal
effects. The thermal length scale is defined as
$l_{thermal}=\sqrt{\kappa/\pi f \rho C_p}$, where the $C_p$ is the
specific heat and $\kappa$ is the thermal conductivity. In air
$C_{pA}=1005$ J/Kg K and $\kappa_A=0.027$ W/m K and in tissue (based
on water) $C_{pT}=1005$ J/Kg K and $\kappa_T=0.49$ W/m K. Hence
$l_{thermal,A} = 0.3 \mu$m and $l_{thermal,T} = 1.5~\mu$m. For
both fluids, on either side of the interface, $l_{thermal}/\ell \ll 1$
such that the thermal boundary layer is small relative to the
characteristic length of the flow. Hence we neglect heat transfer in
our approach to this problem moving forward.\\

\noindent\textit{Surface Tension:} The role of surface tension in the alveoli
is critical to healthy respiratory function. Alveoli secrete pulmonary
surfactant, which lowers the surface tension at the alveolar surface,
helping prevent airway collapse and easing the re-inflation of alveoli
during breathing. As a result of this surfactant, alveolar surface
tension is far below that of water and has been reported as $S_A = 9$
mN/m \citep{Schurch1976}. Hence we define our Weber Number as
$We = \rho_T u_T^2 \ell/S_A = 22222$. This suggests that forces due to
surface tension are small relative to the acoustic pressure at the
interface. Based on this, we neglect surface tension in our
analysis as well.\\

\noindent\textit{Elasticity:} To assess the expected importance of elasticity to
the system we define a Cauchy number $Ca = \rho_T u_a^2/G$ which
becomes the ratio of the acoustic pressure to the elasticity
$ Ca = p_a/G = 200$. This suggests that elastic effects are dominated
by the acoustic pressure during the wave-interface interaction within
the tissue. Within the alveolar air space, there is no elasticity and
the Cauchy number is infinite. Based on this, we neglect elasticity in
our model. Additional calculations considering the relevance of this
assumption at later times, after the passage of
the wave are provided in Appendix \ref{sec:elasticity_appendix}.\\


\noindent\textit{Gravity:} The importance of gravity is assessed based on a
Froude number calculation $Fr= u_T / \sqrt{g \ell} = 714$. This
suggests that gravitational forces are small relative to inertia, and
can be neglected. Another reasonable justification for neglecting
gravity is that the orientation of the model problem in space is
arbitrary and as a 2D model we treat the flow as existing in a plane
that is orthogonal to gravitational forces and thus not affected by
gravity.\\

In summary, based on the dimensional analysis performed, we consider
an acoustically-driven, perturbed, water-air interface. The effects of
viscosity, elasticity, surface tension, gravity, and heat transfer are
neglected.

\subsection{Limitations} 
Before proceeding we would like to acknowledge that the
simplifications and assumptions made in the previous sections, while
justified in the specified regimes, do deviate from the true physical
systems in many situations. The purpose of these simplifications is to
make the relevant problems tractable with the available resources
(computational, intellectual, financial, temporal, etc...). In both
\ac{CEUS} and in ultrasound-alveoli interactions, the presented
dimensional analysis is based on tissue properties such as viscosity
and elasticity and behavior that are poorly characterized in both
nature and quantity. Additionally, the analysis performed here is for
reference cases within the relevant range, and certain dependencies,
such as the frequency dependence of sound speed in bulk lung tissue,
are not captured here. Furthermore, actual tissues are often
characterized by a wide range of physical length scales ranging from
submicron to meter and are heterogeneous at both micro and
macroscopic scales. Despite these limitations, the models developed
remain valid in the specified regimes by presented dimensional
analysis. As such, for cavitation in locally homogeneous Voigt
viscoelastic soft tissues with kPa order elasticity and otherwise
water-like physical properties (e.g., viscosity, surface tension,
density, sound speed) we expect the computed dynamics to be
representative of what one would expect in the real world. Similarly
for gas-liquid and gas-tissue interfaces with kPa order elasticity and
otherwise water-like physical properties, driven by ultrasound waves
with diagnostically relevant parameters of the order of those stated,
we expect that over the timescales of interest ($\sim100 \mu$s) the
dynamics are appropriately represented. The purpose of this work is to
gain insight into the approximate physics applicable to these problems
only within the valid regimes for which the models were designed.
 
\section{Thesis overview}
This part of the thesis presents work studying the physics of two
problems relevant to ultrasound bioeffects: 1) Cavitation of
ultrasound contrast agents microbubbles in human tissue, and 2)
\ac{DUS}-induced hemorrhage of the lung. For each problem, the
objectives of this thesis are to
\begin{enumerate}
\item Develop a computational model to simulate the problem.
\item Perform simulations to gain new insights into the relevant
  physics of \ac{CEUS} and diagnostic lung ultrasound.
\item Use the results of the simulations to develop and test new
  hypothesis and when possible make conclusions about the physics and
  its relevance to the \ac{US} bioeffects.
\end{enumerate}

Upon pursuit of the above objectives, the novel contributions made
throughout this thesis are summarized as follows:
\begin{itemize}
\item A novel approach to investigating \ac{CEUS} related bioeffects
  is taken, combining simulations of bubble dynamics in a soft
  tissue-like viscoelastic medium, with experimentally measured
  ultrasound waveforms and known bioeffects thresholds.
\item Calculated cavitation dynamics are related to known bioeffects
  thresholds associated with the experimentally measured \ac{US}
  inputs.
\item Further evidence is offered to suggest that existing \ac{IC}
  thresholds may not be well suited for viscoelastic media.
\item A novel model of an ultrasound-driven alveolus is developed.
\item It is demonstrated that acoustic waves interacting with
  perturbed liquid-gas interfaces may generate sufficient baroclinic
  vorticity to drive appreciable interface deformation. These
  deformations are a result of nonlinear processes and cannot be
  described by purely linear acoustics.
\item Vorticity-driven perturbation growth of gas-liquid interfaces
  perturbations is shown to exhibit power-law behavior and scale with
  interfacial circulation density.
\item It is shown that because the morphology of the interface is
  changing during the wave-interface interaction, vorticity deposition
  depends on the transient form of the wave, and consequently so does
  the long-term growth (or lack thereof) of the interface perturbation.
\item Approximate stresses and strains associated ultrasound-driven
  gas-liquid interfaces are calculated. These quantities are compared
  with established alveolar failure criteria to conclude that, while
  ultrasound-induced viscous stress in the lung is unlikely to cause
  hemorrhage, there is a potential for vorticity-driven alveolar wall
  strain that may be related to hemorrhage.
\end{itemize}
A more detailed chapter-by-chapter summary of the work and conclusions
of this thesis is provided below.

In Chapter \ref{ch:usbe_bubble}, we simulate the cavitation bubble
dynamics of contrast agent microbubbles in soft tissue
\citep{Patterson2012}. Experimentally measured \ac{US} waves with
known bioeffects occurrence and thresholds are used
\citep{Miller2008b}.  A parametric study is performed, relating
ultrasound and tissue parameters to calculated cavitation bubble
dynamics. The soft tissue is modeled as a Voigt viscoelastic medium
based on the work of \cite{Yang2005}. The calculated cavitation
dynamics and theoretical inertial cavitation thresholds
\citep{Flynn1982,Apfel1982} are compared with bioeffects thresholds
associated with each \ac{US} pulse, as defined by the observation of
kidney hemorrhage in rats exposed to CEUS by \cite{Miller2008b}. While
the results were generally dependent on \ac{US}, gas, and tissue
properties, it was found that the theoretical inertial cavitation
thresholds were lower than observed bioeffects thresholds. It is shown
that these thresholds correlate strongly to calculated metrics of
cavitation, such as dimensionless maximum radius
$R_{max}/R_{equilibrium}$ and that this correlation is lost when
simply looking at the dimensional maximum bubble size $R_{max}$, which
is not a cavitation metric.

In Chapter \ref{ch:usbe_lung}, we develop a model of an
ultrasonically-driven alveolus as a compressible, multi-phase fluid
system. This model is used to study the fundamental problem of an
acoustically-driven perturbed liquid-air interface. We demonstrate
that under the assumptions presented in Section
\ref{sec:lung_assumptions}, trapezoidal acoustic waves of sufficient
pressure amplitude are capable of generating enough baroclinic
vorticity to appreciably deform the interface. The dependence of this
deformation on the amplitude and temporal characteristics of the wave
is studied. It is demonstrated that the deformation rate scales with
the amount of circulation per unit length of the interface. It is
also shown that the amount of circulation deposited by the wave is
heavily dependent on the deformation that occurs during the
wave-interface interaction, and therefore depends on the transient
properties of the wave.

In Chapter \ref{ch:usbe_lung_bio}, the work of the previous chapter is
extended to increase its relevance to clinical \ac{DUS}. The
previously developed model of an ultrasound-driven alveolus is
modified and used to simulate a perturbed liquid-gas driven by an
ultrasound pulse with diagnostically relevant parameters. We calculate
approximate stresses and strains at the interface and compare to
accepted alveolar failure criteria. It is shown that viscous stresses
are small compared to expected failure thresholds. However, it is also
shown that strains at gas-liquid interfaces such as those of the
lungs, driven by acoustically-generated vorticity, may be sufficient
to drive hemorrhage for sufficiently strong ultrasound pulses. This
work concludes that while vorticity may be a possible mechanism for
driving \ac{DUS}-introduced lung hemorrhage, additional work needs to
be completed to account for multiple pulses as well as physical
effects of elasticity and viscosity in order fully understand the role
of vorticity in this problem.

In the final chapter \ref{ch:usbe_conclusions} of Part I of this dissertation, we summarize
the main conclusions takeaways and accomplishments of this work. I
also make recommendations for future work to overcome the limitations
of the presented research and extend this work to address relevant
problems within the field.

% If the conservation  equations were kept separate
\begin{comment}
  \begin{align} \label{eq:intro_coma}
    \frac{\partial \rho}{\partial t} + \nabla\cdot\left(\rho\bs{u}\right) = 0,%
  \end{align}
  \begin{align} \label{eq:intro_como}% 
    \rho\frac{D \bs{u}}{D t} = \nabla\cdot\bs{\tau}+\bs{g},%
  \end{align}%
  \begin{align} \label{eq:intro_coe}%
    \frac{\partial}{\partial t}\left(\rho \left[e + \frac{\bs{u}\cdot\bs{u}}{2}\right]\right) + \nabla\cdot\left(\rho \left[e + \frac{\bs{u}\cdot\bs{u}}{2}\right]\bs{u}\right) = \rho\left(\bs{g}\cdot\bs{u}\right) + \nabla\cdot\left(\bs{\tau u}\right) + \nabla\cdot\bs{q}%
  \end{align}%
  \begin{align} \label{eq:stiffened_eos_intro}%
    E=\frac{\rho\left(u^2+v^2\right)}{2} + \frac{p+\gamma B}{\gamma-1}.
  \end{align}
\end{comment}


%%% Local Variables:
%%% mode: latex
%%% TeX-master: "../../main"
%%% End:
