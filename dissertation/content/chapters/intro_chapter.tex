The purpose of this introduction is to set the stage for the proposed
dissertation research. The problems we approach in this work are all
problems of interest, current to the field of Acoustics. Broadly,
acoustics is the study of sound. In practice, this study is not
limited to just the kinds of sound that can be heard by humans, but
rather any molecular scale vibrations traveling throughout a media. As
sounds both natural and man-made are ubiquitous, it is a topic that
has intrigued man for quite some time and attracted much attention and
study. As such, we have gained not only an understand the physical
nature of sound, but have also learned to harness it as a
tool. Because sound waves travel reflect, transmit, and scatter in a
mathematically describable way, they are ideally suited for gathering
information in certain situations. Because they carry mechanical
energy that can be focused, concentrated, and in some instances
converted into other types of energy, such as heat, they can also be a
powerful tool for physically altering an environment. In some
applications of interest, attempts to use acoustics to gather
information, can unintentionally lead to physical modification of the
system, such is the case when \ac{DUS} for medical imaging leads to
unintended biological effects, or ultrasound bioeffects as we will
refer to them from here on out.

Many problems of contemporary acoustic interest present challenges
that make them difficult to investigate completely through direct
experimentation. Some problems, such as certain ultrasound bioeffects,
often involve physical processes that occur over such small length and
time scales that they cannot be directly observed. When these
phenomena are replicated in simplified lab experiments, as they
frequently are, physical quantities of interest, like stress, are not
always readily measurable. Other problems may call for experiments
that are prohibitively costly and time-intensive, as is often the case
in underwater and ocean acoustics experiments which can require long
cruises with extensive personnel and equipment. Furthermore, in
complex acoustic environments like the ocean or human body, we rarely
have sufficient information to precisely and accurately describe the
system of interest without a high degree of uncertainty. In instances
such as these, where direct experimentation is infeasible or unable to
provide the desired information, carefully designed numerical
experiments can be useful for providing insight into the
problem at hand. 

The unifying theme of the work presented here is the use of
computation to approach modern problems in acoustics. The two main
areas of research considered are \acf{US} bioeffects and underwater
acoustic uncertainty. In the first part of this work, we investigate
two problems related to biological effects of medical
\ac{US}. Specifically, we simulate physics associated with \ac{CEUS}
and \ac{DUS} of the lung, which have both been shown to be capable of
causing hemorrhage in mammals, in order to investigate the damage
mechanism behind each. In the second part of this work we develop and
test area statistics, a computationally efficient method for
estimating the \ac{PDF} of acoustic \ac{TL} in uncertain ocean
environments, which is useful in naval applications. As these areas
are appreciably different, we will refer the reader to later portions
of this document and to the authors relevant submitted and published
works for more detailed introduction and background on each problem.

% %\subsection{\ac{US} background}
% Diagnostic \ac{US} has proven to be among the safest and most powerful
% medical imaging tools currently available and its use has become
% ubiquitous throughout modern medicine. The basic physical principle
% underlying this technology the scattering of sound at material
% interfaces. In practice, high-frequency, typically MHz range, acoustic
% waves and pulses are created at the surface of the body using a
% piezoelectric \ac{US} transducer. These vibrations propagate via an
% acoustic coupling medium from the transducer into the tissue and
% scatter whenever they encounter a change in the material properties of
% the medium. More simply, some of the sound echoes whenever it moves
% from one tissue to another, or hits a cavity in the body. These echoes
% are then picked up by a receiver and recorded. The strength and timing
% of these echoes allow for real-time imaging of the scattering surface.

% While clinical \ac{US} is generally incredibly safe there are
% specific instances during which \ac{US} can interact with tissue in
% such a way that it physically alters tissue. These effects to the body
% are referred to as \ac{US} bioeffects. While the entire field of
% therapeutic \ac{US} is based around intentionally causing
% bioeffects in a way that is beneficial to the patient, diagnostic
% \ac{US} is a different story. Bioeffects that occur during
% diagnostic \ac{US} typically take the form of unintended tissue
% damage or cell death. Depending on the type of physical damage
% mechanism responsible, these bioeffects are classified into two
% groups, thermal and non-thermal. The first group, thermal bioeffects
% are characterized by deposition of acoustic energy into tissue as
% heat. At the cellular and molecular scales, this can lead to the
% release of highly reactive free radicals, protein denaturation, and
% ultimately tissue damage and death. Little else will be said about
% thermal bioeffects, as the the bioeffects problems of interest to this
% work fall into the non-thermal category. The bulk of known non-thermal
% bioeffects are attributed to acoustically-induced cavitation.  

% Acoustic cavitation is the phenomenon by which gas nano and
% microbubbles, called cavitation nuclei, are cyclically grown by low
% pressures within the \ac{US} field and then collapsed high
% pressures within the field. When the bubble dynamics during collapse
% are dominated by the inertia of the surrounding fluid, it is called
% \ac{IC}. \ac{IC} is typically violent and results in the
% bubble collapsing to a fraction of its original size. There are
% several possible damage mechanisms associated with \ac{IC} that may be
% responsible for observed \ac{US} bioeffects. Upon collapse, the
% pressure and temperature within the bubbles spike, often reaching
% billions of pascals and thousands of Kelvin respectively. Due to the
% pressure difference between the vapor/gas mixture within the bubble at
% collapse and the surrounding media, the collapsed bubble can emit a
% powerful shock wave which can be damaging to the bubbles
% surroundings. When cavitation is triggered near a rigid surface, the
% bubble can collapse in a radially asymmetric fashion causing a high
% speed ``re-entrant'' jet of liquid to impinge upon the surface,
% effectively striking the surface with a liquid hammer \hl{CITE}. If
% cavitation occurs at an appropriate distance from a non-rigid surface,
% such as soft tissue boundaries and blood vessel walls, the jet can
% impinge away from the surface, potentially invaginating the surface
% \citep{Brujan2011}. \ac{CEUS}, which uses
% contrast-agent microbubbles injected into patients bloodstream to act
% as additional scattering surfaces. These microbubbles act as
% cavitation nuclei and have been associated with a variety of different
% forms of cellular death and damage.

% This proposal presents past work in which we simulate ultrasonically
% induced cavitation of contrast agent microbubbles in soft tissue
% \citep{Patterson2012} (See Chapter \ref{ch:usbe_bubble}).  We simulate
% experimentally measured \ac{US} waves obtained by \cite{Miller2008b}
% perturbing microbubbles in a Voigt viscoelastic soft tissue
% \cite[]{Yang2005}. The calculated cavitation dynamics and theoretical
% inertial cavitation thresholds \citep{Flynn1982,Apfel1982} are
% compared with bioeffects thresholds associated with each \ac{US}
% pulse, as defined by the observation of kidney hemorrhage in rats
% after exposure to CEUS by \cite{Miller2008b}. While the results were
% generally dependent on US, gas, and tissue properties, it was found
% that the inertial cavitation thresholds were lower than observed
% bioeffects thresholds.

% Another non-thermal \ac{US} bioeffect of interest is \ac{DUS}-induced
% \ac{LH}, which is the only known bioeffect of non-contrast \ac{DUS}
% known to occur in mammals. Despite the fact that this phenomenon was
% first observed in mice over twenty years ago \citep{Child1990}, the
% underlying physical damage mechanisms remain unknown. Research has
% shown that thermal damage mechanisms are unlikely as \ac{DUS}-induced
% lung lesions do not appear similar to those induced by heat
% \citep{Zachary2006}. Furthermore, cavitation mechanisms do not appear
% to be responsible, as the severity of \ac{DUS}-induced \ac{LH} in mice
% increased under raised hydrostatic pressure \citep{OBrien2000} and was
% unaffected by the introduction of \ac{US} contrast agents into
% subjects. Both of these results are inconsistent with what is expected
% of \ac{IC}-induced bioeffects. Works by \cite{Tjan2007,Tjan2008} model
% the evolution of an inviscid, free surface subjected to a Gaussian
% velocity potential and find that this can lead to the ejection of
% liquid droplets. They go on to say that \ac{DUS} of the lung may
% similarly lead to the ejected of droplets capable of puncturing the
% air-filled sacs within the lung. We propose another possible damage
% mechanism, that \ac{DUS} pulses torque tissue-air interfaces around
% alveoli, fragile air-sacs within the lungs. This deposits vorticity, a
% measure of local fluid rotation, in the surrounding fluid which drives
% deformation and ultimately hemorrhage of the thin alveolar walls.

% The concept of vorticity driven interface deformation has been
% extensively studied within the context of the \ac{RM}, which occurs
% when a traveling pressure wave, typically a shock, encounters a
% perturbed interface between fluids of different density. Note that the
% \ac{RM} is not a true instability, as the interface does not exhibit
% exponential growth. When this occurs vorticity is generated along the
% interface.

% As can be seen from the
% vorticity generation equation,
% \begin{align} \label{eq:vorticity}
% \frac{\partial \vec{\omega}}{\partial t}+\left(\vec{u}\cdot\nabla\right)\vec{\omega} = 
% \left(\vec{\omega\cdot\nabla}\right)\vec{u} - \vec{\omega}\left(\nabla\cdot\vec{u}\right)%
% +\frac{\nabla\rho\times\nabla p}{\rho^2} + \nabla\times\left(\frac{\nabla\cdot\tau}{\rho}\right)%
% +\nabla\times\left(\frac{\vec{B}}{\rho}\right)%
% \end{align}


% Computations of acoustic transmission loss are of practical interest in a variety of naval applications.






%%% Local Variables:
%%% mode: latex
%%% TeX-master: "../../main"
%%% End:
