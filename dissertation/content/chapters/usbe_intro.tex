\chapter{Introduction} \label{ch:usbe_intro}%
The purpose of this chapter is two fold. First, we aim to provide a
general physical context for the work presented in this part of the
thesis proposal. Second, we will provide a brief overview of the work
to be presented and its significance. For a more detailed overview of
the relevant literature, the reader is referred to later parts of this
document, and to this authors published works.

\section{Physical context} \label{sec:usbe_intro_physical_context}%
Diagnostic \ac{US} has proven to be among the safest and most powerful
medical imaging tools currently available. Its use has become
ubiquitous throughout modern medicine. The basic physical principle
underlying this technology is the scattering of sound at material
interfaces. In practice, high-frequency, typically MHz range, acoustic
waves and pulses are created at the surface of the body using a
piezoelectric \ac{US} transducer. These vibrations propagate via an
acoustic coupling medium from the transducer into the tissue and
scatter at changes in the material properties of the medium. More
simply, some of the sound echoes whenever it moves from one tissue to
another, or hits a cavity in the body. These echoes are then picked up
by a receiver and recorded. This echo signal is processed to obtain
real-time images of the scattering surface.

While clinical \ac{US} is typically safe there are specific instances
during which \ac{US} can interact with tissue in such a way that the
tissue is physically altered. These effects to the body are referred
to as \ac{US} bioeffects. Understanding these \ac{US}-tissue
interactions is important for the development of safe, effective
\ac{US} techniques \citep{Dalecki2004}. While the entire field of
therapeutic \ac{US} is focused on intentionally causing bioeffects in
a way that is beneficial to the patient, diagnostic \ac{US} is a
different story. Bioeffects that occur during diagnostic \ac{US}
typically take the form of unintended hemorrhage, tissue damage, or
cell death. Depending on the physical damage mechanism responsible,
these bioeffects are broadly classified into two groups, thermal and
non-thermal \citep{OBrien2007}. The first group, thermal bioeffects
are characterized by deposition of acoustic energy into tissue as
heat. At the cellular and molecular scales, this can lead to the
release of highly reactive free radicals, protein denaturation, and
ultimately tissue damage and death. Little else will be said about
thermal bioeffects, as the bioeffects problems of interest to this
work are a result of non-thermal mechanisms. 

The bulk of known non-thermal bioeffects are attributed to
acoustically-induced cavitation. Acoustic cavitation is the phenomenon
by which gas nano and microbubbles, called cavitation nuclei, are
cyclically grown by low pressures within the \ac{US} field and then
collapsed high pressures within the field. Cavitation can be divided
into two categories, stable cavitation, also called gas body
activation, and \ac{IC}, formerly referred to as transient
cavitation. Stable cavitation typically occurs for low \ac{US}
intensity an is characterized by bubbles periodically oscillating
around an equilibrium radius for multiple acoustic cycles. \ac{IC}
typically occurs for higher ultrasound intensities. During \ac{IC} the
bubble dynamics during collapse are dominated by the inertia of the
surrounding fluid. The bubble collapses violently to a tiny fraction
of its original size and then explosively rebounds back. There are
variety of physical phenomena associated with \ac{IC} that may be
responsible for observed \ac{US} bioeffects. Upon collapse, the
pressure and temperature within the bubbles spike, often reaching
billions of pascals and thousands of Kelvin respectively. Due to the
pressure difference between the vapor/gas mixture within the bubble at
collapse and the surrounding media, the collapsed bubble can emit a
powerful shock wave. When cavitation is triggered near a rigid
surface, the bubble can collapse in a radially asymmetric fashion
causing a high speed ``re-entrant'' jet of liquid to impinge upon the
surface, effectively striking the surface with a liquid hammer. If
cavitation occurs at an appropriate distance from a non-rigid surface,
such as soft tissue boundaries and blood vessel walls, the jet can
impinge away from the surface, potentially invaginating the surface
\citep{Brujan2011}. One type of \ac{DUS} for which cavitation is of
particular concern is \ac{CEUS}, which uses contrast-agent
microbubbles injected into patients bloodstream to act as additional
scattering surfaces. These microbubbles can also serve as cavitation
nuclei and have been associated with a variety of \ac{US} bioeffects. 

Another non-thermal \ac{US} bioeffect of interest is \ac{DUS}-induced
\ac{LH}, which is the only known bioeffect of non-contrast \ac{DUS}
known to occur in mammals. Despite the fact that this phenomenon was
first observed in mice over twenty years ago \citep{Child1990}, the
underlying physical damage mechanisms remain unknown. Research has
shown that thermal damage mechanisms are unlikely as \ac{DUS}-induced
lung lesions do not appear similar to those induced by heat
\citep{Zachary2006}. Furthermore, cavitation mechanisms do not appear
to be responsible, as the severity of \ac{DUS}-induced \ac{LH} in mice
increased under raised hydrostatic pressure \citep{OBrien2000} and was
unaffected by the introduction of \ac{US} contrast agents into
subjects. Both of these results are inconsistent with what is expected
of \ac{IC}-induced bioeffects. Works by \cite{Tjan2007,Tjan2008} model
the evolution of an inviscid, free surface subjected to a Gaussian
velocity potential and find that this can lead to the ejection of
liquid droplets. They go on to say that \ac{DUS} of the lung may
similarly lead to the ejected of droplets capable of puncturing the
air-filled sacs within the lung. This problem is central to the our
present and future work, and makes up the bulk of this proposal. As
such, a far more in-depth literature review will be provided in
Chapter \ref{ch:usbe_lung}.

\section{An overview of our work studying \ac{US}
  bioeffects} \label{sec:usbe_intro_overview}%
For the proposed dissertation, we will discuss our work studying two
ultrasound bioeffects problems. 

First, in Chapter \ref{ch:usbe_bubble} we present past work in which
we simulate ultrasonically induced cavitation of contrast agent
microbubbles in soft tissue \citep{Patterson2012}. We use
experimentally measured $1.5-7.5$ MHz \ac{US} waves, previously used
by \cite{Miller2008b} to determine kidney capillary hemorrhage
threshold amplitudes in rats, as input to the simulation. The
calculated cavitation dynamics and theoretical inertial cavitation
thresholds \citep{Flynn1982,Apfel1982} are compared with known
thresholds for kidney hemorrhage to investigate their dependence on
US, gas, and tissue properties. At the time of its publication, this
work was unique in its combination of experimental results and
numerical modeling to approach this problem.

Second, in Chapter \ref{ch:usbe_lung} we present current work
investigating a previously unconsidered potential mechanism for
\ac{DUS}-induced \ac{LH}. We develop a model of \ac{DUS}-alveolus
interaction as an acoustically accelerated interface between two
compressible fluids and perform numerical simulations to show that
acoustically generated baroclinic torque at tissue-air interfaces
within the lungs may be capable of deforming the fragile alveolar
walls within the lungs, possibly to the point of hemorrhage. We
generalize our discussion to acoustically-accelerated, perturbed,
liquid-gas interfaces. Finally we propose future work to be completed
for this dissertation.
%%% Local Variables:
%%% mode: latex
%%% TeX-master: "../../prelim"
%%% End:
