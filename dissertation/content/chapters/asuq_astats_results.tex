\section{Results and comparisons} \label{section:asuq_astats_results}
For an overall accuracy assessment of the area statistics method, the
error from (1) was computed on a coarse rectangular grid in each of
the four uncertain ocean environments. The results are shown in Figure
5 as a grid of test locations overlaid on the baseline TL field for a
200 Hz source in each environment. The number and position of test
locations were chosen based on the physical geometry and computational
grid spacing in each environment without consideration for the results
at the test locations. The number of test locations was increased in
environment 4 due its larger physical size. Test points below the
ocean floor were not considered. In each panel of Fig. 5, a white
circle indicates a test location where the area-statistics estimated
PDF of TL was found to be engineering-level accurate ( ≤ 0.50), while
a black triangle indicates a test location where engineering-level
accuracy of the area-statistics estimated PDF of TL was not achieved (
> 0.50).

In the three shallower environments, area statistics produced PDFs of
TL with errors less than or equal to 0.50 at 88, 95, and 85\% or more
of test locations for 91-m, 137-m, and 183-m deep sources at
frequencies of 100, 200, and 300 Hz respectively. In the deepest
environment where refraction of sound plays a larger role, area
statistics was less successful. A quantitative summary of these
results is provided in Table 2 which also includes additional results
for environment 4 when the area-statistics algorithm described above
is adjusted (marked with a †).

To improve the area statistics results for deeper refractive
environments, a closer look was taken at the TL fields and PDFs of TL
calculated for environment 4. The most common failure in environment 4
typically occurred at test locations below the critical depth at which
the sound speed equals the sound speed at the ocean surface (herein
referred to as deep water test locations). For this failure, the
shapes of the area-statistics and Monte-Carlo generated PDFs of TL
were similar but the mean TL values were sufficiently different such
that the errors were high. Furthermore, in these failures of area
statistics, the baseline TL value at the center of the sample
rectangle was noticeably different from the mean and median TL values
of the sample area, indicating a highly non-uniform distribution of TL
values around the sample rectangle center point. To partially correct
the area statistics PDFs of TL for failures of this type, half the
difference between the baseline and median TL values from the AS
sample area was added to every TL value in the sample area to
appropriately shift the area-statistics PDF of TL. This algorithm
adjustment improved the percentage of engineering accurate tests
locations in environment 4 from 64\% to 73\% for a 137m deep, 200 Hz
source. And, this algorithm adjustment did not affect the success
percentages of area statistics in environments 1, 2, or 3 at any of
the frequencies considered here (100, 200, and 300 Hz).

The computational effort associated with area statistics was also
compared to that associated with the Monte Carlo calculations using
the MATLAB profiler. As might be expected, given its simple
formulation, area statistics is significantly more efficient than
Monte Carlo calculations. With the baseline TL calculation as a
starting point for both approaches, area statistics does not require
another TL field calculation while the Monte-Carlo approach – as
implemented here – involves 1000 more. Thus, the difference in
computational burden is substantial. For a single location, the
Monte-Carlo calculations (using 10 Padé terms in RAM's PE solver)
required 4.6 million times more computational effort than area
statistics on average. For 100 field locations in a single range-depth
plane of an uncertain ocean, the Monte-Carlo calculations required an
average of 46 thousand times more computational effort than area
statistics. Additionally, once the baseline TL calculation is
complete, area statistics can provide PDFs of TL in milliseconds of
real time, making it practical for real-time applications of TL
uncertainty.

%%% Local Variables:
%%% mode: latex
%%% TeX-master: "../../prelim"
%%% End:
