\section{Analysis}%
\label{sec:usbe_lung_analysis}%
We perform analysis to make quantifiable predictions about the
vorticity and interface dynamics. The results of these
analyses are compared with the results of our numerical experiments in
section \ref{sec:usbe_lung_results}.

To better understand the source of circulation within our problem we
look to the vorticity generation equation for a 2D, inviscid fluid
system,
\begin{align} \label{eq:vorticity_euler}
  \frac{\partial \vec{\omega}}{\partial t}+\left(\vec{u}\cdot\nabla\right)\vec{\omega} =% 
  - \vec{\omega}\left(\nabla\cdot\vec{u}\right) + \frac{\nabla\rho\times\nabla p}{\rho^2}.%
\end{align}
Each term in equation \eqref{eq:vorticity_euler} represents a
different physical mechanism by which the vorticity $\vec{\omega}$ is
changing. The terms on the left-hand side of the equation represent
changes in the existing vorticity field and the terms on the right
represent vorticity sources and sinks. The first term on the left
represents the total change of vorticity at a location in the flow
field with respect to time. The second term on the left represents the
advection of vorticity within the field. The first term on the right
describes changes in vorticity due to compressibility. The last term
on the right is the baroclinic term which represents vorticity
generated by the misalignment of the pressure and density gradients in
the flow. We seek to understand the relative importance of these
mechanisms on the dynamics of the acoustically driven interface.

\subsection{Order of magnitude analysis of vorticity generation mechanisms}
To quantifiably compare the various mechanisms by which vorticity
changes within the flow, we recognize that any vorticity generated
must be a result of acoustic energy being converted to kinetic
energy. As the only mechanism for this to occur in an inviscid fluid
without pre-existing vorticity is baroclinic, we require misaligned
density and pressure gradients. Hence we choose to perform our
analysis at the water-air interface during the period in which the
interface is interacting with the incoming wave. For simplicity, we
narrow this down to only consider the period in which the incoming
compressive portion of the wave encounters the interface. As this
interaction occurs quickly, over an approximate time span
$\Delta t_a\approx5\lambda/c_{w}$, we assume that the interface is
static and remains undeformed from its initial state during this
interaction. We will show in section \ref{sec:usbe_lung_results} that
this assumption is reasonable for the cases considered. Having now
established the point at which the analysis is to be performed we
evaluate the order of magnitude of the compressible, advective, and
baroclinic terms of the vorticity generation equation
\eqref{eq:vorticity_euler}. We note that the advective term is not a
true source of vorticity, but is useful in understanding the change of
vorticity at a given time and location within the flow.

In our evaluation of the order of magnitude of the individual terms of
the vorticity generation equation \eqref{eq:vorticity_euler}, we treat
gradient, curl, and divergence terms of any arbitrary quantity $f$
such that $\nabla f= \orderof{\left|\Delta f\right|/\Delta L}$,
$\nabla\cdot f=\orderof{\left|\Delta f\right|/\Delta L}$, and
$\nabla\times f=\orderof{\left|\Delta f\right|/\Delta L}$. Here
$\Delta f$ is a change in $f$ over a characteristic length scale
$\Delta L$. Because the only motion in the flow is generated by the
acoustic wave. Accordingly, we consider acoustic pressure, velocity,
and density perturbations such that $\Delta p=\Delta p_a$,
$\Delta \vec{u}=\Delta \vec{u}_a$, and $\Delta \rho=\Delta \rho_a$,
and use acoustic relations to relate these quantities
\citep{Anderson1990},
\begin{align}%
  \label{eq:acoustic_relations}%
  \Delta p_a=\pm\Delta u_a \rho c=c^2\Delta \rho_a.%
\end{align}
Additionally, to evaluate the expressions in this section we use the
values in in tables \ref{tab:usbe_lung_dimensional_parameters} and
\ref{tab:usbe_lung_dimensionless_parameters} consider our base
trapezoidal wave case where $p_a = \Delta p_a = 10$ MPa. The length
scale associated with the acoustic wave is the initial length of the
pressure rise $\Delta L_a=5\lambda$. The initial interface length
scale $\Delta L_I$, defined as the thickness of the mixed gas-liquid
region of the interface from volume fraction of water $y_0=0.05$ to
$0.95$ volume fraction is estimated as
$\Delta L_I \approx 0.05\lambda$. We approximate the order of theta
based on its average value along a half-wavelength of the interface
for our initial condition $a_0=0.03\lambda$ such that
$\overline{\abs{\theta}}\approx0.12$.

To assess the baroclinic contribution to vorticity, we write the cross
product of the density and pressure gradients as
$\abs{\nabla \rho} \abs{\nabla p} \sin{\left(\theta\right)}$. Here
$\theta$ is the angle between the acoustic pressure gradient, treated
as constantly in the $\plus y$-direction, and the direction of the
density gradient which we treat as the outward normal direction to the
interface. For $a_0/\lambda<<1$, we can approximate
$\sin{\left(\theta\right)}\approx\theta$ at the interface. The density
gradient due to the water-air interface is far greater than that due
to the acoustic wave. As such we use the change in density across the
interface $\Delta \rho_I$ and associated length scale $\Delta L_I$ to
write the density gradient. The pressure change is a result of the
acoustic wave, and as such we use the acoustic pressure change
$\Delta p_a$ and associated length scale $\Delta L_a$ to express the
pressure gradient. And thus we write the order of magnitude of the
baroclinic vorticity generation term at the interface,
\begin{align}
  \label{eq:baroclinic_vorticity}%
  \norm{\frac{\nabla\rho\times\nabla p}{\rho^2}} = \orderof{\frac{\abs{\Delta \rho_I}}{\abs{\Delta L_I}}\frac{\abs{\Delta p_a}}{\abs{\Delta L_a}}\frac{1}{\abs{\rho}^2}\abs{\theta}}.%
\end{align}

In the evaluation of the compressible and advective terms we consider
two possible cases for the evaluation of the vorticity, based on
whether the dominant vorticity arises from the acoustic flow-field or
is baroclinically generated. Thus we will ultimately treat
$\vec{\omega}$ as either the curl of the acoustic velocity field
$\vec{\omega}=\nabla\times\vec{u}$ or the integral of the baroclinic
vorticity generation term \eqref{eq:baroclinic_vorticity}, treated as
constant, over the characteristic time of the pressure rise
$\Delta t_a\approx\Delta L_a/c_w$.

We first consider the case in which vorticity is predominately a
product of the acoustic velocity field such that the approximate order
of magnitude of the compressible contribution to vorticity generation
is expressed as
\begin{align}
  \label{eq:compressible_vorticity}%
  \norm{-\vec{\omega}\left(\nabla\cdot\vec{u}\right)} = \orderof{\left[\frac{\abs{\Delta u_a}}{\abs{\Delta L_a}}\right]^2},%
\end{align}
and for the advective contribution we find
\begin{align}
  \label{eq:advective_vorticity}%
  \norm{\left(\vec{u}\cdot\nabla\right)\vec{\omega}} = \orderof{\left[\frac{\abs{\Delta u_a}}{\abs{\Delta L_a}}\right]^2}.%
\end{align}
We note that from this analysis, we expect the advective and
compressible vorticity effects to be of the same order during
considered period and will treat them as such for the remaining
analysis.

Now, to compare the relative importance of the baroclinic and
compressible (or advective) contributions to vorticity for this case
we will look at the ratio of the two vorticity generation
approximations. We divide equation \eqref{eq:baroclinic_vorticity} by
equation \eqref{eq:compressible_vorticity} use
\eqref{eq:acoustic_relations} to express acoustic quantities in terms
of the density perturbation $\Delta \rho_a$ and simplify,

\begin{align} \label{eq:vorticity_comparison_broke}
%\left(\frac{\partial\omega}{\partial t}\right)_{\text{baroclinic}} / \left(\frac{\partial \omega}{\partial t}\right)_{\substack{\text{compressible/}\\\text{advective}}} =&%
\frac{\norm{\frac{\nabla\rho\times\nabla p}{\rho^2}}}{\norm{-\vec{\omega}\left(\nabla\cdot\vec{u}\right)}} =&%
\orderof{\left(\frac{\abs{\Delta \rho_I}}{\abs{\Delta L_I}}\frac{\abs{\Delta p_a}}{\abs{\Delta L_a}}\frac{1}{\abs{\rho}^2}\abs{\theta}\right) /%
\left( \left[\frac{\abs{\Delta u_a}}{\abs{\Delta L_a}}\right]^2 \right)}\nonumber\\%
%
=&\orderof{\left[\frac{\abs{\Delta \rho_I}}{\abs{\Delta L_I}}\frac{\abs{\Delta \rho_a}}{\abs{\Delta L_a}}\frac{\abs{c}^2}{\abs{\rho}^2}\abs{\theta}\right] /
\left[\frac{\abs{c}}{\abs{\rho}}\frac{\abs{\Delta \rho_a}}{\abs{\Delta L_a}}\right]^2}\nonumber\\%
%
=&\orderof{\frac{\abs{\Delta \rho_I}/\abs{\Delta L_I}}{\abs{\Delta \rho_a}/\abs{\Delta L_a}} \abs{\theta} }.
%=&\orderof{\abs{\theta}\frac{\abs{\Delta \rho_I}}{\abs{\Delta L_I}} / \frac{\abs{\Delta \rho_a}}{\abs{\Delta L_a}}}.
\end{align}
Evaluating the right-hand side of
\eqref{eq:vorticity_comparison_broke} using the previously described
approximations and values we find that the ratio of baroclinic
vorticity and compressible contributions to vorticity to be of order
$\orderof{10^3}$.

As the previous result would suggest that baroclinic vorticity
generation is strongly dominant, we must check the result by again
evaluating the compressible and advective contributions to vorticity
generation, using an expression for vorticity based on the time
integration of \eqref{eq:baroclinic_vorticity},
\begin{align}
  \label{eq:compressible_advective_vorticity}%
\norm{-\vec{\omega}\left(\nabla\cdot\vec{u}\right)}\sim \norm{\left(\vec{u}\cdot\nabla\right)\vec{\omega}} = %
\orderof{\frac{\abs{\Delta u_a}}{\abs{\Delta L_a}} \frac{\abs{\Delta \rho_I}}{\abs{\Delta L_I}}\frac{\abs{\Delta p_a}}{\abs{\Delta L_a}}\frac{1}{\abs{\rho}^2}\abs{\theta}\frac{\abs{c}}{\abs{\Delta L_a}}}.%
\end{align}
%
Again, comparing the relative importance of the baroclinic and
compressible (or advective) contributions to vorticity as we did before,
\begin{align} \label{eq:vorticity_comparison}
\frac{\norm{\frac{\nabla\rho\times\nabla p}{\rho^2}}}{\norm{-\vec{\omega}\left(\nabla\cdot\vec{u}\right)}} = \frac{c}{\abs{\Delta u_a}} = \frac{\rho}{\abs{\Delta \rho_a}}%
\end{align}
Now evaluating this expression we expect that the relative contribution of
baroclinic to compressible/advective vorticity generation is
approximately of order $\orderof{10^2}$ at the end of the
compression-interface interaction.

Comparing the evaluations of expressions
\eqref{eq:vorticity_comparison_broke} and
\eqref{eq:vorticity_comparison} we expect two things. First, that
baroclinicity will be the dominant physical mechanism by which
circulation is generated. Hence, we expect
that \eqref{eq:vorticity_comparison_broke} is likely to overestimate
the dominance of baroclinic vorticity after a small amount of
baroclinic vorticity has been generated. Second, we expect the ratio
of the baroclinic to compressible contributions to vorticity
generation will range from $\orderof{10^3}$ to $\orderof{10^2}$ during
the compression-interface interaction.

\subsection{Comparison of vorticity generation in air and water}
Having established that the dominant source of vorticity is
baroclinicity we now aim to determine where this vorticity will
generated within the mixed gas-liquid interface region. Specifically, we
aim to compare the order of baroclinic vorticity generation from
equation \eqref{eq:baroclinic_vorticity} in pure water and air. As
this can already be evaluated in water from what we have provided up
to this point, we will focus on evaluation of the order of baroclinic
vorticity generation in air, from equation
\eqref{eq:baroclinic_vorticity}. Throughout the analysis we will
denote the properties of the incoming wave and water with a subscript
$-$, and the transmitted wave and air with a subscript $+$. For water,
we will use the values for
$\Delta \rho_I, \Delta L_I, \Delta \rho_a, \Delta L_a$ and $\theta$
defined in the previous section based on our initial condition. Our
treatment of the density gradient at the interface will remain
unchanged for evaluation in air such that
$\Delta \rho_I^-=\Delta \rho_I^+$ and $\Delta L_I^-=\Delta L_I^+$.

As a portion of the acoustic wave is transmitted into air, it
undergoes several physical changes relative to the incident wave. To
describe the properties of the transmitted wave in air we will borrow
techniques from linear acoustics. To find the pressure change in the
transmitted compression wave $\Delta p_a^+$, we recognize that
$a_0/\lambda<<1$ and treat the incoming wave as a plane wave impinging
normally on a flat material interface such that
$\Delta p_a^+=\bs{T} \Delta p_a^-$, where $\bs{T}$ is the acoustic
transmission coefficient,
$\bs{T}=2\rho^+ c^+/\left(\rho^+ c^+ + \rho^- c^- \right)$
\citep{Kinsler1982}. For our water-air interface
$\bs{T}\approx4.97\times10^{-4}$. Because of the strong impedance
mismatch between fluids, the acoustic wave is almost entirely
reflected, decreasing the pressure gradient of the transmitted wave
relative to the incident wave. Because of the drop in sound speed
across the interface, the transmitted wave is compressed into a
smaller physical area (i.e., the wavelength decreases) relative to the
incoming wave, such that $\Delta L_a^+=\Delta L_a^- (c^+/c^-)$. This
effect increases the pressure gradient in the transmitted wave. To
evaluate $\theta^+$, we utilize Snell's law which states that
$c^-\sin{(\theta^-)}=c^+\sin{(\theta^+)}$. Tedious, but simple
geometric arguments can be used to show that because $a_0/\lambda<<1$
it is also true that $\theta^-<<1$. Thus we use the small angle
approximation of $\sin$ to find that
$\theta^+\approx\theta^-(c^+/c^-)$. We note that this decreases the
misalignment between the pressure and density gradients in air, and
quantitatively approximately cancels the increase in pressure gradient
due to the decrease in wavelength of the transmitted wave.

To determine where the vorticity will be generated at the interface,
we consider equation \eqref{eq:baroclinic_vorticity} in air and water
and write the ratio to find
\begin{align}%
\label{eq:baroclinic_air_water}%
%\left(\frac{\partial\omega}{\partial t}\right)_{\substack{\text{baroclinic}\\\text{air}}} / \left(\frac{\partial\omega}{\partial t}\right)_{\substack{\text{baroclinic}\\\text{water}}}%
\frac{\norm{\frac{\nabla\rho\times\nabla p}{\rho^2}}_{air\quad}}{\norm{\frac{\nabla\rho\times\nabla p}{\rho^2}}_{water}}
=&\orderof{\frac{\left[\frac{\abs{\Delta \rho_I^+}}{\abs{\Delta L_I^+}}\frac{\abs{\Delta p_a^+}}{\abs{\Delta L_a^+}}\frac{1}{\abs{\rho^+}^2}\abs{\theta^+}\right]}
{\left[\frac{\abs{\Delta \rho_I^+}}{\abs{\Delta L_I^+}}\frac{\left(\abs{\Delta p_a^+}/\abs{\bs{T}}\right)}{\abs{\Delta L_a^+}\left(\abs{c^+}/\abs{c^-}\right)}\frac{1}{\abs{\rho^-}^2}\left(\abs{c^+}/\abs{c^-}\right)\abs{\theta^+}\right]}},\nonumber\\%
=&\orderof{\abs{\bs{T}}\left(\frac{\abs{\rho^-}}{\abs{\rho^+}}\right)^2}.%
\end{align}
For our water-air interface, we evaluate equation
\eqref{eq:baroclinic_air_water} to find that the ratio of baroclinic
vorticity generation in air to that in water would be of order
$\orderof{10^2}$. While this result considers vorticity generation in
pure air and water, as opposed to the mixed fluid region relevant to
this work, it provides a useful upper bound on the change we expect in
the vorticity across the interface. Additionally, this result suggests
that for the mixed water-air region, where the strongest density
gradient exists, vorticity generation is likely to occur in areas with
a lower volume fraction of water (i.e., gas-dominated fluid).

\subsection{Considerations of circulation}
In order to verify our analyses numerically we will consider not the
vorticity generation, but rather the circulation and circulation
generation as functions of time. As circulation is a global quantity
of vorticity integrated over a region, it is more practical to compare
to our numerical experiments. The expressions previously obtained for
estimates of vorticity generation can be integrated in space to obtain
integral expressions for circulation generation. As the expressions
derived were approximate and spatially independent, we expect that the
approximate vorticity relationships found in this section can be
extended to considerations of the circulation. For instance, based on
the results of equation \eqref{eq:vorticity_comparison} we expect the
baroclinic circulation generation in the left or right half-domain to
be $\orderof{10^2}$ larger than the compressible and advective terms
toward the end of interaction between the interface and the acoustic
compression.

To access this, we integrate equation \eqref{eq:vorticity_euler} over
the half-domain, $A_R$, to get
\begin{align} \label{eq:circulation_generation}
  \left(\frac{\partial \Gamma}{\partial t}\right)_{total} =%
  \left(\frac{\partial \Gamma}{\partial t}\right)_{compressible} + \left(\frac{\partial \Gamma}{\partial t}\right)_{baroclinic} - \left(\frac{\partial \Gamma}{\partial t}\right)_{advective},
\end{align}
%
Each term will be analyzed separately to determine the individual
physical contributions to circulation. Here
%
\addtocounter{equation}{-1}
\begin{subequations}\label{eq:circulation_generation_components}%
  \begin{align}% 
    &\left(\frac{\partial \Gamma}{\partial t}\right)_{compressible} &=& -\int_{A_R} \vec{\omega}\left(\nabla\cdot\vec{u}\right) \, dA_R,&\\
    &\left(\frac{\partial \Gamma}{\partial t}\right)_{baroclinic} &=& +\int_{A_R} \frac{\nabla\rho\times\nabla p}{\rho^2} dA_R,&\\
    &\left(\frac{\partial \Gamma}{\partial t}\right)_{advective} &=& +\int_{A_R} \left(\vec{u}\cdot\nabla\right)\vec{\omega} \, dA_R.&
  \end{align}
\end{subequations}
%

Finally, as we expect the interface growth to be purely circulation driven
long after all waves have left the domain, we perform dimensional
analysis to find a scaling law for the corresponding interface
perturbation amplitude $a(t)$ as a function of circulation and time,
\begin{align} \label{eq:intf_circ_scaling}
  a(t) \sim \sqrt{\Gamma t}.
\end{align}
This proposed scaling law will be compared to the late time dynamics
of the interface, after the acoustic wave has left the domain in
Section \ref{subsubsec:usbe_lung_amplitude_dependence}.















% % To perform the analysis, we recognize that any vorticity generated must
% % be a result of acoustic energy being convert to kinetic energy within
% % the flow at and around the interface. As such specifically we consider
% % the period in which the incoming compression wave encounters the
% % interface. As this interaction occurs quickly, over an approximate
% % time span $\Delta t_p\approx5\lambda/c_{w}$, we assume that the
% % interface is static and remains undeformed from its initial state
% % during this interaction. 

% % To evaluate the vorticity in the compressible and advective terms, we
% % write it as the curl of the velocity field
% % $\vec{\omega}=\nabla\times\vec{u}$. Because the only motion in the
% % flow is generated by the acoustic wave, we use acoustic relations to
% % write the velocity as $u=\pm p/(\rho c)$. Lastly we treat gradient, curl, and
% % divergence terms of any arbitrary quantity $f$ such that
% % $\nabla\cdot f= \orderof{=\left|f\right|/dY}$ and
% % $\nabla\times f=\left|f\right|/dY$. Note that for our a uniform grid
% % is used except where stretched at the top and bottom boundaries such
% % that $dY=dX$ for our interests.

% % With these treatments and assumptions we can immediately approximate
% % the order of the advective contribution to vorticity as
% % \begin{align}
% % \norm{\left(\vec{u}\cdot\nabla\right)\vec{\omega}} = \orderof{\left[\frac{\abs{p}}{\abs{\rho}\abs{c}}\frac{1}{\abs{dY}}\right]^2},
% % \end{align}
% % and the compressible contribution as 
% % \begin{align}
% % \norm{-\vec{\omega}\left(\nabla\cdot\vec{u}\right)} = \orderof{\left[\frac{\abs{p}}{\abs{\rho}\abs{c}}\frac{1}{\abs{dY}}\right]^2}.
% % \end{align}

% % In consideration of the baroclinic contribution to vorticity we expect
% % the density gradient to be dominated by that across the
% % interface. While the interface of our simulation is designed to obey
% % thermodynamic conditions, and as such does not occur over a single
% % $dY$, the thickness based on the distance from 5 to 95\% volume
% % fraction of water is initially $4.7~dY$ which we treat as
% % $\orderof{dY}$. Additionally, to account for the degree of
% % misalignment between the pressure and density gradients, we can
% % rewrite the cross product as
% % $\abs{\nabla \rho} \abs{\nabla p} \sin{\left(\theta\right)}$. Here
% % $\theta$ is the angle between the direction of the acoustic pressure
% % gradient which as being in the $+y$-direction and the direction of the
% % density gradient which we treat as the outward normal direction to the
% % interface. Thus we estimate the order of the baroclinic vorticity
% % generation rate as 
% % \begin{align}
% %  \frac{\nabla\rho\times\nabla p}{\rho^2} = \orderof{\frac{\abs{p}\overline{\sin{\left(\theta\right)}}}{\abs{\rho}}\frac{1}{dY^2}}.
% % \end{align}
% % Recognizing that we expect the advective and compressible
% % contributions to be of the same order, we compare them to the
% % baroclinic term for the strongest point in the compression wave in
% % which $p=p_a$ in water and $p=\bs{T} p_a$ in air. Here $\bs{T}$ is the
% % acoustic transmission coefficient, approximated based on sinusoidal
% % plane wave impinging normally upon an interface as
% % $\bs{T}=2\left[\rho c\right]_{air}/\left(\left[\rho
% %     c\right]_{air}+\left[\rho c\right]_{water}\right)$
% % \citep{Kinsler1982}. Here we note that we expect the majority of the
% % generated vorticity will likely occur in the mixed-fluid interface
% % region and that the actual values will likely lie between those
% % calculated for either pure fluid. To get an order of magnitude
% % estimate for the baroclinic term of we compute the mean
% % $\overline{\sin{\left(\theta\right)}}\approx0.12$ over the right half
% % of the interface for our setup with $a_0=0.03\lambda$. Plugging in the dimensionless material
% % values from Table \ref{tab:usbe_lung_dimensional_parameters} we
% % approximate that for our minimum acoustic pressure, $p_a=1$ MPa,

% % \begin{subequations} \label{eq:vorticity_oom}
% % \begin{align}
% % -\vec{\omega}\left(\nabla\cdot\vec{u}\right)~\text{and}~\left(\vec{u}\cdot\nabla\right)\vec{\omega} =%
% %   \begin{cases} 
% %     \orderof{10^{-6} \frac{1}{dY^2} } &\text(water), \\
% %     \orderof{10^{-5} \frac{1}{dY^2} } &\text(air),
% %   \end{cases}\\
% % \frac{\nabla\rho\times\nabla p}{\rho^2} =% \orderof{\frac{\abs{p}}{\abs{\rho}}\frac{1}{dY^2}}.
% %   \begin{cases}
% %     \orderof{10^{-4} \frac{1}{dY^2} } &\text(water),\\
% %     \orderof{10^{-1} \frac{1}{dY^2} }&\text(air),
% %   \end{cases}
% % \end{align}

% % and for our maximum acoustic pressure $p_a=10$ MPa, 

% % \begin{align}
% % -\vec{\omega}\left(\nabla\cdot\vec{u}\right)~\text{and}~\left(\vec{u}\cdot\nabla\right)\vec{\omega} =%
% %   \begin{cases}
% %     \orderof{10^{-4} \frac{1}{dY^2} } &\text(water),\\
% %     \orderof{10^{-3} \frac{1}{dY^2} } &\text(air),
% %   \end{cases}\\
% % \frac{\nabla\rho\times\nabla p}{\rho^2} =% \orderof{\frac{\abs{p}}{\abs{\rho}}\frac{1}{dY^2}}.
% %   \begin{cases}
% %     \orderof{10^{-2} \frac{1}{dY^2} } &\text(water),\\
% %     \orderof{10^{0} \frac{1}{dY^2} }&\text(air).
% %   \end{cases}
% % \end{align}
% % \end{subequations}


% % Based on this order of magnitude analysis we expect two things. First,
% % baroclinically generated vorticity is dominant over vorticity
% % generated through all other mechanisms. Hence the quantities
% % associated with this, such as acoustic pressure, are those of most
% % interest to our study. Second, due to the relatively high density of
% % water, we expect the majority of vorticity to occur in fluid with a
% % higher volume fraction of air than water. Additionally, we expect that
% % these trends will extend to considerations of the half-domain
% % circulation $\gamma$, which is an integral quantity of vorticity
% % $\omega$.

% % To verify the above in our results, we will look for two
% % things. First, as the above analysis suggests circulation generated
% % during the compression wave-interface interaction is predominantly
% % baroclinically generated. Because our acoustic pressure is
% % linearly-increasing we predict that circulation deposited during this
% % will also increase linearly with maximum acoustic pressure $p_a$, i.e.,
% % \begin{align} \label{eq:linear_circulation}
% %   \Gamma \sim \norm{\frac{\nabla \rho\times\nabla p}{\rho^2}} \sim p_a.
% % \end{align}

% % Second, to numerically verify our predictions for the types of
% % vorticity generated in a visualizable way, we integrate the vorticity
% % generation equation \eqref{eq:vorticity_euler} over the
% % half-domain. 
% % %
% % \begin{align} \label{eq:circulation_generation}
% %   \left(\frac{\partial \Gamma}{\partial t}\right)_{total} = \left(\frac{\partial \Gamma}{\partial t}\right)_{compressible} + \left(\frac{\partial \Gamma}{\partial t}\right)_{baroclinic} - \left(\frac{\partial \Gamma}{\partial t}\right)_{advective},
% % \end{align}
% % %
% % Each term will be analyzed separately to determine the individual
% % physical contributions to circulation at any point time. Here
% % %
% % \addtocounter{equation}{-1}
% % \begin{subequations}\label{eq:circulation_generation_components}%
% %   \begin{align}% 
% %     &\left(\frac{\partial \Gamma}{\partial t}\right)_{compressible} &=& -\int_{A_R} \vec{\omega}\left(\nabla\cdot\vec{u}\right) \, dA_R,&\\
% %     &\left(\frac{\partial \Gamma}{\partial t}\right)_{baroclinic} &=& +\int_{A_R} \frac{\nabla\rho\times\nabla p}{\rho^2}, dA_R,&\\
% %     &\left(\frac{\partial \Gamma}{\partial t}\right)_{advective} &=& +\int_{A_R} \left(\vec{u}\cdot\nabla\right)\vec{\omega} \, dA_R,&
% %   \end{align}
% % \end{subequations}
% % %

% % Finally, as we expect the interface growth to be purely circulation
% % driven long after all waves have left the domain, we perform
% % dimensional analysis to find a scaling law for the corresponding
% % interface perturbation amplitude $a(t)$ as a function of circulation
% % and time,
% % %
% % \begin{align} \label{eq:intf_circ_scaling}%
% %   a(t) \sim \sqrt{\Gamma t}.
% % \end{align}
% % %
% % This proposed scaling law will be compared to the late time dynamics
% % of the interface, after the acoustic wave has left the domain in
% % Section \ref{subsubsec:usbe_lung_amplitude_dependence}.

%%% Local Variables:
%%% mode: latex
%%% TeX-master: "../../prelim"
%%% End:
