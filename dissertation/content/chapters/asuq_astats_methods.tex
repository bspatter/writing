\section{Methods} \label{section:asuq_astats_methods}
This section describes the main components of this investigation: the
uncertain ocean environments and their characterization, the algorithm
steps followed for the area statistics technique, the implementation
details for the Monte Carlo calculations, and the quantitative means
for assessing the accuracy of the PDFs of TL produced from area
statistics. Sample results are provided to illustrate each component.

Four ocean environments with uncertain bathymetry, sound speed profile
(SSP), and seabed properties were considered in this study. The
environments, labeled 1 through 4 and ordered according to their
maximum depth from shallowest to deepest, are shown in cross section
in Figure 1. For environments 1 and 3, bathymetric data were obtained
from National Oceanographic and Atmospheric Administration’s multibeam
bathymetric survey database \nocite{multibeam_km0504,multibeam_ew9509}
(Multibeam collections for KM0504, EW9509), and are taken from the
challenger plateau west of New Zealand, in the South China Sea, and
off the coast of Taiwan. For environments 2 and 4, bathymetric data
were obtained from the US Office of Naval Research’s
(Vuln-TDA-STTR\_10-25-14\_samples), and are taken from the North
Pacific, near the Hawaiian Islands.

Five uncertain parameters were considered within each environment:
bathymetry offset , sound speed increment multiplier , seabed density
, seabed sound speed , and seabed attenuation . These are labeled and
depicted in Figure 2. Detailed specifications of these were needed to
complete the Monte Carlo calculations described below. The bathymetry
offset represented ocean depth uncertainty and was a random
range-independent distance that was added uniformly to the
range-dependent bathymetric survey depth of the environment. The
seabed was treated as a single layer with random range independent
properties ($\rho_b, c_b, \text{and} \alpha_b$).

The sound speed increment multiplier $\gamma$ requires a little more
description. For the environments where range- and depth-dependent SSP
information was provided (2 and 4), the range-averaged, but still
depth-dependent, mean $<c(z)>$ and standard deviation $<\sigma_c(z)>$ of the sound
speed were calculated. This range-averaged mean SSP $<c(z)>$ was then used
as the baseline SSP for TL calculations in each sample
environment. Uncertainty in the SSP was introduced by adding $\gamma<\sigma_c(z)>$ to
$<c(z)>$ for each individual Monte Carlo calculation, with $\gamma$ varying
randomly between calculations. Because only $<c(z)>$ was known for
environments 1 and 3, $<\sigma_c(z)>$ calculated for environment 2 was used to
incorporate SSP uncertainty in these environments.

All 5 uncertain parameters were assumed to have Gaussian-like
distributions, however the extremes of the distributions were
truncated as described below and the distribution values were then
renormalized.  The most-probable value and Gaussian standard deviation
of each uncertain parameter is provided in Table 1. Samples were drawn
randomly from these distributions for the Monte Carlo calculations.
The most-probable values, standard deviations, and parametric ranges
in Table 1 were intended to mimic the actual uncertainties associated
with readily available information about ocean environmental
parameters. The bathymetry offset values were intended to mimic tidal
fluctuations. The seabed values were developed from the mean and
standard deviation of the tabulated bottom layer properties for clay,
silt, sand, gravel, moraine, chalk, and limestone found in \citet{Jensen2011}. The distributions of , , , and were truncated at ± 3.
And, the distribution of was truncated by the properties of clay on
the low end and limestone on the high end to ensure that the value of
remained physically realistic.  The sound speed increment multiplier
values were based on appropriately incorporating tabulated variations
in ocean sound speed. Note that for each environment, the situation
where all of the uncertain parameters take on their most likely values
are referred to as the baseline case.

The area statistics technique is based on the assumption that the
uncertainty in the TL value at any range-depth ($r,z$) location in an
uncertain ocean sound channel can be obtained by considering the
variations in TL found near that location in the baseline TL
calculation. The procedure for area statistics merely requires the
baseline TL calculation at the frequency of the sound source, the ($r,z$)
location of interest within the calculated TL field, and an algorithm
or recipe for choosing and combining TL values near the location of
interest to form an estimate of the PDF of TL. Given the simple
assumption upon which the area statistics technique is based, the
primary development effort involved empirically determining how to
sample the TL field near the point of interest to achieve acceptable
results for the five uncertain environmental parameters. Thus, some of
the details of the algorithm described here would likely need revision
if additional (or fewer) environmental parameters were uncertain.

For the area statistics technique, the baseline TL field calculation
may come from any acoustic propagation model. For this investigation,
computed TL fields were obtained from the Range-dependent Acoustic
Model (RAM)\citep{Collins1994}. In each case, TL field calculations were
performed along an outward radial from a unity-strength harmonic
monopole sound source with frequency $f_s=$ 100 Hz, 200 Hz, or 300 Hz, and
placed 91 m, 137 m, or 183 m, below the ocean surface. The source
frequency and a nominal sound speed $c_s=$ 1500 m/s were used to calculate
the nominal wavelength, $\lambda_s$.

Using the baseline TL-field calculation, a simple algorithm or recipe
with three steps was developed to produce area statistics results for
any range-depth ($r,z$) location of interest. First, all the TL values
within a range-depth rectangle centered on ($r,z$) were collected and
weighted uniformly. Second, these TL values were sorted into a
histogram.  And third, this histogram was normalized to form an
estimate of the PDF of TL. For this study, the sample rectangle's
range and depth dimensions were 40$\lambda_s$ and 13$\lambda_s$,
respectively, and a nominal histogram bin width of 1 dB was
used. Here, the sample rectangle's dimensions were chosen to produce
suitable results for the uncertainties defined in Table 1. Different
sample rectangle dimensions, different TL sample weighting, and other
adjustments to the area statistics algorithm are likely to be
necessary for a different set of uncertainties.

Figure 3 illustrates this procedure at the range-depth location of
(5720 km, 240 m) in environment 2 for a 200 Hz sound source. In
Figure 3a), the TL sample area is indicated by the black box near the
center of the TL field plot. The TL sample area is shown in an
expanded view in Figure 3b) and is comprised of 720 individual TL
samples.  Two-dimensional linear interpolation was used to increase
the number of range columns in the computational grid such that the
final normalized histogram is comprised of 1715 TL samples. The
area-statistics-estimated PDF of TL developed from the TL samples in
Figure 3b) is shown in Figure 3c). For successful area statistics results,
at least 400 TL samples are needed, and these can be obtained by
two-dimensional linear interpolation if the baseline TL calculation
was performed with coarse resolution and does not provide enough TL
values within the sample rectangle. In addition, TL samples were not
collected from any portion of the sample rectangle lying above the
ocean surface or below the ocean floor. When necessary, the bathymetry
was linearly interpolated to calculate the water column depth at the
range associated with each column of the computational grid.

To assess the accuracy of the estimated PDFs of TL, they were compared
with Monte Carlo-generated PDFs of TL that were created from the
computed TL values at the location of interest. Here, 1000 separate TL
field calculations were completed used, each with values for the
uncertain environmental parameters, drawn randomly from the
probability distributions described in Table 1. These 1000 TL samples
were sorted into a histogram with 1 dB nominal width bins, and the
histogram was normalized to create a PDF of TL. Here the number of
Monte-Carlo samples (1000) was high enough so that the Monte-Carlo
generated PDFs of TL were comparably converged when compared to their
counterpart area-statistics-estimated PDFs of TL (typically 400 to
3000 samples). A sensitivity analysis for a 200 Hz source in each of
the four environments that involved increasing the number of
Monte-Carlo samples to 2000 did not produce significant differences.
Specifically, when compared to the 1000 sample results, the number of
test points at which engineering accuracy was achieved changed by 1\%
or less.

Quantitative comparisons of the area-statistics estimated PDFs of TL
(subscript 'AS') and the Monte-Carlo PDFs of TL (subscript 'MC') were
made with the $L_1$ error-norm:
\begin{align}
L_1=\int^{+\infty}_{-\infty}|PDF_{MC}(TL)-PDF_{AS}(TL)|d(TL),
\end{align}
the integrated absolute value of the difference between the two PDFs,
or, more intuitively, the non-overlapping area of the two PDFs.  For
PDF comparisons, is a convenient metric of accuracy because it is a
single quantity that inherently accounts for differences in the mean,
width, and shape of two PDFs. The error-norm is a dimensionless
quantity that is bounded between 0 and 2, with corresponding to a
perfect match between the two PDFs and corresponding to total mismatch
(no overlap) of the two of the PDFs. In this study, ≤ 0.50 was chosen
as the criterion for which an area-statistics-generated PDF was deemed
engineering-level accurate.  This criterion corresponds to a
difference of ≤ 2 dB mean error (a measure of PDF location) and ≤ 2 dB
standard deviation error (a measure of PDF width) in 84\% and 85\% of
test cases, respectively.

The $L_1$ error-norm is illustrated in Figure 4 for
engineering-level-accurate (Figure 4a) and -inaccurate (Figure 4b)
estimates for the PDF of TL. In both panels, the jagged marked area is
the error. The result shown in Figure 4a) comes from the range-depth
location (5720 m, 288 m) in environment 2, and the mean and standard
deviation of the area statistics PDF (solid curve) are 1.23 dB and
1.02 dB smaller, respectively, than those of the Monte-Carlo PDF
(dashed curve). The result shown in Figure 4b) comes from the
range-depth location (1140 m, 384 m) in environment 2, and the mean
and standard deviation of the area statistics PDF (solid curve) are
0.86 dB and 1.92 dB smaller, respectively, than those of the
Monte-Carlo PDF (dashed curve).

%%% Local Variables:
%%% mode: latex
%%% TeX-master: "../../prelim"
%%% End:
