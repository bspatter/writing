The objectives of this part of the presented dissertation work are (1)
to develop models for use in the study of diagnostic ultrasound wave
interactions within tissue and (2) to perform studies to investigate
the physics and fluid mechanics potentially relevant to specific
biological effects of diagnostic ultrasound. These objectives were
pursued for two motivating problems, bioeffects resulting from
ultrasound-induced cavitation of contrast agent microbubbles, and
diagnostic ultrasound-induced lung hemorrhage.

The following work was performed to accomplish the first objective:

\begin{itemize}
\item \textbf{A model for spherical bubble dynamics in viscoelastic soft tissue was developed.}\\
  In this work, the dynamics of ultrasound contrast agents driven by
  pulsed ultrasound was simulated \citep{Patterson2012}. Tissue was
  treated as a Voigt type viscoelastic material with properties based
  on those accepted in the relevant literature. In a novel
  contribution to this area, experimental ultrasound waves with known
  bioeffects amplitude thresholds \citep{Miller2008b} were used to
  drive the bubble dynamics.

\item \textbf{A novel model of an ultrasound pulse driven alveolus was developed.}\\
  A model of an ultrasound pulse-driven alveolus was developed. The
  system was modeled as a compressible fluid system consisting of an
  acoustic wave in water propagating at a perturbed air-water
  interface. Using dimensional analysis, it is shown that this model
  is appropriate for studying the dynamics that occur during the
  ultrasound-alveolar interaction, and for at least a brief period
  thereafter. This work is unique, as the developed model appears to
  be the first model of diagnostic-ultrasound alveolar interaction to
  consider the full set of conservation equations for mass, momentum,
  and energy. As a consequence of this, the developed model is able to
  capture nonlinear phenomena that cannot be explained by linear
  acoustics.
\end{itemize}


To accomplish the second objective, the presented models were used to
simulate the physics of the relevant problems. 

With regards to ultrasound-driven cavitation of contrast agent
microbubbles in a viscoelastic, soft tissue-like media -- simulations
of spherical ultrasound contrast agents driven by experimentally
generated pulse waveforms were performed. The following conclusions
were drawn:
\begin{itemize}
\item Calculated cavitation metrics in a theoretical viscoelastic
  media correlate strongly with experimentally observed
  bioeffects. Simulation results for the maximum dimensionless bubble
  radius $R_{max}/R_0$, a measure of the violence of a cavitation
  event, were classified based on whether or not the waveform was
  known to cause kidney hemorrhage in mice during contrast-enhanced
  ultrasound, in a previous study \citep{Miller2008}. From a plot of
  $R_{max}/R_0$ vs \ac{US} frequency it can be seen there are clear
  regimes in which bioeffects do and do not occur. The observed peak
  rarefaction amplitude pressure and likely the cavitation threshold
  rise with increasing frequency.
\item Cavitation dynamics and bioeffects thresholds depend on
  elasticity. Within the simulations, a range of elasticity values
  were considered for the media surrounding the bubble. It was found
  that increasing the elasticity could either enhance or diminish the
  strength of the simulated bubble response, based on standard
  cavitation metrics. The effect of elasticity was found to depend on
  the waveform of the driving \ac{US} pulse.
\item Previously established thresholds for inertial cavitation in
  water, $T_{max} = 5000$ K and $R_{max}/R_0 = 2$
  \citep{Apfel1991,Yang2005}, are not likely to apply directly to
  bioeffects. Based on the results of this study, these thresholds do
  not correspond to cases in which bioeffects were observed. Moreover,
  observed bioeffects thresholds for $R_{max}/R_0$ were shown to have
  strong frequency dependence. It seems unlikely that bioeffects
  thresholds are independent of cavitation thresholds, but rather that
  these thresholds are not generally valid for ultrasonically driven
  cavitation in a viscoelastic media.
\item Better mechanical characterization of tissue is needed. Existing
  values and models for the mechanical properties and behavior of
  tissue are incomplete. There is particularly little information
  available for stresses and strains that occur on the length and time
  scales relevant to cavitation and ultrasound.
\end{itemize}

In the interest of studying \ac{DUS}-induced lung hemorrhage, a two
part study was performed. Using variants on the aforementioned model
of an ultrasound-driven alveolus, the first study aimed to physically
describe the dynamics of acoustically-driven gas-liquid interfaces,
such as those that exist in the alveoli of the lungs.
\begin{itemize}
\item A model of an US pulse driven alveolus was developed.
\item The dynamics of an alveolar wall driven by $1.5$ MHz ultrasound
  pulses of $1, 2.5,$ and $5$ MPa amplitudes were simulated.
\item Interfaces were shown to continue straining after interacting
  with the pulse as a result of US-generated baroclinic vorticity. This
  occurred on timescales less than those expected between pulses for KHz
  PRF.
\item Viscous stresses were also computed.
\item Computed stresses and strains were compared to expected values
  from the literature.
\end{itemize}


\section{Recommendations for future work}