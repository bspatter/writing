The objectives of this part of the presented dissertation work are (1)
to develop models for use in the study of diagnostic ultrasound wave
interactions within tissue and (2) to perform studies to investigate
the physics and fluid mechanics potentially relevant to specific
biological effects of diagnostic ultrasound. These objectives were
pursued for two motivating problems, bioeffects resulting from
ultrasound-induced cavitation of contrast agent microbubbles, and
diagnostic ultrasound-induced lung hemorrhage.

The following work was performed to accomplish the first objective:

\begin{itemize}
\item \textbf{A model for spherical bubble dynamics in viscoelastic soft tissue was developed.}\\
  In this work, the dynamics of ultrasound contrast agents driven by
  pulsed ultrasound was simulated. Tissue was treated as a Voigt type
  viscoelastic material with properties based on those accepted in the
  relevant literature. In a novel contribution to this area,
  experimental ultrasound waves with known bioeffects amplitude
  thresholds \citep{Miller2008b} were used to drive the bubble
  dynamics.

\item \textbf{A novel model of an ultrasound pulse driven alveolus was developed.}\\
  An ultrasound pulse driven alveolus was modeled as a compressible
  fluid system consisting of an acoustic wave in water propagating at
  an air-water interface. Using dimensional analysis, it was shown
  that this model is appropriate for studying the dynamics that occur
  during the ultrasound-alveolar interaction, and for at least a brief
  period thereafter. This appears to be the first model of
  diagnostic-ultrasound alveolar interaction to consider the full set
  of conservation equations for mass, momentum, and energy.
\end{itemize}


To accomplish the second objective, the presented models were used to
simulate the physics of the relevant problems. With regards to
ultrasound-driven cavitation of contrast agent microbubbles in a
viscoelastic, soft tissue-like media, the following conclusions were
drawn:
\begin{itemize}
\item Calculated cavitation metrics in a theoretical viscoelastic
  media correlate strongly with experimentally observed bioeffects.
\item Better mechanical characterization of tissue is needed,
  particularly at the time and length scales of cavitation and
  ultrasound.
\item Cavitation dynamics and bioeffects thresholds depend on
  frequency and elasticity.
\item Accepted thresholds for Inertial cavitation in water are not
  likely to apply directly to bioeffects.
\end{itemize}




\section{US Lung Conclusions}
\begin{itemize}
\item A model of an US pulse driven alveolus was developed.
\item The dynamics of an alveolar wall driven by $1.5$ MHz ultrasound
  pulses of $1, 2.5,$ and $5$ MPa amplitudes were simulated.
\item Interfaces were shown to continue straining after interacting
  with the pulse as a result of US-generated baroclinic vorticity. This
  occurred on timescales less than those expected between pulses for KHz
  PRF.
\item Viscous stresses were also computed.
\item Computed stresses and strains were compared to expected values
  from the literature.
\end{itemize}


\section{Recommendations for future work}