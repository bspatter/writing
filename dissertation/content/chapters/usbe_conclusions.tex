Two problems within the area of diagnostic ultrasound bioeffects
motivate the work of this thesis presented up to this point. For the
first problem, we study the dynamics of ultrasonically driven
microbubbles as it relates to \ac{CEUS}-induced bioeffects. For the
second problem, we study the physics of acoustic wave interactions with
gas-liquid interfaces, as it relates \ac{DUS}-induced lung hemorrhage.
The two primary objectives of this work are:
\begin{enumerate}
\item To develop computational models of the aforementioned \ac{US}
  bioeffects problems.
\item To perform numerical experiments using these computational
  models to gain insight into the physics and fluid mechanics
  underlying \ac{DUS} bioeffects in the context of \ac{CEUS} and
  \ac{DUS} of the lung.
\end{enumerate}
\section{Summary of key contributions and findings}
\subsection{Bubble Dynamics of Contrast Enhanced Ultrasound and Related Bioeffects}
To accomplish the first objective in the context of \ac{CEUS} and
related cavitation bioeffects, we developed \textbf{a model of a
  contrast agent microbubble subjected to a \ac{DUS} pulse within a
  compressible, viscoelastic soft tissue \citep{Patterson2012}}, based
on the works of \cite{Keller1980,Yang2005}. As such, the bubble is
modeled as a spherically symmetric ideal gas body. The surrounding
tissue is modeled as a compressible, Voigt viscoelastic material with
properties relevant to soft tissues based on the literature. The
ultrasound wave is modeled as a uniform change in the pressure
immediately surrounding the bubble. In a novel contribution to the
field, experimentally measured ultrasound waveforms, with known
bioeffects thresholds \citep{Miller2008b} were used to drive the
bubble.

  To accomplish the second objective, \textbf{we performed simulations
    of bubble dynamics for bubbles driven by several experimentally
    measured \ac{US} pulses in soft tissue with variable material
    parameters including viscosity and elasticity}. Ultrasound waves
  with frequencies ranging from $1.5$ to $7.5$ MHz and \acs{PRPA}
  ranging from less than $1$ to greater than $6$ MPa were used. For
  each frequency, the threshold \ac{PRPA} associated with the onset of
  bioeffects (specifically, glomerular kidney hemorrhage in rats) was
  known \citep{Miller2008b}. Metrics associated with the simulated
  cavitation and bubble dynamics were related to the ultrasound
  parameters and bioeffects thresholds and the following four conclusions
  were drawn:
\begin{itemize}
\item Calculated cavitation metrics in a theoretical viscoelastic
  media correlate with experimentally observed bioeffects. Simulation
  results for the maximum dimensionless bubble radius $R_{max}/R_0$, a
  measure of the violence of a cavitation event, were classified based
  on whether or not the waveform was known to cause kidney hemorrhage
  in rats subject to \ac{CEUS} in a previous study
  \citep{Miller2008b}. From a plot of $R_{max}/R_0$ (a common
  cavitation metric) vs \ac{US} frequency, it is clear that for a
  given frequency there are distinct regimes in which bioeffects do
  and do not occur. Explicitly, it was observed that above a certain
  threshold values of $R_{max}/R_0$ and \ac{PRPA}, bioeffects always
  occurred, below a different set of threshold values they did not
  occur. These bioeffects thresholds increased with increasing
  frequency, and it is likely that the inertial cavitation thresholds
  increased in a similar fashion.
\item Cavitation dynamics and bioeffects thresholds depend on
  elasticity, though the relationship is not trivial. Within the
  simulations, tissue elasticity ranged from $5$ to $1000$ kPa. It was
  found that increasing the elasticity could either enhance or
  diminish the strength of the simulated bubble response, based on
  standard cavitation metrics. For kilopascal order values of
  elasticity, the bubble dynamics mimicked those expected for an
  identical experiment in water. The effect of elasticity was
  found to depend on the waveform of the driving \ac{US} pulse. With
  the bubble response showing a greater deviation from that expected
  in water, for higher amplitude and increasingly nonlinear wave.
\item While never intended to accurately represent cavitation in
  tissue, previously established thresholds for inertial cavitation in
  water, $T_{max} = 5000$ K and $R_{max}/R_0 = 2$
  \citep{Apfel1991,Flynn1975a}, are not equivalent to bioeffects
  thresholds. Based on the results of this study, these thresholds do
  not correspond to cases in which bioeffects were observed. Moreover,
  observed bioeffects thresholds for $R_{max}/R_0$ were shown to have
  strong frequency dependence. It seems unlikely that bioeffects
  thresholds are independent of cavitation thresholds, but rather that
  these thresholds need to be adjusted for ultrasonically driven
  cavitation in a viscoelastic media.
\item Here, we perform a parameter sweep of several uncertain values,
  such as elasticity and bubble size demonstrate that deviations of
  these parameters, within a range of reasonable values, can lead to
  significantly different results in the simulated dynamics. However,
  better characterization of tissue and bubble properties is
  needed. Existing values and viscoelastic constitutive models for the
  mechanical properties and behavior of tissue are incomplete. There
  is particularly little information available for stresses and
  strains that occur on the length and time scales relevant to
  cavitation and ultrasound.
\end{itemize}

\subsection{Diagnostic Ultrasound-induced lung hemorrhage
  and acoustically driven gas-liquid interfaces}
To accomplish the first objective, \textbf{a novel model of an
  ultrasound pulse-driven alveolus was developed.} An ultrasonically
driven alveolus is modeled as a 2D compressible fluid system. The
alveolus is modeled as air and the surrounding tissue as water, with a
sinusoidally perturbed interface between the two. The ultrasound wave
is treated as an acoustic wave existing initially in water, which is
then allowed to propagate toward the interface. As such the overall
computational model system consists of a rectangular domain containing
an acoustic wave in water propagating toward a perturbed air
interface. Using dimensional analysis, it is shown that this model is
appropriate for studying the dynamics that occur during the
ultrasound-alveolar interaction, and for at least a brief period
thereafter. This work is unique, as the developed model appears to be
the first model of diagnostic-ultrasound alveolar interaction to
consider the nonlinear conservation equations for mass, momentum, and energy. As
a consequence of this, the developed model is able to capture
nonlinear phenomena that cannot be explained by linear acoustics, but
which we show are important to the system dynamics.

In pursuit of the second objective, the model described above was used
in a two-part study: \textbf{First, numerical experiments of perturbed
  gas-liquid interfaces driven by trapezoidal acoustic waves were
  performed and studied to describe the fundamental fluid dynamics of
  an acoustically driven gas-liquid interface.} Acoustic parameters
such as the peak pressure amplitude and wave duration (length) are
varied and their effect on the system dynamics, particularly with
regards to vorticity and interface perturbation growth, is
studied. Using dimensional analysis we develop mathematical
relationships to describe the response of the interface to the
trapezoidal acoustic wave.  \textbf{Second, we simulate alveoli-like
  perturbed liquid gas-interfaces, driven by ultrasound pulse
  waveforms.} Peak acoustic pressure amplitude and initial
perturbation amplitude (which corresponds to geometry) are varied to
study the dynamics for a range of parameters relevant to diagnostic
lung ultrasound. Computed interfacial stresses and strains are
compared with alveolar failure criteria. The following five
conclusions are drawn with regard to acoustically driven gas-liquid
interfaces and \ac{DUS}-induced lung hemorrhage.
\begin{itemize}
\item Acoustically generated baroclinic vorticity may be capable of
  appreciably deforming perturbed liquid-gas interfaces. For the cases
  studied here, this was due to the substantial acoustic pressure
  difference over a short length (megapascals over millimeters), and
  the nearly discontinuous density profile of the interface. Although
  the rise from and return to ambient pressures associated with
  acoustic waves suggests that net vorticity deposited should be zero,
  such an argument overlooks the transient nature of the process,
  namely the fact that the baroclinic torque may drive the interface
  throughout the wave-interface interaction, such that the density
  gradient is non-constant.
\item Initially smooth interface perturbations, driven by residual
  baroclinic vorticity may experience asymptotic power-law growth. The
  rate of this growth depends on the circulation density at the point
  in time when the direction of the bubble and spike (interfacial
  peaks and troughs) velocity can no longer be explained by linear
  acoustics, necessitating a description of the dynamics which
  captures the effects of vorticity.
\item Changes in the acoustic waveform that have little effect on the
  interface dynamics during the wave-interface interaction, may have a
  significant long-term effect on the evolution of the interface
  through the residual vorticity deposited at the interface.
\item \ac{US} pulses with diagnostically relevant parameters may be
  capable of inducing significant deformation of gas-liquid interfaces
  through the generation of baroclinic vorticity at interface
  perturbations. Gas-liquid interfaces driven by $1.5$ MHz \ac{DUS}
  pulses with \ac{PRPA}s ranging from $1$ to $5$ MPa were found to
  deform long after the passage of the wave. For the $5$ MPa waves,
  observed interfacial strains were reached as high as 38\%, far
  greater than 8\% expected strain failure thresholds for alveolar
  walls \cite{Belete2010}. This deformation occurred over a period of
  fewer than $300~\mu$s, which is less the time period between pulses
  for a typical \ac{DUS} pulse repetition frequency of 1 kHz.
  Furthermore, based on dimensional arguments, it was found that the
  baroclinic vorticity driving the deformation is likely to persist
  over multiple milliseconds ($\ell^2/\nu=\orderof{\mbox{ms}}$),
  suggesting that the use of many subsequent pulses, as is the case
  for clinical \ac{DUS}, may result in an accumulation of vorticity,
  and thus greater interfacial strain.
\item Newtonian viscous stresses alone are not likely to be
  responsible for \ac{DUS}-induced lung hemorrhage. The largest
  estimated viscous stresses observed were on the order of tens of
  pascals and were multiple orders of magnitude beneath expected
  stress failure thresholds.
\end{itemize}

\section{Overall conclusions}
Beyond the specific problems of interest studied in this thesis, we
consider the bigger picture of using computational modeling and
numerical experiments to study \ac{US} bioeffects problems. With
regard to this theme, the following two conclusions are drawn based on
the cumulative efforts presented in this thesis:
\begin{itemize}
\item Computational modeling can play a unique and useful role in
  investigating the physics that underlies ultrasound bioeffects. The
  purpose of computational studies such as those presented here is to
  gain insight which can be useful for supplementing, explaining, and
  guiding experiments. When used as a supplement to experimental
  techniques computational models can provide estimates of difficult
  to measure physical quantities that may play an important role in
  the occurrence of the biological effects such as elevated
  temperature and pressure in a collapsing bubble, as Chapter
  \ref{ch:usbe_bubble}. Additionally computational models can be
  useful when trying to answer questions that cannot be readily
  treated through modern experimental methods. For instance
  \ac{DUS}-induced lung hemorrhage cannot be directly observed in real
  time through modern medical imaging techniques due to the structural
  complexity of the lung and surrounding tissues and the small spatial
  and time scales associated with the hemorrhage. Partly as a
  consequence of this, the mechanism driving the hemorrhage is still
  unknown. However, potential physical damage mechanisms can be
  discovered and studied through computational modeling. This is the
  case in Chapters \ref{ch:usbe_lung} and \ref{ch:usbe_lung_bio}, in
  which acoustically-generated baroclinic vorticity induced strain is
  shown to occur at alveoli-like gas-liquid interfaces. This possible
  damage mechanism has not been previously considered. And while this
  mechanism cannot be confirmed numerically, it highlights the
  importance of nonlinearity in the problem, an aspect of the physics
  that is often ignored and worthy of further study.
  \begin{comment}
    The potential of computational modeling to answer questions of
    ultrasound bioeffects can only be realized if the models
    themselves hold at least qualitatively true to reality within the
    regimes of interest, which leads us to our next conclusion.
  \end{comment}

\item Based on the cumulative results of this part of the
  dissertation, we conclude that for computational modeling of
  ultrasound bioeffects to be optimally useful for research purposes,
  there is a necessity for more accurate physical characterizations of
  tissue than are currently available. And secondarily, for these
  models to ever be clinically useful for predicting individual
  occurrence of bioeffects, they will likely need to be adapted on a
  case-by-case basis. The models developed and used in this thesis are
  justified for their stated regimes based accepted data available at
  the time of their creation. However, it is widely known that soft
  tissues are complex media which exhibit a wide variety of physical
  properties that are not well characterized in all regimes. Physical
  properties such as elasticity, viscosity, and stress-strain
  relationships can vary widely depending on variable physical state
  of tissue (e.g., stress, strain, strain rate, temperature, degree of
  hydration, etc...). Reported values for these properties vary
  widely in the literature and are frequently unavailable entirely for
  the regimes of interest to many ultrasound bioeffects problems. For
  example, repeatedly experimentally validated viscoelastic models for
  soft tissues subject to strain rates of the order of those
  associated with inertial cavitation are rather hard to come by in
  this author's experience. In spite of this scarcity of information,
  the dynamics of the \ac{US} bioeffects problems can be highly
  dependent on these poorly characterized parameters, as in the case
  of the elasticity-dependent cavitation bubble dynamics of Chapter
  \ref{ch:usbe_bubble}. Furthermore, there are certain aspects of
  these problems that can vary widely from person to person, and as
  such for computational models to be of clinical use for predicting
  or estimating ultrasound bioeffects they will likely need to be
  adapted for these variations. For instance, in our consideration of
  diagnostic lung ultrasound in Chapters \ref{ch:usbe_lung} and
  \ref{ch:usbe_lung_bio}, we neglect the attenuation of the wave that
  would realistically occur before it enters the lungs. This
  attenuation depends on the thickness of the thoracic wall and as can
  be readily observed in any populated area, human geometries vary
  significantly between individuals. As such, two different people subject to
  identical \ac{DUS} pulses for diagnostic lung imaging may experience
  very different degrees of biological effects. If the models
  developed here are to be used in the future, it will be important to
  adapt them for the specific problem of interest based on the best
  available data at the time.
\end{itemize}

\section{Recommendations for future work}
In this section, we offer suggestions for which some of the works of
this thesis may be improved upon or expanded. With the end goals of
increasing the relevance of this dissertation work to the motivating
problems and applications and better understanding the underlying
physics, the suggested strategies for advancing the works of this
thesis can be generally summarized into a broad three-part
strategy. First, advance the computational model by adding physical
phenomena neglected here for fundamental study and problem
tractability, but which may be of relevance to the motivating problem
or application. Second, where possible, use experimental data as part
of the initial problem setup, such as experimentally measured acoustic
waves or physical geometries. Third, perform simulations which attempt
to computationally replicate experimental studies or portions thereof
and compare results. For example, parametric studies with variable
ultrasound parameters could examine trends and dependencies which
have already been observed experimentally, such as the difference in
thresholds pressures with varying exposure duration. Where sensible,
use the resulting information gained from the numerical experiments to
look for new insight into the cause of the observed trends. In the
remainder of this section there is a short piece with more specific
recommendations for future work related to the study of \ac{CEUS}
bioeffects as in Chapter \ref{ch:usbe_bubble} and a more extensive
piece focusing on extending the work of Chapters \ref{ch:usbe_lung}
and \ref{ch:usbe_lung_bio}.

\subsection{Extending and improving the study of bubble dynamics at
  capillary breaching thresholds}
As the work of Chapter \ref{ch:usbe_bubble} was published several
years before the writing of this thesis, it is perhaps unsurprising
that the author's labmates and colleagues have already extended areas
of that work significantly. Specifically, bubble dynamics models that
account for heat and mass flux and incorporate more robust
viscoelastic constitutive models have been developed
\citep{Gaudron2015,Warnez2015,Barajas2017}. By combining these
advanced bubble models with the experimentally measured \ac{US} pulses
and bioeffects thresholds featured in Chapter \ref{ch:usbe_bubble}, it
may be possible to obtain a better understanding of the relationship
between the cavitation dynamics and observed bioeffects
thresholds. Specifically one could look for qualitative changes in the
bubble dynamics behavior that happens at or around the \ac{PRPA}
thresholds associated with the onset of hemorrhage. Though our work
considers contrast agent microbubbles after their shells have
ruptured, one could extend this work by also incorporating the
effects of the protein and lipid coatings that surround \ac{CEUS}
microbubbles, as was done by \cite{Marmottant2005}. While these
suggestions extend the work by advancing the spherically symmetric
bubble model, none of the proposed recommendations thus far captures
3D effects which may also be of importance. Along these lines, one
could use the basic problem setup, driving pressure waves, and initial
conditions from Chapter \ref{ch:usbe_bubble} to design direct
numerical simulation experiments, solving appropriate forms of the
equations for conservation of mass, momentum, and energy with a
relevant constitutive relationship and equation of state for closure.

\subsection{Extending and improving the physical model of \ac{DUS} lung-interaction}
One of the directions upon which one could build upon the work of
Chapters \ref{ch:usbe_lung} and \ref{ch:usbe_lung_bio} is to increase
the relevance to actual physical lung ultrasound. In the design of our
model, the consider an idealized problem setup with a well defined
interface and waveform. This is well suited for the type of
fundamental studies that are performed in this dissertation, because
various parameters carefully controlled and the dependence of the
dynamics on each can be isolated. Additionally, results from the
simplified model are more generalizable than those obtained from a
more realistic setup which would inherently require more specified,
unique geometries and waveforms. However, there is value in aiming to
increase the relevance of the model to the physical problem and there
are a variety of areas upon which our model could be made more
realistic and relevance to the motivating problem, by adjust the model
system's physics and geometry. Here we discuss a few of the
limitations of the present work in this regard and offer suggestions
for ways to overcome these limitations and extend and improve the
current work.
\subsubsection{Improving the lung model}
\begin{itemize}
\item \textbf{Inclusion of presently neglected physical mechanisms}\\
  In the introduction to the present work we present conservation
  equations \ref{eq:intro_conservation} for mass momentum and energy
  and then perform dimensional analysis to justify neglecting various
  physical effects during the time and spacial scales of interest to
  the studies performed here, and in Appendix
  \ref{sec:elasticity_appendix} we go one step further modeling the
  elastic and inertial forces at the interface to show that elasticity
  is of relatively little importance for the problems we
  consider. However, we note within this work that two of these
  effects, viscosity and elasticity, are likely to be more important
  if considering longer time scales, over which viscosity will
  dissipate energy and elastic forces will increase with increasing
  strain. These effects may be considered by adding the appropriate
  terms to the equations of motion that are solved here. While the
  ideal constitutive equation to relate the stress and strain is not
  well known for ultrasonic regimes, there has been work in this area
  that could be integrated into the existing framework and built
  upon. \cite{Lanir1983} developed a viscoelastic constitutive model
  relating the alveolar membrane and its liquid interface to the bulk
  tissue properties and later \cite{Kowe1986} and \cite{Denny2000}
  built upon this, developing alveolar finite element models. As these
  effects are completely excluded in the present work, any reasonable
  extension in this direction is likely to help generate solutions
  that are closer to the physical reality, particularly over the
  longer timescales associated with typical diagnostic imaging
  exposure durations.
\item \textbf{Use of realistic alveolar geometries} In the present
  work, the morphology of the alveolar wall is approximated as a
  smooth sinusoidal perturbation between fluids. However, as was
  illustrated in Figure \ref{fig:alveolar_histology}, the true
  morphology is far more complicated. This relevance of this work to
  mammalian \ac{DUS} could greatly be improved by building a problem
  geometry based on a histological cross-section of alveolar tissue,
  which can be obtained at micron-resolution using microfocal X-ray
  \cite[]{Litzlbauer2006}. These images could then be used to define
  an initial volume fraction field based on the brightness of each
  pixel. This could be further extended to 3D by using the methods
  described by \cite{Parameswaran2009}. While the numerical methods
  used in this work are too computationally expensive to be practical
  for geometries such as this over the timescales of interest,
  experiments such as this would be useful for investigating the
  mechanism by which hemorrhage propagates into successive layers of
  alveoli.
\end{itemize}


\subsubsection{Improving the ultrasound model}
\begin{itemize}
\item \textbf{Experimentally measured \ac{US} pulse waveforms}\\
  The ultrasound pulse used in the work presented in Chapter
  \ref{ch:usbe_lung_bio} was a simplified waveform consisting of a
  sinusoidal wave modulated by a Gaussian envelope. Consequently,
  certain potentially important features of experimental ultrasound
  waveforms, such as nonlinearity and the resulting high pressure
  gradients, are not captured. By using experimentally measured
  waveforms to drive the problem, the effects of these features can be
  studied. This would also aid in the comparison of numerical and
  experimental results, which will be discussed in more detail at the
  end of this section.

\item \textbf{Simulations over clinically relevant timescales, with
    multiple pulses}\\Research suggests that the total number of
  pulses (\ac{PRF} $\times$ \ac{ED}) used in \ac{DUS} of the lung has
  a significant effect on \ac{US}-induced hemorrhage
  \citep{OBrien2005,OBrien2001a}. And the results of Section
  \ref{subsubsec:transient} suggest that circulation deposition and
  therefore interfacial strain and perturbation growth may be
  controllable using multiple carefully designed pulses. However, the
  current computational model of an ultrasound-driven alveolus is too
  computationally expensive to simulate multiple \ac{US} pulses at a
  realistic \ac{PRF} (e.g, $\sim 1$ KHz). This problem is in large
  part a consequence of the highly variable length and timescales that
  exist in \ac{DUS} of the lung. First, we consider the difference in
  length scales between the alveolus ($\orderof{10^{-4}}$ m) and the
  physical length of the acoustic wave ($\orderof{10^{-3}}$
  m). Because the acoustic wave begins in the domain, the domain must
  be sufficiently large enough to capture it entirely and allow it to
  leave without significant boundary effects. However, to also capture
  the dynamics of the interface, it is necessary to use sufficiently
  high resolution, particularly within portion of the domain
  containing the interface, which often changes considerably over the
  course of a simulation. Second, we consider the difference in
  timescales between the wave-interface interaction ($\sim\mu$s) and
  the amount of time between pulses, over which the interface is
  expected to continually evolve ($\sim$ms). Such that even using an
  adaptive timestep obeying the Courant-Friedrichs-Lewy condition
  ($\mbox{CFL} = \frac{\boldsymbol{u}_{max}\Delta t}{\Delta x}\leq
  0.5$), the problem must run for many timesteps (typically
  $\sim{10^5}$ for a $300 \mu$ s simulation) due to the high spacial
  resolution. Thus we have a large domain, at high resolution, running
  for many timesteps, and hence a computationally expensive problem,
  which in practice takes weeks to months of real-time to simulate.
  
  To decrease the computational cost of the simulation in a way that
  simultaneously allows for the use of multiple ultrasound pulses we
  suggest implementing time-dependent boundary conditions to prescribe
  the incoming acoustic wave and prevent reflections. This could be
  done using the methods described by
  \cite{Thompson1987a,Thompson1990a}. The dynamic creation of the
  acoustic waves at the boundary, such that they do not need to be
  prescribed in the initial domain, would allow the computational
  domain to be shortened by the length of the wave at the very
  least. Furthermore, because these boundaries can be designed to be
  strongly non-reflective, the need for the stretched grid at the top
  and bottom of the domain is removed. As such, the overall domain
  size may be decreased by as much as tenfold in the vertical
  direction, greatly reducing the computational costs. Additionally, a
  time-dependent boundary formulation would allow for the creation of
  waves at late times, such that multiple pulses could be simulated.

  The implementation of the suggested time-dependent boundary
  conditions \ac{DG} spatial schemes is technically difficult. While
  it is possible to adapt these techniques \citep{Toulopoulos2011}, it
  is recommended that if this course of research is undergone, one
  might consider other numerical techniques suitable for these
  problems, such as \ac{WENO} methods. While these methods lack
  certain advantages of \ac{DG} (e.g., compact stencil), they offer
  benefits such as being more readily developed for the solution of
  the Navier-Stokes equations, if one wished to add viscosity to the
  problem, as suggested above.

\item \textbf{Comparison of computational and experimental results}\\
  In order to test the hypotheses proposed within this thesis, there
  is a need for fundamental liquid-gas interface experiments with
  waveforms relevant to \ac{DUS}. Once an theoretical explanation of
  the fluid mechanics of these interface problems (as is offered in
  this thesis) is experimentally validated, simulations closer to
  reality should be performed. While the studies and simulations
  performed as part of this dissertation work do not occur over the
  typical timescales associated with \ac{DUS}-induced lung hemorrhage,
  by implementing the previous suggestions we can begin to simulate
  something much closer to the typical experimental setups used to
  study this problem. As was done for \ac{CEUS} in Chapter
  \ref{ch:usbe_bubble}, simulated dynamics can be compared to
  experimental results. To further study the feasibility of the
  proposed physical mechanism underlying \ac{DUS}-induced lung
  hemorrhage: baroclinic vorticity driven strain of the alveolar wall,
  one could begin to more rigorously study the dependence of the
  vortex dynamics and associated strain on ultrasonic parameters for
  which the hemorrhage dependence is already well understood. The
  dependence of lung hemorrhage on a variety of ultrasonic parameters
  (e.g., pulse repetition frequency, pulse duration, effective dose,
  pulse frequency, exposure duration) been studied extensively. The
  dependence of baroclinic vorticity-induced strain on these
  parameters has not been rigorously investigated. One could go a long
  way toward supporting or ruling out the proposed damage mechanism by
  comparing the relationships between these parameters and hemorrhage
  to relationships between these parameters and simulated vorticity
  dynamics and strain.
\end{itemize}


