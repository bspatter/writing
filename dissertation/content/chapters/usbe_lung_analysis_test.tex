},\nonumber\\%
=&\orderof{\abs{\bs{T}}\left(\frac{\abs{\rho^-}}{\abs{\rho^+}}\right)^2}.%
\end{align}
For our water-air interface, we evaluate equation
\eqref{eq:baroclinic_air_water} to find that the ratio of baroclinic
vorticity generation in air to that in water would be of order
$\orderof{10^2}$. While this result considers vorticity generation in
pure air and water, as opposed to the mixed fluid region relevant to
this work, it provides a useful upper bound on the change we expect in
the vorticity across the interface. Additionally, this result suggests
that for the mixed water-air region, where the strongest density
gradient exists, vorticity generation is likely to occur in areas with
a lower volume fraction of water.

\subsection{Considerations of circulation}
In order to verify our analyses numerically we will consider not the
vorticity generation, but rather the circulation as a function of
time. As circulation is a global quantity of vorticity integrated over
a region, it is more practical to compare to our numerical
experiments. The expressions previously obtained for estimates of
vorticity generation can be integrated in space to obtain integral
expressions for circulation generation. As the expressions derived
were approximate and spatially independent, we expect that the
approximate vorticity relationships found in this section will remain
relevant in considerations of the circulation.  For instance, based on
the results of equation \eqref{eq:vorticity_comparison} we expect the
baroclinic circulation generation to be $\orderof{10^2}$ larger than
the compressible and advective terms toward the end of interaction
between the interface and the acoustic compression.

To access this, we integrate equation \eqref{eq:vorticity_euler} over
the half-domain, $A_R$, to get
\begin{align} \label{eq:circulation_generation}
  \left(\frac{\partial \Gamma}{\partial t}\right)_{total} =%
  \left(\frac{\partial \Gamma}{\partial t}\right)_{compressible} + \left(\frac{\partial \Gamma}{\partial t}\right)_{baroclinic} - \left(\frac{\partial \Gamma}{\partial t}\right)_{advective},
\end{align}
%
Each term will be analyzed separately to determine the individual
physical contributions to circulation. Here
%
\addtocounter{equation}{-1}
\begin{subequations}\label{eq:circulation_generation_components}%
  \begin{align}% 
    &\left(\frac{\partial \Gamma}{\partial t}\right)_{compressible} &=& -\int_{A_R} \vec{\omega}\left(\nabla\cdot\vec{u}\right) \, dA_R,&\\
    &\left(\frac{\partial \Gamma}{\partial t}\right)_{baroclinic} &=& +\int_{A_R} \frac{\nabla\rho\times\nabla p}{\rho^2} dA_R,&\\
    &\left(\frac{\partial \Gamma}{\partial t}\right)_{advective} &=& +\int_{A_R} \left(\vec{u}\cdot\nabla\right)\vec{\omega} \, dA_R.&
  \end{align}
\end{subequations}
%

Finally, as we expect the interface growth to be purely circulation driven
long after all waves have left the domain, we perform dimensional
analysis to find a scaling law for the corresponding interface
perturbation amplitude $a(t)$ as a function of circulation and time,
\begin{align} \label{eq:intf_circ_scaling}
  a(t) \sim \sqrt{\Gamma t}.
\end{align}
This proposed scaling law will be compared to the late time dynamics
of the interface, after the acoustic wave has left the domain in
Section \ref{subsubsec:amplitude_dependence}. Tacotacotaco

%%% Local Variables:
%%% mode: latex
%%% TeX-master: "../../prelim"
%%% End:
