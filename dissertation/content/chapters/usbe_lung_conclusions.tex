\section{Conclusions}
\label{sec:usbe_lung_conclusions}
This work is unique in that it demonstrates that acoustic waves may
trigger significant deformation of perturbed liquid-gas interfaces
over long periods of time. The driving mechanism behind this
deformation is baroclinic vorticity, which occurs as a result of
misalignment between the pressure gradient of the acoustic wave and
density gradient of the perturbed interface. This mechanism arises as
a result of nonlinear, compressible fluid mechanics, and cannot be
predicted through traditional linear acoustics. We suggest that
nonlinear effects such as baroclinic vorticity are important at
liquid-gas interfaces, such as those in the lungs, because of the
sharp density discontinuities between air and tissue within the
lungs. To demonstrate this we simulate acoustic waves with properties
relevant to \ac{DUS} impinging from water into air.

The work presented here supports the following three conclusions:%
%
(1) Baroclinic vorticity generated by acoustic waves within the
\ac{DUS} regime is capable of significantly deforming perturbed
liquid-gas interfaces. We observed that much of the vorticity
generated by the acoustic wave at the interface remains with the
interface as it evolves and deforms even long after the passage of all
acoustic waves. Part of this is attributed to a lack of physical
mechanism for dissipating vorticity in the inviscid case
considered. From dimensional analysis we find scaling law
\eqref{eq:intf_circ_scaling}, suggesting that the interface
perturbation amplitude will grow as $t^{0.5}$ for purely circulation
driven growth. In our computed results we find the actual perturbation
amplitude grows as $t^{0.6}$. This discrepancy does not appear to
change between $t=500$ and $1000$ and may be a result of the inability
of a global quantity $\Gamma$ to completely describe $a(t)$ which is
governed by local fluid mechanics.
%
(2) During interactions between acoustic waves and perturbed
liquid-gas interfaces, baroclinic vorticity is predominantly deposited
in the gas-dominated fluid. We perform analysis to predict that on either
side of an infinitely sharp water-air interface, the vorticity
generation rate would be approximately two orders of magnitude greater
on the air side of the interface than in the water. This is
qualitatively supported by our computational results which find that
near the end of the initial compression wave-interface interaction
nearly all of the circulation exists in fluid dominated by air. For
the $10$ MPa wave, for instance, 97\% of the circulation is found in
fluid with volume fraction of water $\alpha<0.5$ at $t=1$, after 91\%
of the compression has passed.
%
(3) Changes in the acoustic waveform that have little effect on the
interface dynamics during their interaction can substantially effect
the interface over longer periods of time, via vorticity. By comparing
the effects of $10$ MPa trapezoidal waves with varying static pressure
durations between compression and expansion, we observe that the
evolution of the interface between these two wave components
drastically effects the ultimate growth rate of the interface. The
phase and amplitude of the interface perturbation at the time it
encounters the expansion wave determine the direction and magnitude
respectively of the vorticity deposited. Consequently, the amount of
vorticity remaining at the interface and in the surrounding fluid
after the passage of the wave changes greatly based on the
time-dependent features of the wave.

This work is a step toward understanding the effects of acoustically
generated vorticity on gas-liquid interfaces, however we acknowledge
that there are many questions left to be answered. However, we
consider our findings in the context of \ac{DUS}-induced \ac{LH} and
propose a previously unconsidered potential damage mechanism. We
hypothesize that baroclinic torque occurs at fragile air-tissue
interfaces of the lung due to misalignment between the \ac{US}
pressure gradient and material interface density gradient, causing
stress, deformation, and ultimately rupture at the interface. We note
that evaluation of the hypothesized damage mechanism will require
considerable further work including experiments, and numerical
simulations that incorporate realistic viscosity, elasticity,
attenuation, and realistic waveforms and lung geometries. This
validation is beyond the scope of the proposed dissertation.

%%% Local Variables:
%%% mode: latex
%%% TeX-master: "../../main"
%%% End:
