\section{Introduction} \label{section:asuq_astats_intro}
Remote sensing in the ocean is primarily managed through the broadcast
and/or reception of acoustic waves. Computational acoustic field
models are commonly used to assess the extent of radiated sound
fields, and to extract information from recorded signals via signal
processing routines. Unfortunately, the ocean environment parameters
necessary for fully exploiting the capabilities of modern acoustic
field models are seldom (if ever) known with sufficient precision to
prevent uncertainty in ocean parameters from influencing the predicted
acoustic fields. Yet, understanding and quantifying the uncertainty
associated with a given field calculation is important for determining
its utility.

In underwater acoustics, the uncertainty in acoustic field predictions
arises from limited knowledge of the physical and geometric properties
of the ocean environment of interest. Consequently, acoustic field
predictions are typically made using imperfect estimates of
environmental parameters, and, as a result, the predicted fields
themselves are also uncertain. For a harmonic acoustic field produced
by a point source, the most fundamental attribute is the field's
amplitude, and it is commonly reported as transmission loss (TL), a
field quantity that has been part of sonar engineering for many
decades \citep{Urick1962}. Uncertainty in TL predictions has received
increased attention in recent years due to its utility in practical
naval applications \citep{Abbot2002,Pace2002} and ocean measurement
system design \citep{Munk1994}.  Unfortunately, there is no known
general relationship between environmental uncertainty and TL field
uncertainty, and the most common techniques for calculating TL
uncertainty, Monte Carlo and direct sampling methods, are too
computationally expensive to be practical for real time
applications. The purpose of this paper is to introduce and describe
an approximation technique, area statistics, as a computationally
efficient alternative to Monte Carlo and direct sampling methods, or
other means for producing the probability density function of TL at a
point of interest within an uncertain ocean environment.

The topic of acoustic uncertainty in ocean environments has seen
considerable interest in the last decade or so
\citep{Livingston2006}. The physical uncertainty of an ocean
environment has been shown to have considerable impact on naval
applications ranging from sonar performance prediction to tactical
decision aids and threat assessment. Accordingly, there has been much
work within the field of underwater acoustics toward two goals: (1)
understanding and quantifying environmental and acoustic field
uncertainties, and (2) determining how these uncertainties affect
relevant applications
\citep{Abbot2002,Emerson2014,Sha2005,Stone2004}. The technique
described here primarily addresses the first goal through
computationally efficient predictions of TL field uncertainty based on
typical ocean environment uncertainties.

There have been multiple studies aimed at accurately describing
environmental uncertainties. This is a challenging task given the
complexity and variability of the ocean water column and seabed
properties, especially in shallow waters
\citep{Livingston2006}. Uncertainties associated with archived
bathymetry data sets obtained without the use of modern multi-beam
technology have been reported \citep{Calder2006}, and historical data
have been used to describe seasonal sound-speed uncertainties on the
continental shelf and slope in the Middle Atlantic Bight
\citep{Linder2006}. 

The task of predicting acoustic TL field uncertainties that arise as a
result of the uncertain environment is the focus of this paper. The
starting point is a single baseline TL-field calculation that provides
TL as function of range and depth within the ocean along a chosen
azimuthal direction. For this baseline calculation, all uncertain
environmental parameters are set to their most probable values. The
Probability Density Function (PDF) of TL is used here to quantify the
uncertainty of baseline TL values since it contains all the relevant
TL statistics for ocean applications \citep{Gerstoft2006}. The mean
and standard deviation of TL, which may be reflective of the macro-
and micro-states of the ocean, respectively \citep{Abbot2006}, are
readily calculated from the PDF of TL. The techniques currently
available for predicting the PDF of TL require differing levels of
computational effort. These are described in the following paragraphs
from highest to lowest computational cost, as assessed by the number
of additional TL field calculations – beyond the baseline calculation
– necessary to implement the technique.

Monte Carlo and direct sampling methods are well-accepted techniques
for obtaining PDFs of TL, but their computational effort increases
exponentially as the number (N) of uncertain environmental parameters
increases. For both techniques, a potentially large set of TL
calculations is undertaken that sample the N-dimensional space of
uncertain environmental parameters in a random (Monte Carlo) or
structured (direct sampling) manner. The PDF of TL at any location in
physical space is then constructed from the computed TL values found
at that location in each of the many field TL calculations. Monte
Carlo calculations have been used to obtain the probability
distribution of TL subject to geoacoustic inversion uncertainty
\citep{Gerstoft2006}., and to explore acoustic sensitivity to
environmental parameters and assess the utility of a stochastic
description of environmental variables \citep{Heaney2006}. More
recently, Monte Carlo and direct sampling calculations have been used
to generate reference PDFs of acoustic field amplitude to assess the
accuracy of approximate PDF construction techniques
\citep{James2008,James2011}. Monte Carlo calculations based on 1000 TL
field calculations are used for this purpose in the work reported
here.

The mathematically rich technique of polynomial chaos expansions (PCE)
has also been used to assess acoustic uncertainty
\citep{Finette2005,Finette2006,Finette2009}. Here the uncertain
acoustic field is represented as a series of Q basis functions with
each function having its own range-, depth-, and frequency-dependent
coefficient. The coefficients are determined from the solution of a
set of Q coupled partial differential equations. The technique
produces converged uncertainty assessments as Q increases, with Q
being a proxy for the number field calculations when there is a single
uncertain environmental parameter (N = 1). However, when there are
more uncertain parameters (N ≥ 2), a different PCE solution technique
is needed and the approximate correspondence between Q and the number
of field calculations is lost. For comparison, the area statistics
technique described herein is simpler to implement than PCE and it
does not require the solution of any additional partial differential
equations beyond the baseline TL field calculation.

There are also approximate methods for predicting acoustic uncertainty
that do not involve the computational expense of Monte Carlo
simulation or mathematical complexity of PCE. A technique for
estimating TL confidence bounds for environments in which acoustic
propagation can be described by a sum of propagating modes has been
previously described \citep{Zingarelli2008}. The technique can be
applied when there are multiple uncertain parameters and it is
computationally efficient, as it only requires the baseline field
calculation. However, it inherently relies on range, depth, or
frequency averaging, and does not provide the full PDF of TL. Another
approximate method for predicting the PDF of acoustic field amplitude
for multiple uncertain environmental parameters is based on
determining spatial shifts between acoustic field calculations
completed with a difference in one uncertain parameter
\citep{James2008}. However, this field-shifting technique requires one
additional field calculation for each uncertain parameter.

The area statistics technique described here provides estimates of the
PDF of TL, and only requires the baseline TL field calculation. It is
simple and fast enough for implementation in real-time sonar
applications, and can be used in any environment for which TL field
calculations can be completed. As implemented here, it incorporates N
= 5 uncertain parameters, but the number and selection of these can be
altered. When compared to Monte Carlo results based on 1000 TL field
calculations, it reaches engineering level accuracy in 93\% of test
cases in downward-refracting acoustic environments that support
consistent bottom reflection.  The remainder of this paper is divided
into three sections. The next section describes the four uncertain
ocean environments used in this study, the area statistics technique,
and the procedures followed for generating the Monte Carlo
results. The third section presents quantitative comparisons between
area statistics and Monte Carlo results in the four ocean environments
at depths from 20 m to 5 km, and ranges from a few km to more than 60
km. The final section summarizes this effort and states the
conclusions drawn from it.

%%% Local Variables:
%%% mode: latex
%%% TeX-master: "../../prelim"
%%% End:
