
%%% I Already copied over and adapted the portion of this in the comment
\begin{comment}
  The proposed damage mechanism is based on the idea that acoustic waves
  are capable generating baroclinic vorticity at a perturbed fluid-fluid
  interface, which is capable of then driving the interface to
  deform. There has been extensive research into the fundamental physics
  describing interactions between mechanical waves, acceleration, and
  fluid-fluid interfaces. Much of this research is motivated by
  applications in fusion energy and astrophysics and accordingly has
  sought to investigate regimes outside of those of acoustic
  interests. \cite{Taylor1950} predicted that for an interface between
  two fluids of different density, if the fluid was accelerated normal
  to the interface in the heavy-to-light direction, perturbations at the
  interface would grow. That is to say that a ``bubble'' of light fluid
  penetrates the heavy fluid, and a ``spike'' of heavy fluid penetrates
  the light fluid. This is known as the \ac{RTI}. A similar topic of
  past study is the \ac{RMI}, which occurs when a perturbed fluid-fluid
  interface is instantaneously accelerated by a shock, causing the
  interface perturbation to grow \citep{Brouillette2002,Drake2006}. This
  growth is driven by a sheet of baroclinic vorticity deposited along
  the interface as a result of misalignment between the pressure
  gradient across the shock and the density gradient across the
  perturbed interface. This physical mechanism by which these misaligned
  gradients create a torque on fluid particles and generate vorticity
  can be thought of in terms of a hydrostatic balance upon a
  particle. Pressure gradients result in acceleration of the flow that
  is inversely proportional to density. When these two gradients are
  misaligned, the result is a shearing effect on the fluid and vorticity
  is generated. A graphical explanations of baroclinic vorticity
  generation and the resulting interface deformation can be found in
  \citep{Heifetz2015}. Analytically, baroclinic vorticity generation can
  be shown by taking the curl of the conservation of momentum equation
  for a compressible fluid. However we note that it is a nonlinear
  effect and cannot be explained by traditional linear acoustics.

  The physics of the \ac{RMI} are fairly well understood. For the
  classical \ac{RMI} setup a planar shock impinges normally upon the
  peaks and troughs of a sinusoidal interface. The interface is
  accelerated non-uniformly counter-rotating vorticies are generated
  across the interface. This drives peaks and troughs of the interface
  to accelerate in the opposite direction. Much like in the case of the
  \ac{RTI} instability, this too results in light fluid penetrating the
  heavy fluid and vice versa. For the case of a wave moving from a light
  fluid into a heavy one, the peaks and troughs of the interface
  accelerate away from one another, growing the interface perturbation
  perturbation. For the case of a wave moving from a heavy fluid to a
  lighter fluid, the peaks and troughs interface initially accelerate
  toward one another. They then pass each, inverting the phase of the
  interface perturbation, and continue moving in opposite directions,
  growing the perturbation amplitude.
200~As the degree of misalignment varies along the interface, the inter-
face is accelerated non-uniformly. The direction of the vorticity changes where the slope of the
interface changes. This counter rotation on either side of interface peaks and troughs entrains
nearby fluid causing interface peaks to accelerate in one direction and troughs to accelerate
in the opposite direction. This results in a “bubble” of light fluid penetrating the heavy fluid,
and a “spike” of heavy fluid penetrating the light fluid. How exactly this occurs varies slightly
depending on the relative densities of the two fluids. For the case of a wave moving from
a light fluid into a heavy one, the peaks and troughs of the interface are initially accelerated
to move away from one another, and the interface perturbation amplitude undergoes growth
exclusively. For the case of a wave moving from a heavy fluid to a lighter fluid, the peaks
and troughs of the interface are accelerated such that they initially move closer to one another
decreasing the perturbation amplitude. They then pass one another, inverting the phase of
the interface perturbation, and then continue moving in opposite directions, growing the per-
turbation amplitude. This process is illustrated in Figure 4.2, which has been adapted from
Brouillette (2002).\end{comment}


Previous studies of the \ac{RMI} have utilized theory, computation,
and experiments to describe the behavior of the interface after the
wave has passed. \cite{Richtmyer1960} performed the linear stability
perturbation analysis developed by \cite{Taylor1950} for the case of
an impulsive acceleration to create a model for the initial growth of
the interface perturbation. \cite{Meshkov1969} experimentally
confirmed Richtmyer's qualitative predictions, hence the name of the
instability. \cite{Meyer1972} performed numerical simulations of the
\ac{RMI} and found good agreement with Richtmyer for the case of a
shock impinging upon a light-heavy interface. \cite{Fraley1986} used
Laplace transforms in order to find the first analytical solution for
the asymptotic growth rate for a shocked interface between perfect
gases. To describe the late time, nonlinear growth of the
perturbation, \cite{Zhang1997} used single mode perturbation, keeping
many high order terms, to describe the velocity of the bubble and
spike regions of the fluid. \cite{Sadot1998} combined the linear,
impulsive solution with potential flow models of the asymptotic
behavior of the bubble and spike to develop a model for the
perturbation growth that is in good agreement with shock tube
experiments for shocks with Mach numbers Ma=1.3, 3.5. Vortex theory
has also been used to describe the behavior of the
interface. \cite{Jacobs1996} horizontally oscillated a container with
two vertically stratified liquids to obtain standing waves and then
bounced the container off of a coil spring to study the incompressible
\ac{RMI}. The late time evolution of is interface is modeled using a
row of line vorticies to obtain qualitatively similar results to those
experimentally observed, however the late-time growth rate is
underestimated. \cite{Samtaney1994} used shock polar analysis to find
the circulation deposited by a shock on planar and non-planar
interfaces. Their results are validated using and Euler code and found
to be within 10\% of the computed value for $1.0\,<\,$Ma$\,\leq\,1.32$
for all $\rho_2/\rho_1\,>\,1$, and
$5.8\,\leq\,\rho_2/\rho_1\,\leq\,32.6$ for all Ma.

%%% Local Variables:
%%% mode: latex
%%% TeX-master: t
%%% End:
