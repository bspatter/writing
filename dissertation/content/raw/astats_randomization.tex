\section{Area statistics randomizations}
\addcontentsline{loa}{section}{\protect\numberline{\thesection}\Sectionname}
%
This appendix is meant largely to serve for personal reference and
serves to explain some of the finer points of the Monte Carlo
randomization techniques used in Chapter
\ref{ch:astats}. Specifically, some of the details useful in
replicating or programming the bathymetry and sound speed
randomization are detailed herein.
\subsection{Bathymetry}
For each of $N$ Monte Carlo sample runs, the bathymetry profile is a
function of range $r$, depth $z$, and is determined through a
stochastic process represented by $\xi$.  
\begin{align*}
  D_n(r; \xi) = D_\mu(r) + D_\sigma(r,\xi_n(r)).%
\end{align*}
Where $n=0,1,2,...,N$. $D_\mu$ is the best estimate bathymetry profile
based on available databases. $D_\sigma(r,z;\xi)$ is the stochastic
portion of the sound speed profile and $\xi_n$ is a random event. The
random portion of the sound speed profile $D_\sigma$ is defined based
on the work of \cite{Lermusiaux2010}, and is dependent upon local
depth, normalized slope, and a global parameter $\epsilon$
representing relative deviation from the best guess (e.g. 1\%, 2\%, 3\%),
\begin{align*}
  D_\sigma(r,z)=D_mu(r)\epsilon\hat{S}(x,y)\xi,
\end{align*}
where $\hat{S}$ is local slope normalized by the maximum slope for the
best guess bathymetry profile.
\begin{align*}
  \hat{S}(r)=\frac{\left|\nabla D_\mu(r)\right|}{max(\left|\nabla D(r\right|)},
\end{align*}
such that $\hat{S}(x,y)\subseteq [0,1]$.

\begin{align*}
  D_n(r; \xi) = D_\mu(r) + D_\sigma(r,\xi_n).%
\end{align*}



%
\subsection{Sound speed profile}
For each of $N$ Monte Carlo sample runs, the sound speed profile
$c = c(r,z;\xi)$ is a function of range $r$, depth $z$, and is
determined through a stochastic process represented by $\xi$.
\begin{align*}%
  c_n(r,z;\xi) = c_\mu(r,z) + c_\sigma(r,z;\xi_n)%
\end{align*}%
Where $n=0,1,2,...,N$. $c_\mu$ is the best estimate sound speed
profile, which is calculated as the month-averaged velocity profile at
a given location $(r,z)$. $c_\sigma(r,z;\xi)$ is the stochastic
portion of the sound speed profile and $\xi_n$ is a random event. At
each range of interest, $c_\mu$ is obtained by finding the average
averaging 1 sound speed profile per month over an $M$ month
timespan. Monthly profiles are obtained from private
databases. %GDEM (Generalized Digital Environment Model), part of Oceanographic and Atmospheric Master Library (OAML).
% Unique sound speed profiles are obtained at approximately 30 arcminute
% intervals based on the resolution limits of the database. For the sake
% of implementing range-dependence in the sound speed profiles, these 30
% arcminute intervals are converted to approximately $56$ km. Note that
% while this is not strictly accurate as the conversion is spatially
% dependent, it is reasonable at sea level.

To calculate $c_\sigma(r,z;\xi_n)$ empirical orthogonal functions are
used to randomize the sound speed profile for each Monte Carlo sample
calculation. At a given range, the sound speed variation at each fixed depth can
be thought of an independent variable, such that the value of that
sound speed calculated for each month represents a new observation of
that variable. Hence, at each range, a matrix $\mcbs{C}$ is
constructed such that each column of $\mcbs{C}$ constitutes a single
sound speed profile taken at $D$ constant depths from a each of the
$M$ months considered. Such that we have $D$ random variables, each
with $M$ observations and $\mcbs{C}$ is of dimensions $D \times M$
(rows $\times$ columns). The covariance matrix of $\mcbs{C}^\intercal$ is
constructed
\begin{align*}
  \mcbs{X} = \frac{\left(\mcbs{C}-\mcbs{M}_\mu\right) \left(\mcbs{C}-\mcbs{M}_\mu\right)^\intercal}{M-1}.
\end{align*}
Here $\mcbs{M}_\mu$ is a matrix with the dimensions equal to that of
$\mcbs{C}$, for which each column is $c_\mu$, a $D \times 1$ vector
containing the row average of $\mcbs{C}$. We solve
\begin{align*}
  \mcbs{X}\mbbs{v}_i = \lambda_i\mbbs{v}_i
\end{align*}
to find the eigenvalues $\lambda_i$ and corresponding right
eigenvectors $\mbbs{v}_i$ of $\mcbs{X}$. Each of the random sound
speed profiles will be constructed from the mean sound speed profile
$c_\mu$, and a sum over $S$ randomly weighted orthogonal
eigenfunctions. We define $S$ as the number of eigenfunctions
necessary to capture 95\% of the variance in the sample sound speed
profiles, such that $S$ is the minimum integer which satisfies
\begin{align*}
  0.95\leq\frac{\sum_{j=1}^S \lambda_j}{\sum_{j=1}^\infty \lambda_j}.
\end{align*}

Hence the random component of each of the $n^{\text{th}}$ sound speed
profiles is defined as
\begin{align*}
  c_{\sigma,n}(r,z;\xi) = \sum_{j=1}^S \xi_{j,n} \sqrt{\lambda_j(r)}\mbbs{v}_j(r,z),
\end{align*}
where random event $\xi_{j,n}$ is sampled from a Gaussian distribution
centered at $0$ with unit variance. Thus for the $n^{\text{th}}$ sample calculation, the 
randomized sound speed profile at a range-depth location $(r,z)$ is described by
\begin{align*}
  c_n(r,z;\xi) = c_\mu(r,z) + \sum_{j=1}^S \xi_{j,n} \sqrt{\lambda_j(r)}\mbbs{v}_j(r,z).
\end{align*}




% A matrix of
% eigenvectors $\mcbs{V}$ is constructed such that the $i^{th}$ column
% is $\mbbs{v}_i$, where $i$ is ordered from largest to smallest
% eigenvalue such that $\lambda_i>\lambda_{i+1}$.

% \begin{align*}
%   PC=\mcbs{V} \left(\mcbs{C} - \mcbs{M}_\mu\right)^\intercal,
% \end{align*}
% where $\mcbs{M}_\mu$ is a matrix with the dimensions of $\mcbs{C}$,
% for which each column is the row average of $\mcbs{C}$. 

% \begin{align*}
%   \mcbs{C}_{rand} = \left( \bs{xi}^\intercal \left(PC \mcbs{V}^\intercal\right)\right)^\intercal
% \end{align*}

%%% Local Variables:
%%% mode: latex
%%% TeX-master: t
%%% End:
