\title{Dissertation Contributions}
\author{
        Brandon Patterson \\
        Department of Mechanical Engineering\\
        University of Michigan\\
}
\date{\today}

\documentclass[12pt]{article}

\usepackage[cm]{fullpage}


\begin{document}
\maketitle


\section{Ultasound bioeffects and related physics}

\subsection{Theoretical dynamics of contrast agent microbubbles at capillary breaching ultrasound amplitude thresholds}
\begin{itemize}
\item A model for spherical bubble dynamics in a compressible, Voigt viscoelastic medium is developed.
\item The cavitation dynamics of bubbles driven by experimentally obtained ultrasound pulse waveforms are calculated.
\item Accepted inertial cavitation thresholds for temperature and maximum bubble radius are compared with calculated results.
\item It is demonstrated that the cavitation dynamics and bioeffects thresholds are strongly dependent on frequency.
\item It is demonstrated that the cavitation dynamics are strongly
  dependent on elasticity, which is poorly characterized in tissue.
\end{itemize}

\subsection{Theoretical stresses and strains of ultrasound pulse-driven gas-liquid interfaces}
\begin{itemize}
\item A model of ultrasound-pulse driven alveoli was developed as a compressible, multi-phase fluid system with an acoustically-driven perturbed air-water interface.
\item The theoretical interface dynamics are calculated.
\item It is shown that baroclinic vorticity is capable of deforming the interface, long after waves have passed.
\item Theoretical passive viscous stress estimates are shown to be far less than expected damage thresholds.
\item It is shown that vorticity-induced strains exceed damage thresholds under certain circumstances.
\end{itemize}

\subsection{Dynamics of acoustically-driven gas-liquid interfaces}
\begin{itemize}
\item A model was developed for acoustically-driven gas liquid interfaces is developed.
\item The theoretical interface dynamics of trapezoidal acoustic wave-driven interfaces is calculated.
\item It is shown that baroclinic vorticity is capable of deforming the interface, long after waves have passed.
\item It is shown that interface perturbation grows approximately as $t^{3/5}$ and scales with the circulation density.
\item It is shown that the inverse circulation density or interface arc length per unit circulation scales with the wave amplitude.
\item It is demonstrated that the circulation deposition and therefore late-time interface dynamics depend heavily on time-dependent wave features.
\end{itemize}


\section{Efficiently estimating the probability density function of transmission loss in uncertain ocean environments}
\begin{itemize}
\item Area statistics, a method for quickly estimating the Probability Density Function (PDF) of Transmission Loss (TL) in uncertain ocean environments is developed.
\item An $L_1$-error norm is calculated to compare TL PDFs generated by area statistics to PDFs generated by Monte Carlo runs with randomized sound speed profile, bathymetry, and bottom layer density, sound speed, and absorption.
\item Area statistics is tested in environments of varying depth and bathymetries and for source frequencies of $100$-$300$ HZ.
\end{itemize}

\end{document}



%%% Local Variables:
%%% mode: latex
%%% TeX-master: t
%%% End:
