\begin{frame} \frametitle{We hypothesize that US waves generate baroclinic vorticity at gas-liquid interfaces, driving deformation.}%
  \only<1>{
    \begin{figure}
      \centering
      \begin{tikzpicture}%
        \node[anchor=south west,inner sep=0] (image) at (0,0) {%
          \def\svgwidth{0.6\textwidth} {\footnotesize
            \import{../figs/lung_figs/}{usbe_lung_schematic2.pdf_tex}
            \hfill%
          }};%
        \begin{scope}[x={(image.south east)},y={(image.north west)}]%
          \node[font=\footnotesize,right] at (0.6,0.72){ $\nabla p$};%
          \node[font=\footnotesize,right] at (0.59,0.515){
            $\nabla \rho$};%
          \node[font=\tiny,right] at
          (0.7,0.05){\textcolor{gray}{wikimedia.org}};%
        \end{scope}%
      \end{tikzpicture}%
    \end{figure}
        }%

  \visible<2->{
    The vorticity generation equation\vspace{2pt}
    \only<1-2>{
      \scalebox{1.0}{$
%        \frac{\partial \boldsymbol{\omega}}{\partial t}%
%        +\left(\boldsymbol{u}\cdot\nabla\right)\boldsymbol{\omega}=% 
        \frac{D\boldsymbol{\omega}}{Dt}=%
        \left(\boldsymbol{\omega}\cdot\nabla\right)\boldsymbol{u}\quad%
        -\boldsymbol{\omega}\left(\nabla\cdot\boldsymbol{u}\right)\,\,\,%
        +\frac{\nabla\rho\times\nabla p}{\rho^2}\,\,\,%
        -\nabla\times\left(\frac{\nabla\cdot\boldsymbol{\tau}}{\rho}\right)\quad%
        +\nabla\times\boldsymbol{B}%
        $%
      }
    }
    \only<3->{
      \scalebox{0.94}{$
%        \frac{\partial \boldsymbol{\omega}}{\partial t}%
%        +\left(\boldsymbol{u}\cdot\nabla\right)\boldsymbol{\omega}=% 
        \frac{D\boldsymbol{\omega}}{Dt}=%
        \cancelto{\sim 0}{\left(\boldsymbol{\omega}\cdot\nabla\right)\boldsymbol{u}}%
        \cancelto{\sim 0}{-\boldsymbol{\omega}\left(\nabla\cdot\boldsymbol{u}\right)}%
        +\alert{\frac{\boldsymbol{\nabla\rho\times\nabla p}}{\boldsymbol{\rho^2}}}%
        -\cancelto{\sim 0}{\nabla\times\left(\frac{\nabla\cdot\boldsymbol{\tau}}{\rho}\right)}%
        +\cancelto{\sim 0}{\nabla\times\boldsymbol{B}}%
        $
      }%
    }%
  }%
  \vfill 
%  \vspace*{0.25cm}%
  \only<2->{
    \begin{minipage}{\textwidth}
      \begin{minipage}{0.5\textwidth}
        \visible<4->{
          \footnotesize
          \begin{itemize}%
          \item Air-tissue interfaces have sharp density gradients%
            \vspace*{6pt}%
          \item US has strong pressure gradients%
            \vspace*{6pt}%
          \item US-induced baroclinic vorticity may cause strain, similar to shock-driven interfaces%
            \vspace*{6pt}%
          \item Linear acoustics does not capture this.
          \end{itemize}
        }
      \end{minipage}
      \begin{minipage}{0.5\textwidth}
        \begin{figure}
          \centering
          \begin{tikzpicture}%
            \node[anchor=south west,inner sep=0] (image) at (0,0) {%
              \def\svgwidth{\textwidth}
              {\footnotesize
                \import{../figs/lung_figs/}{usbe_lung_schematic2.pdf_tex} \hfill%
              }  
            };%
            \begin{scope}[x={(image.south east)},y={(image.north west)}]%
              \node[font=\footnotesize,right] at (0.6,0.72){ $\nabla p$};%
              \node[font=\footnotesize,right] at (0.59,0.515){ $\nabla \rho$};%
              \node[font=\tiny,right] at (0.7,0.05){\textcolor{gray}{wikimedia.org}};%
            \end{scope}%  
          \end{tikzpicture}%
        \end{figure}
      \end{minipage}
    \end{minipage}
    \note{
      {\tiny
        \begin{enumerate}
        \item In DUS-lung interaction, you have a nearly infinite density gradient where the tissue and lung meet within the alveolas and strong pressure gradients in the US.
        \item If we look at vorticity equation, which shows all of the
          ways that vorticity or local fluid rotation can be generated.
        \item Vorticity is curl of velocity field, can be obtained from curl of conservation of momentum equation.
          $\nabla \rho \times \nabla p$, which is generally small for
          acoustics $\orderof{\Delta u^2}$ has the potential to be quite
          large.
        \item For the model problems we'll be looking at, other appear less relevant.
        \item Gravity term is small based on Freud number calculation, so gravity will be neglected in our model problem.
        \item Acoustic viscous boundary length scale is micron order, which is small relative to 100 micron alveolar length scales.
        \item For our toy problem, we consider two-D, so vortex stretching term is identically zero, and expected to be of order $\Delta u^2$, velocity perturbation squared, in 3D anyway.
        \item compressible and advective terms also appear to be of order $\Delta u^2$.
        \item Linear acoustics won't capture this, and acoustic simulations may not get this right either because of huge $\nabla \rho$ is produces a second order compressible effect.
        \item Full nonlinear equations of motion needed.
        \end{enumerate}
      }
    }
  }
\end{frame}
%%% Local Variables:
%%% mode: latex
%%% TeX-master: "../main"
%%% End:
